% !TEX root = ../analisis.tex
 % El tipo de regla de negocio (tercer parámetro del entorno 'BusinessRule') se describe en la siguiente tabla:
%---------------------------------------------------------------------------------------------------------------,
% TIPOS			|		DEFINICION						|	EJEMPLO												|
%---------------|---------------------------------------|-------------------------------------------------------|
% Habilitador   | La sentencia habilita o restringe 	| * Se pueden recibir solicitudes del tipo A, B y C.	|
%				| hacer algo o  una funcionalidad.		| * Se permite hacer algo si se tiene el estado X.		|
%---------------|---------------------------------------|-------------------------------------------------------|
% Cronometrado	| Se permite de manera controlada 		| * Se permiten hasta dos solicitudes del tipo D		| 
%				| por un contador.						|   por persona.										|
% 				|										| * El acceso al sistema se permite si no se tiene 		|
%  				|										|   más de X número de intentos fallidos.				|
%---------------|---------------------------------------|-------------------------------------------------------|
% Ejecutivo		| Autorizado por un superior, un perfil | * Se permite registrar extemporaneamente si lo 		|
%				| particular debe autorizar.			|   autoriza X.											|
%---------------|---------------------------------------|-------------------------------------------------------|
% Derivación	| Son de cálculo e inferencia, 			| * Un alumno irregular es aquel que tiene las 			|
%				| es un cálculo o conclusión derivados 	|   siguientes cacteristicas: A, B, C. 					|
%				| de un conjunto de datos. Puede ser una| * El formato de un correo o CURP.						|
%				| fórmula que dice cómo calcular algo 	|														|
%				| o el formato de un dato.				|														|
%---------------|---------------------------------------|-------------------------------------------------------|
% Restricción	| Restringe una funcionalidad o relación| * Traslape de fechas o periodos empalmados.			|
% 				| entre dos o mas objetos.				|														|
%  				|										|														|
%---------------------------------------------------------------------------------------------------------------'

% No editar las reglas cuyo estatus es APROBADO.
\newpage
\section{Reglas de negocio}
\subsection{Reglas derivadas del sistema}
%------------------------------------------------------------------------------------------------------------------
\begin{BusinessRule}{RN-S1}{Información correcta}
	{Restricción}
	{Controla la operación}
	\BRitem{Versión}{1.0}
	\BRitem{Autor}{Adrian Flores Torres}
	\BRitem{Estatus}{Terminado}
	\BRitem{Descripción}{Todos los datos proporcionados al sistema deben pertenecer al \cdtRef{gls:tipoDato}{tipo de dato} especificado en el modelo de información y respetar el formato con base en lo definido en la entidad del diccionario de datos}
	\BRitem{Referenciado por}{ 
	}	
\end{BusinessRule}

\begin{BusinessRule}{RN-S2}{Formato de correo electrónico}
	{Restricción}
	{Controla la operación}
	\BRitem{Versión}{1.0}
	\BRitem{Autor}{Fabiola Jaramillo Loredo}
	\BRitem{Estatus}{Terminado}
	\item[Descripción:] El correo electrónico debe ser una cadena de caracteres con una estructura que cumpla la siguiente expresión regular:
	\begin{verbatim}
	\\b^['_a-z0-9-\\+](\\.['_a-z0-9-\\+])@[a-z0-9-](\\.[a-z0-9-])\\.([a-z]{2,6})\\b
	\end{verbatim}
	
	\BRitem{Referenciado por}{
		\cdtIdRef{CUPA1.3-2}{Crear cuenta}, 
		\cdtIdRef{CUPA1.4-5}{Registrar medios de contacto}
	}
\end{BusinessRule}

\begin{BusinessRule}{RN-S3}{Datos requeridos}
	{Restricción}
	{Controla la operación}
	\BRitem{Versión}{1.0}
	\BRitem{Autor}{Fabiola Jaramillo Loredo}
	\BRitem{Estatus}{Terminado}
	\BRitem{Descripción}{Los datos proporcionados al sistema que son marcados como 'requeridos' no se deben omitir}
	\BRitem{Referenciado por}{
		\cdtIdRef{CUPA1.1-3}{Editar ciclo escolar}, 
		\cdtIdRef{CUPA1.1-4}{Registrar ciclo escolar}
	}
\end{BusinessRule}

\subsection{Reglas derivadas del negocio}

%------------------------------------------------------------------------------------------------------------------

\begin{BusinessRule}{RN-N1}{Selección de Grupo}
	{Derivación}
	{Controla la operación}
	\BRitem{Versión}{1.0}
	\BRitem{Autor}{Eduardo Espino Maldonado}
	\BRitem{Estatus}{Edición}
	\item[Descripción:] 
	
	Para la conformación de un grupo de una Unidad de Aprendizaje se conformara por:
	
	$$ <NIVEL> C <TURNO> <GRUPO> $$
	
	en donde: \\
	
	\begin{UClist}
    
    	\UCli {\bf NIVEL:} El \cdtRef{unidadaprendizaje:nivel}{Nivel} seleccionado.
	
    	\UCli {\bf TURNO:} El \textbf{Turno} seleccionado.
	
	\UCli {\bf GRUPO:} El \textbf{Grupo} seleccionado.
	    	    
  	\end{UClist}

	\BRitem{Referenciado por}{
		\cdtIdRef{CUH1.1}{Registrar Materia}
		}
\end{BusinessRule}