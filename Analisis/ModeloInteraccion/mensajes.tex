\section{Diseño de mensajes}
%\DONE{} 
	En esta sección se describen los mensajes utilizados en el prototipo actual del sistema. Los mensajes se refieren a todos
	aquellos avisos que el sistema muestra al actor a través de la pantalla debido a diversas
	razones, por ejemplo: informar acerca de algún fallo en el sistema o para notificar acerca de alguna operación importante sobre
	la información.

%===========================================================removiendo los puntos generados
\section{Parámetros comunes}
    Cuando un mensaje es recurrente se parametrizan sus elementos, por ejemplo los mensajes: ``Aún no existen registros de {\em escuelas} en el sistema.'', ``Aún no existen registros de {\em responsables del programa} en el sistema.'', 
    ``Aún no existen registros de {\em integrantes de líneas de acción} en el sistema.'', tienen una estructura similar 
    por lo que para definir el mensaje se utilizan parámetros, con el objetivo de que el mensaje sea genérico y  
    pueda utilizarse en todos los casos que se considere necesario.\\
    
    Los parámetros también se utilizan cuando la redacción del mensaje tiene datos que son ingresados por el actor o que dependen del resultado de la operación, por ejemplo: 
    ``La {\em escuela  15DPR2497K} ha sido {\em modificada} exitosamente.''. En este caso la redacción se presenta parametrizada de la forma: ``DETERMINADO ENTIDAD VALOR ha sido OPERACIÓN exitosamente.'' y los 
    parámetros se describen de la siguiente forma:
    
    \begin{itemize}
	\item DETERMINADO ENTIDAD: Es un artículo determinado más el nombre de la entidad sobre la cual se realizó la acción.
	\item VALOR: Es el valor asignado al atributo de la entidad, generalmente es el nombre o la clave.
	\item OPERACIÓN: Es la acción que el actor solicitó realizar.
    \end{itemize}

    En el ejemplo anterior se hace referencia a VALOR, es decir: {\em 15DPR2497K} es el {\bf valor}  de la entidad {\bf escuela}. Cada mensaje enlista los parámetros 
    que utiliza, sin embargo aquí se definen los más comunes a fin de simplificar la descripción de los mensajes:

    \begin{description}
	\item [ARTÍCULO:] Se refiere a un {\em artículo} el cual puede ser DETERMINADO (El $\mid$ La $\mid$ Lo $\mid$ Los $\mid$ Las) o INDETERMINADO (Un $\mid$ Una $\mid$ 
	Uno $\mid$ Unos $\mid$Unas) se aplica generalmente sobre una ENTIDAD, ATRIBUTO o VALOR.
	\item [CAMPO:] Se refiere a un campo del formulario. Por lo regular es el nombre de un atributo en una entidad.
	\item [CONDICIÓN:] Define una expresión booleana cuyo resultado deriva en {\em falso} o {\em verdadero} y suele ser la causa del mensaje.
	\item [DATO:] Es un sustantivo y generalmente se refiere a un atributo de una entidad descrito en el modelo estructural del negocio, por ejemplo: número de incendio,
	brigada de apoyo del incendio, uso de suelo autorizado del predio, etc. %ATRIBUTO
	\item [ENTIDAD:] Es un sustantivo y generalmente se refiere a una entidad del modelo estructural del negocio, por ejemplo: incendio, pago por servicios ambientales hidrológicos, reforestación, etc.
	\item [OPERACIÓN:] Se refiere a una acción que se debe realizar sobre los datos de una o varias entidades. Por ejemplo: registrar, eliminar, actualizar, etc. Comúnmente 
	la OPERACIÓN va concatenada con el sustantivo, por ejemplo: Registro de un nuevo beneficio, registro de una actividad, eliminar una tarea, etc.
	\item [VALOR:] Es un sustantivo concreto y generalmente se refiere a un valor en específico. Por ejemplo: ``2014-003'', que es un valor concreto del DATO de la 
	ENTIDAD ``incendio''.
	\item [TAMAÑO:] Es el tamaño del atributo de una entidad, el cual se encuentra definido en el diccionario de datos.
	\item [MOTIVO:] Es una explicación acerca de la operación que se pretende realizar.
    \end{description}\begin{UClist}
    \UCli Tabla que muestra: Egresados.
    \UCli Tabla que muestra: Profesores.
\end{UClist}


\section{Mensajes a través de la pantalla}



%===========================  MSG1 ==================================
\begin{mensaje}{MSG1}{Operación exitosa}{Notificación}
	\item[Ubicación:] \msjSuperior
	\item[Estatus:] Terminado
	\item[Objetivo:] Notificar al actor que la acción solicitada fue realizada exitosamente.
	\item[Redacción:] DETERMINADO ENTIDAD VALOR se OPERACIÓN exitosamente.
	\item[Parámetros:] El mensaje se muestra con base en los siguientes parámetros:
	\begin{Citemize} 
		\item DETERMINADO ENTIDAD: Es un artículo determinado más el nombre de la entidad sobre la cual se realizó la acción.
		\item VALOR: Es el valor asignado al atributo de la entidad, generalmente es el nombre o la clave.
		\item OPERACIÓN: Es la acción que el actor solicitó realizar redactada en pasado.
	\end{Citemize}
	\item[Ejemplo:] El aspirante Adriana Suárez Cruz se registró exitosamente.

	\item[Referenciado por:] 
	\cdtIdRef{CUPA1.1-2}{Gestionar actividades del ciclo escolar},
	\cdtIdRef{CUPA1.1-3}{Editar ciclo escolar},
	\cdtIdRef{CUPA1.1-4}{Registrar ciclo escolar},
	\cdtIdRef{CUPA1.1-5}{Editar actividad de la etapa},
	\cdtIdRef{CUPA1.1-6}{Registrar actividad de la etapa},
	\cdtIdRef{CUPA1.1-9}{Agregar actividad al ciclo escolar}, 
	\cdtIdRef{CUPA1.1-12}{Registrar etapa}, 
	\cdtIdRef{CUPA1.1-13}{Eliminar etapa}, 
	\cdtIdRef{CUPA1.1-14}{Editar actividad del ciclo escolar},
	\cdtIdRef{CUPA1.1-21}{Establecer calendario escolar como publicado},  
	\cdtIdRef{CUGA7.1-2}{Registrar grupo}, 
	\cdtIdRef{CUGA7.1-3}{Editar grupo}, 
	\cdtIdRef{CUGA7.1-4}{Eliminar grupo}, 
	\cdtIdRef{CUGA7.1.1-2}{Agregar horario de grupo}, \cdtIdRef{CUGA7.1.1-3}{Eliminar horario de grupo}, \cdtIdRef{CUGA7.1.1-4}{Editar horario de grupo}, \cdtIdRef{CUGA7.2-2}{Registrar salón}, 
	\cdtIdRef{CUGA7.2-3}{Editar salón}, 
	\cdtIdRef{CUGA7.2-4}{Eliminar salón}, 
	\cdtIdRef{CUGA7.3-2}{Registrar profesor}, 
	\cdtIdRef{CUGA7.3-2-1}{Crear cuenta}, 
	\cdtIdRef{CUGA7.3-3}{Editar profesor}, 
	\cdtIdRef{CUGA7.3-4}{Eliminar profesor},
	\cdtIdRef{CUGA7.3-6}{Registrar domicilio},
	\cdtIdRef{CUGA7.3-7}{Editar domicilio}, 
	\cdtIdRef{CUGA7.3-8}{Gestionar medios de contacto}, 
	\cdtIdRef{CUGA7.3-10}{Registrar historial académico}, \cdtIdRef{CUGA7.3-11}{Registrar trayectoria docente}, \cdtIdRef{CUGA7.3-13}{Cargar identificación}, \cdtIdRef{CUGA7.3-14}{Eliminar domicilio}, \cdtIdRef{CUGA7.3-15}{Eliminar medio de contacto}, \cdtIdRef{CUGA7.3-16}{Eliminar Historial Académico}, \cdtIdRef{CUGA7.3-17}{Eliminar trayectoria profesional}, \cdtIdRef{CUGA7.3-19}{Cambiar contraseña}, \cdtIdRef{CUGA7.4-3-3}{Eliminar ausencia temporal}, \cdtIdRef{CUPA1.1-2}{Gestionar actividades del ciclo escolar}, \cdtIdRef{CUPA1.1-6}{Registrar actividad}, \cdtIdRef{CUPA1.2-3}{Registrar criterio}, 
	\cdtIdRef{CUPA1.2-5}{Agregar requisito}, 
	\cdtIdRef{CUPA1.2-6}{Editar requisito}, 
	\cdtIdRef{CUPA1.2-7}{Eliminar requisito},
	\cdtIdRef{CUPA1.2-12}{Visualizar convocatoria de ingreso},
	\cdtIdRef{CUPA1.2-13}{Aprobar convocatoria de ingreso},
	\cdtIdRef{CUPA1.2-16}{Eliminar periodo},
	\cdtIdRef{CUPA1.2-17}{Configurar convocatoria de ingreso},
	\cdtIdRef{CUPA1.2-18}{Agregar actividad a convocatoria de ingreso},
	\cdtIdRef{CUPA1.2-19}{Editar actividad de convocatoria de ingreso},
	\cdtIdRef{CUPA1.2-20}{Registrar convocatoria de ingreso},
	\cdtIdRef{CUPA1.2-21}{Editar convocatoria de ingreso},
	\cdtIdRef{CUPA1.2-22}{Eliminar actividad de convocatoria de ingreso},
	\cdtIdRef{CUPA1.2-24}{Eliminar convocatoria de ingreso},
	\cdtIdRef{CUPA1.2-25}{Habilitar edición de convocatoria de ingreso},
	\cdtIdRef{CUPA1.2-26}{Cerrar convocatoria de ingreso},
	\cdtIdRef{CUPA1.4-2}{Eliminar medio de contacto},
	\cdtIdRef{CUPA1.4-3}{Registrar datos personales},
	\cdtIdRef{CUPA1.4-4}{Registrar domicilio},
	\cdtIdRef{CUPA1.4-5}{Registrar medios de contacto},
	\cdtIdRef{CUPA1.4-7}{Registrar información escolar},
	\cdtIdRef{CUPA1.9-1}{Configurar lista de aspirantes},
	\cdtIdRef{CUPA1.9-2}{Agregar aspirantes a la lista de aceptados},
	\cdtIdRef{CUPA1.9-4}{Eliminar aspirante de la lista de aceptados},
	\cdtIdRef{CUPA1.6-1}{Prerregistrar aspirantes},
	\cdtIdRef{CUPA1.6-4}{Registrar resultados},
	\cdtIdRef{CUPA1.6-7}{Eliminar aspirante},
	\cdtIdRef{CUPA1.7-2}{Reservar salones},
	\cdtIdRef{CUPA1.7-4}{Registrar psicólogo},
	\cdtIdRef{CUPA1.7-5}{Editar psicólogo},
	\cdtIdRef{CUPA1.7-6}{Eliminar psicólogo}
	\cdtIdRef{CUPA1.7-7}{Asignar aspirantes a evaluar},
	\cdtIdRef{CUPA1.7-9}{Registrar vigencia},
	\cdtIdRef{CUPA1.7-10}{Editar vigencia},
	\cdtIdRef{CUPA1.711}{Eliminar vigencia},
	\cdtIdRef{CUPA1.7-15}{Registrar contraseña},
	\cdtIdRef{CUPA1.7-17}{Registrar concepto},
	\cdtIdRef{CUPA1.7-18}{Editar concepto},
	\cdtIdRef{CUPA1.7-19}{Eliminar concepto},
	\cdtIdRef{CUPA1.7-21}{Registrar entrevista},
	\cdtIdRef{CUPA1.7-22}{Registrar examen electrónico},
	\cdtIdRef{CUPA1.7-24}{Aprobar evaluación final},
	\cdtIdRef{CUPA1.7-25}{Crear cuenta de psicólogo},
	\cdtIdRef{CUPA1.7-26}{Editar cuenta del psicólogo}.
\end{mensaje}

%============================== MSG2 =================================
\begin{mensaje}{MSG2}{Cancelar operación}{Notificación}
	\item[Ubicación:] \msjEmergente
	\item[Estatus:] Terminado
	\item[Objetivo:] Preguntar al actor si desea cancelar esta operación.
	\item[Redacción:] ¿Está seguro que desea cancelar la operación?
	\item[Referenciado por:] \cdtIdRef{CUPA1.1-3}{Editar ciclo}, \cdtIdRef{CUPA1.1-4}{Registrar ciclo escolar}, \cdtIdRef{CUPA1.3-2}{Crear cuenta}, 
	\cdtIdRef{CUPA1.9-1}{Configurar lista de aspirantes}.
\end{mensaje}

%============================== MSG3 =================================
\begin{mensaje}{MSG3}{Operación fallida}{Notificación}
	\item[Ubicación:] \msjSuperior, \msjEmergente
	\item[Estatus:] Terminado
	\item[Objetivo:] Notificar al actor que una operación no se pudo llevar a cabo.
	\item[Redacción:] A continuación se listan las posibles redacciones que este mensaje puede tener dependiendo de lo especificado en el diccionario de datos:
	\begin{enumerate}
		\item No se ha podido realizar DETERMINADO OPERACIÓN. Vuelva a intentarlo.
		\item No se puede realizar esta acción.
	\end{enumerate}

	\item[Parámetros:] El mensaje se muestra con base en los siguientes parámetros:
	\begin{itemize}
		\item DETERMINADO OPERACIÓN: Es al artículo determinado más el nombre de la operación que no se pudo llevar a cabo.
	\end{itemize}
	\item[Ejemplo:] No se ha podido realizar el registro. Vuelva a intentarlo.
	\item[Referenciado por:] \cdtIdRef{CUGA7.1-2}{Registrar grupo},
	\cdtIdRef{CUPA1.1-3}{Editar ciclo escolar},
	\cdtIdRef{CUPA1.1-4}{Registrar ciclo escolar},
	\cdtIdRef{CUPA1.1-5}{Editar actividad de la etapa},
	\cdtIdRef{CUPA1.1-6}{Registrar actividad de la etapa},
	\cdtIdRef{CUPA1.1-7}{Eliminar ciclo escolar},
	\cdtIdRef{CUPA1.1-8}{Eliminar actividad},
	\cdtIdRef{CUPA1.1-9}{Agregar actividad al ciclo escolar},
	\cdtIdRef{CUPA1.1-12}{Registrar etapa}, 
	\cdtIdRef{CUPA1.1-14}{Editar actividad del ciclo escolar}, 
	\cdtIdRef{CUPA1.1-21}{Establecer calendario escolar como publicado},
	\cdtIdRef{CUGA7.1-3}{Editar grupo}, 
	\cdtIdRef{CUGA7.1-4}{Eliminar grupo}, 
	\cdtIdRef{CUGA7.1.1-2}{Agregar horario de grupo}, \cdtIdRef{CUGA7.1.1-3}{Eliminar horario de grupo}, \cdtIdRef{CUGA7.1.1-4}{Editar horario de grupo}, \cdtIdRef{CUGA7.2-2}{Registrar salón}, 
	\cdtIdRef{CUGA7.2-3}{Editar salón}, 
	\cdtIdRef{CUGA7.2-4}{Eliminar salón}, 
	\cdtIdRef{CUGA7.3-2}{Registrar profesor},
	\cdtIdRef{CUGA7.3-2-1}{Crear cuenta},
	\cdtIdRef{CUGA7.3-3}{Editar profesor}, 
	\cdtIdRef{CUGA7.3-4}{Eliminar profesor}, 
	\cdtIdRef{CUGA7.3-6}{Registrar domicilio}, 
	\cdtIdRef{CUGA7.3-7}{Editar domicilio},
	\cdtIdRef{CUGA7.3-8}{Gestionar medios de contacto}, \cdtIdRef{CUGA7.3-10}{Registrar historial académico}, \cdtIdRef{CUGA7.3-11}{Registrar trayectoria docente}, \cdtIdRef{CUGA7.3-13}{Cargar identificación}, \cdtIdRef{CUGA7.3-14}{Eliminar domicilio}, \cdtIdRef{CUGA7.3-15}{Eliminar medio de contacto}, \cdtIdRef{CUGA7.3-16}{Eliminar Historial Académico}, \cdtIdRef{CUGA7.3-17}{Eliminar trayectoria profesional}, \cdtIdRef{CUGA7.3-18}{Definir prioridad de domicilio}, \cdtIdRef{CUGA7.3-19}{Cambiar contraseña}, \cdtIdRef{CUGA7.4-3-3}{Eliminar ausencia temporal}, \cdtIdRef{CUPA1.1-6}{Registrar actividad}, 
	\cdtIdRef{CUPA1.2-5}{Agregar requisito}, 
	\cdtIdRef{CUPA1.2-6}{Editar requisito}, 
	\cdtIdRef{CUPA1.2-7}{Eliminar requisito}, \cdtIdRef{CUPA1.2-12}{Visualizar convocatoria de ingreso},
	\cdtIdRef{CUPA1.2-13}{Aprobar convocatoria de ingreso},
	\cdtIdRef{CUPA1.2-14}{Asociar criterio}, 
	\cdtIdRef{CUPA1.2-15}{Configurar criterio},
	\cdtIdRef{CUPA1.2-16}{Eliminar periodo},
	\cdtIdRef{CUPA1.2-17}{Configurar convocatoria de ingreso},
	\cdtIdRef{CUPA1.2-18}{Agregar actividad a convocatoria de ingreso},
	\cdtIdRef{CUPA1.2-19}{Editar actividad de convocatoria de ingreso},
	\cdtIdRef{CUPA1.2-20}{Registrar convocatoria de ingreso},
	\cdtIdRef{CUPA1.2-21}{Editar convocatoria de ingreso},
	\cdtIdRef{CUPA1.2-22}{Eliminar actividad de convocatoria de ingreso},
	\cdtIdRef{CUPA1.2-24}{Eliminar convocatoria de ingreso},
	\cdtIdRef{CUPA1.2-25}{Habilitar edición de convocatoria de ingreso},
	\cdtIdRef{CUPA1.2-26}{Cerrar convocatoria de ingreso},
	\cdtIdRef{CUPA1.4-2}{Eliminar medio de contacto},
	\cdtIdRef{CUPA1.4-4}{Registrar domicilio},
	\cdtIdRef{CUPA1.4-5}{Registrar medios de contacto},
	\cdtIdRef{CUPA1.4-7}{Registrar información escolar},
	\cdtIdRef{CUPA1.6-1}{Prerregistrar aspirantes},
	\cdtIdRef{CUPA1.6-2}{Reservar salones},
	\cdtIdRef{CUPA1.6-4}{Registrar resultados},
	\cdtIdRef{CUPA1.9-1}{Configurar lista de aspirantes},
	\cdtIdRef{CUPA1.9-2}{Agregar aspirantes a la lista de aceptados},
	\cdtIdRef{CUPA1.9-4}{Eliminar aspirante de la lista de aceptados},
	\cdtIdRef{CUIR3.2-2}{Visualizar horario},
	\cdtIdRef{CUPA1.6-4}{Registrar resultados},
	\cdtIdRef{CUPA1.6-7}{Eliminar aspirante},
	\cdtIdRef{CUPA1.7-2}{Reservar salones},
	\cdtIdRef{CUPA1.7-4}{Registrar psicólogo},
	\cdtIdRef{CUPA1.7-5}{Editar psicólogo},
	\cdtIdRef{CUPA1.7-6}{Eliminar psicólogo},
	\cdtIdRef{CUPA1.7-7}{Asignar aspirantes a evaluar},
	\cdtIdRef{CUPA1.7-9}{Registrar vigencia},
	\cdtIdRef{CUPA1.7-10}{Editar vigencia},
	\cdtIdRef{CUPA1.711}{Eliminar vigencia},
	\cdtIdRef{CUPA1.7-12}{Consultar asignación de psicólogos},
	\cdtIdRef{CUPA1.7-13}{Consultar asignación},
	\cdtIdRef{CUPA1.7-15}{Registrar contraseña},
	\cdtIdRef{CUPA1.7-17}{Registrar concepto},
	\cdtIdRef{CUPA1.7-18}{Editar concepto},
	\cdtIdRef{CUPA1.7-19}{Eliminar concepto},
	\cdtIdRef{CUPA1.7-21}{Registrar entrevista},
	\cdtIdRef{CUPA1.7-22}{Registrar examen electrónico},
	\cdtIdRef{CUPA1.7-25}{Crear cuenta de psicólogo},
	\cdtIdRef{CUPA1.7-26}{Editar cuenta del psicólogo}.
\end{mensaje}


%============================== MSG4 =================================
\begin{mensaje}{MSG4}{Correos electrónicos enviados}{Notificación}
	\item[Ubicación:] \msjSuperior
	\item[Estatus:] Terminado
	\item[Objetivo:] Notificar al actor que se han enviado correos electrónicos.
	\item[Redacción:] Los correos electrónicos de ASUNTO han sido enviados a DESTINATARIO.
	\item[Parámetros:] El mensaje se muestra con base en los siguientes parámetros:
		\begin{itemize}
			\item ASUNTO: Es el motivo del envío del correo electrónico.
			\item DESTINATARIO: Es el destinatario del correo electrónico.
		\end{itemize}
	\item[Ejemplo:] Los correos electrónicos de notificación de entrevista han sido enviados a los aspirantes seleccionados.
	\item[Referenciado por:] \cdtIdRef{CUPA1.8.1-1}{Generar lista de aspirantes para entrevistar},
	\cdtIdRef{CUPA1.6-3}{Visualizar aspirantes con pre-registro sin encuesta}.
\end{mensaje}

%============================== MSG5 =================================
\begin{mensaje}{MSG5}{Fecha condicionada a un período}{Error}
	\item[Ubicación:] \msjEmergente, \msjCampo, \msjSuperior
	\item[Estatus:] Terminado
	\item[Objetivo:] Notificar al actor que la fecha de una operación está fuera del período permitido.
	\item[Redacción:] La fecha de ACTIVIDAD está CONDICIÓN del periodo PREPOSICIÓN PERÍODO.
	\item[Parámetros:] El mensaje se muestra con base en los siguientes parámetros:
	\begin{itemize}
		\item ACTIVIDAD: Es la acción cuya fecha no cumple con la condición del período.
		\item CONDICIÓN: Es la condición que debe satisfacer la fecha de la actividad.
		\item PERÍODO: Es el intervalo de tiempo definido previamente con formato día, mes y año.
		\item PREPOSICIÓN: Es la preposición que puede ser utilizada en la oración y puede tener los siguientes valores: de y por.
	\end{itemize}
	\item[Ejemplo:] A continuación se muestran algunos ejemplos: 
	\begin{enumerate}
		\item La fecha de alguna actividad está fuera del periodo del ciclo escolar.
		\item La fecha de solicitud está fuera del periodo de registro.
		\item La fecha de reserva está fuera del periodo para configurar citas de psicométrico.
		\item La fecha de aplicación de entrevistas está fuera del periodo del ciclo escolar.
		\item La fecha de solicitud está fuera del periodo de registro CENEVAL.
	
\end{enumerate}
	
	\item[Referenciado por:] 
	\cdtIdRef{CUPA1.1-2}{Gestionar actividades del ciclo escolar}, \cdtIdRef{CUPA1.1-9}{Agregar actividad al ciclo escolar},
	\cdtIdRef{CUPA1.4}{Registrar Aspirante}, 
	\cdtIdRef{CUPA1.1-5}{Editar actividad de la etapa}, 
	\cdtIdRef{CUPA1.1-14}{Editar actividad del ciclo escolar},
	\cdtIdRef{CUPA1.2-18}{Agregar actividad a convocatoria de ingreso},
	\cdtIdRef{CUPA1.6-3}{Visualizar aspirantes sin con pre-registro sin encuesta},
	\cdtIdRef{CUPA1.2-19}{Editar actividad de convocatoria de ingreso},
	\cdtIdRef{CUPA1.9-1}{Configurar lista de aspirantes},
	\cdtIdRef{CUPA1.7-2}{Reservar Salones},
\end{mensaje}

%%============================== MSG6 =================================
%\begin{mensaje}{MSG6}{Incumplimiento de promedio}{Notificación}
%	\item[Ubicación:] \msjSuperior
%	\item[Estatus:] Terminado
%	\item[Objetivo:] Notificar al actor que su promedio no cumple con el promedio especificado así como la consecuencia de este hecho.
%	\item[Redacción:] Su promedio es menor que el promedio TIPO por lo cual CONSECUENCIA.
%	\item[Parámetros:] El mensaje se muestra con base en los siguientes parámetros:
%	\begin{itemize}
%		\item TIPO: Es el tipo de promedio con el cuál se realiza la comparación. Puede ser mínimo o solicitado.
%		\item CONSECUENCIA: Es la acción que resulta del incumplimiento de promedio.
%	\end{itemize}
%	\item[Ejemplo:] Su promedio es menor que el promedio solicitado por lo cual se le permitirá continuar con tu proceso de admisión, pero las probabilidades de ingresar a la ELD son bajas.
%	\item[Referenciado por:] \cdtIdRef{CUPA1.4-5}{Registrar medios de contacto},
%	\cdtIdRef{CUPA1.4-7}{Registrar información escolar}.
%\end{mensaje}
%
%%============================== MSG7 =================================
%\begin{mensaje}{MSG7}{Selección de elementos incorrecta}{Error}
%	\item[Ubicación:] \msjEmergente, \msjCentro, \msjSuperior
%	\item[Estatus:] Terminado
%	\item[Objetivo:] Notificar al actor que el número de elementos seleccionados es incorrecto.
%	\item[Redacción:] Debe seleccionar CONDICIÓN NÚMERO ELEMENTO.
%	\item[Parámetros:] El mensaje se muestra con base en los siguientes parámetros:
%	\begin{itemize}
%		\item CONDICIÓN: Es la condición que debe satisfacer el NÚMERO de ELEMENTO. Puede ser como mínimo, a lo más o exactamente.
%		\item NÚMERO: Es el número de ELEMENTO.
%		\item ELEMENTO: Es el nombre del elemento que se seleccionó.
%	\end{itemize}
%	\item[Ejemplo:]
%	\begin{itemize}
%		\item Debe seleccionar como mínimo 2 criterios.
%		\item Debe seleccionar exactamente 2 materias optativas.
%		\item Debe seleccionar como mínimo 1 salón.
%	\end{itemize}		
%	\item[Referenciado por:] \cdtIdRef{CUPA1.2-14}{Asociar criterio},
%	\cdtIdRef{CUPA1.4-3}{Registrar datos personales},
%	\cdtIdRef{CUPA1.4-7}{Registrar información escolar},
%	\cdtIdRef{CUIR3.2-2}{Visualizar Horario},
%	\cdtIdRef{CUPA1.7-7}{Asignar Aspirantes a Evaluar},
%	\cdtIdRef{CUPA1.7-2}{Reservar Salones},
%	\cdtIdRef{CUPA1.7-7}{Asignar aspirantes a evaluar},
%	\cdtIdRef{CUPA1.7-22}{Registrar examen electrónico}.
% 
%\end{mensaje}
%
%%============================== MSG8 =================================
%\begin{mensaje}{MSG8}{Elementos sin registrar}{Error}
%	\item[Ubicación:] \msjCentro.
%	\item[Estatus:] Edición.
%	\item[Objetivo:] Notificar al actor que faltan elementos por registrar para continuar con la operación solicitada.
%	\item[Redacción:] Primero debe registrar ARTICULO ELEMENTOS para continuar con la operación solicitada.
%	\item[Parámetros:] El mensaje se muestra con base en los siguientes parámetros:
%	\begin{Citemize} 
%		\item ARTICULO: Es la parte de la oración que se ocupa de expresar el género (masculino/femenino).
%		\item ELEMENTOS: Es el nombre del elemento sobre el cuál se está realizando la operación.
%	\end{Citemize}
%	\item[Ejemplo:] A continuación se muestran ejemplos de la redacción:
%	\begin{itemize}
%		\item Primero debe registrar los grupos faltantes para poder continuar con la operación solicitada.
%		\item Primero debe registrar los planes de estudios vigentes faltantes para poder continuar con la operación solicitada.
%		\item Primero debe registrar los planes de estudios liquidados o derogados faltantes para poder continuar con la operación solicitada.
%	\end{itemize}
%	\item[Referenciado por:] 
%\end{mensaje}
%
%%============================== MSG9 =================================
%\begin{mensaje}{MSG9}{Información de cita del aspirante}{Notificación}
%	\item[Ubicación:] \msjCorreo, \msjCentro
%	\item[Estatus:] Terminado
%	\item[Objetivo:] Notificar al actor que su cita fue programada el día y hora seleccionada.
%	\item[Redacción:] NOMBRE, su cita de OPERACIÓN fue programa para el FECHA a las HORA, en el LUGAR de la ELD. Puede realizar el cambio de su cita mientras haya horarios disponibles.
%	\item[Parámetros:] El mensaje se muestra con base en los siguientes parámetros:
%		\begin{Citemize} 
%			\item NOMBRE: Es el nombre completo del destinatario del correo.
%			\item OPERACIÓN: Es el motivo de la cita.
%			\item FECHA: Es la fecha de la cita.
%			\item HORA: Es la hora de la cita.
%			\item LUGAR: Es la ubicación donde la cita tendrá lugar.
%		\end{Citemize}
%	\item[Ejemplo:] Bruno Suárez Cruz, su cita de entrevista fue programa para el 05/05/17 a las 18:00 hrs, en el salón 10 de la ELD. Puede realizar el cambio de su cita mientras haya horarios disponibles.
%	\item[Referenciado por:] 
%\end{mensaje}
%
%
%%============================== MSG10 =================================
%\begin{mensaje}{MSG10}{Faltan medios de contacto por registrar}{Error}
%	\item[Ubicación:] \msjEmergente
%	\item[Estatus:] Edición
%	\item[Objetivo:] Notificar al actor los medios de contacto mínimos que debe registrar.
%	\item[Redacción:] Debe registrar como mínimo MEDIOS DE CONTACTO.
%	\item[Parámetros:] El mensaje se muestra con base en los siguientes parámetros:
%	\begin{Citemize} 
%		\item MEDIOS DE CONTACTO: Es la lista de medios de contacto que el actor debe registrar de forma obligatoria.
%	\end{Citemize}
%	\item[Ejemplo:]	Debe registrar como mínimo un número local, un celular y un contacto de emergencia.
%	\item[Referenciado por:] \cdtIdRef{CUPA1.4-5}{Registrar medios de contacto}, 
%	\cdtIdRef{CUGA7.3-8}{Gestionar medios de contacto}.
%\end{mensaje}
%
%
%%============================== MSG11 =================================
%\begin{mensaje}{MSG11}{Faltan valores de ponderación}{Error}
%	\item[Ubicación:] \msjSuperior
%	\item[Estatus:] Edición
%	\item[Objetivo:] Notificar al actor que la suma de las ponderaciones no cubren un 100\%.
%	\item[Redacción:] La suma de las ponderaciones de los criterios debe ser igual a 100\%.
%	\item[Referenciado por:] \cdtIdRef{CUPA1.2-2}{Gestionar criterios}, \cdtIdRef{CUPA1.4-5}{Registrar medios de contacto},
%	\cdtIdRef{CUPA1.2-12}{Visualizar convocatoria de ingreso}.
%\end{mensaje}
%
%
%%============================== MSG12 =================================
%\begin{mensaje}{MSG12}{Confirmación de pago}{Notificación}
%	\item[Ubicación:] \msjCorreo, \msjSuperior
%	\item[Estatus:] Terminado
%	\item[Objetivo:] Notificar al actor el estado del pago realizado. 
%	\item[Redacción:] NOMBRE, ha concluido con la etapa de Pago de Derechos satisfactoriamente. \\ 
%	Para continuar con la etapa de Evaluaciones, es indispensable que ingrese al sistema a partir del día FECHA DE INICIO hasta el día FECHA DE FIN, de lo contrario su proceso será cancelado.
%	\item[Parámetros:] El mensaje se muestra con base en los siguientes parámetros:
%	\begin{Citemize} 
%		\item NOMBRE: Es el nombre completo del aspirante.
%		\item FECHA DE INICIO: Es la fecha de inicio del periodo donde las evaluaciones estarán disponibles, esta fecha estará definida en la convocatoria de ingreso.
%		\item FECHA DE FIN: Es la fecha de fin del periodo donde las evaluaciones estarán disponibles, esta fecha estará definida en la convocatoria de ingreso.
%	\end{Citemize}
%	
%	\item[Ejemplo:] Bruno Suárez Cruz, ha concluido con la etapa de Pago de Derechos satisfactoriamente. \\ 
%	Para continuar con la etapa de Evaluaciones, es indispensable que ingrese al sistema a partir del día 05/mar/2017 hasta el día 05/abr/2017, de lo contrario su proceso será cancelado.
%	\item[Referenciado por:] \cdtIdRef {CUPA1.5-2}{Realizar pago con tarjeta}
%	
%\end{mensaje}
%
%%============================== MSG13 =================================
%\begin{mensaje}{MSG13}{Característica de archivo incorrecta}{Error}
%	\item[Ubicación:] \msjEmergente, \msjSuperior, \msjCampo
%	\item[Estatus:] Edición
%	\item[Objetivo:] Notificar al actor que el archivo seleccionado no cumple con las especificaciones solicitadas.
%	\item[Redacción:] A continuación se listan las posibles redacciones que este mensaje puede tener dependiendo de lo especificado en el diccionario de datos:
%	\begin{itemize}
%		\item El formato del archivo debe ser FORMATO.
%		\item El tamaño máximo del archivo debe ser de NÚMERO UNIDADES.
%		\item El archivo debe tener la siguiente estructura: ESTRUCTURA.
%	
%	\end{itemize}
%	\item[Parámetros:] El mensaje se muestra con base en los siguientes parámetros:
%	\begin{itemize}
%		\item FORMATO: Es el formato que debe cumplir el archivo que se desea subir. Puede ser .pdf, .zip, .xls 
%		\item NÚMERO: Es el número con el que se debe cumplir la condición.
%		\item UNIDADES: Unidad de información utilizada para medir el archivo. Pueden ser KB, MB o GB.
%		\item ESTRUCTURA: Nombre de las columnas y el orden en que deben estar en el archivo.
%	\end{itemize}
%	\item[Ejemplo:] 
%	\begin{enumerate}
%		\item El tamaño máximo del archivo debe ser de 2 MB.
%		\item El formato del archivo debe ser xls.
%		\item El archivo debe tener la siguiente estructura: Folio ELD, Folio CENEVAL, ICNE.
%		\item El formato del archivo debe ser jpg.
%		\item El tamaño máximo del archivo debe ser de 2.5 MB.
%	\end{enumerate}
%	\item[Referenciado por:] \cdtIdRef{CUGA7.3-13}{Cargar identificación},
%	\cdtIdRef{CUPA1.4-3}{Registrar datos personales},
%	\cdtIdRef{CUPA1.4-7}{Registrar información escolar},
%	\cdtIdRef{CUPA1.7-22}{Registrar examen electrónico},
%	\cdtIdRef{CUPA1.6-4}{Registrar resultados},
%	\cdtIdRef{CUPA1.7-22}{Registrar examen electrónico},
%	\cdtIdRef{CUGA7.3-2}{Registrar profesor},
%	\cdtIdRef{CUGA7.3-3}{Editar profesor}.
%\end{mensaje}
%
%
%%%============================== MSG13 =================================
%%\begin{mensaje}{MSG13}{Característica de archivo incorrecta}{Error}
%%	\item[Ubicación:] \msjEmergente, \msjSuperior, \msjCampo
%%	\item[Estatus:] Terminado
%%	\item[Objetivo:] Notificar al actor que el archivo seleccionado no cumple con las especificaciones solicitadas.
%%	\item[Redacción:] DETERMINADO CARACTERÍSTICA del archivo seleccionado no cumple con las especificaciones solicitadas.
%%	\item[Parámetros:] El mensaje se muestra con base en los siguientes parámetros:
%%	\begin{itemize}
%%		\item DETERMINADO CARACTERÍSTICA: Es la característica del archivo que no cumple con las especificaciones solicitadas. Puede ser tamaño, formato, estructura y nombre.
%%	\end{itemize}
%%	\item[Ejemplo:] El tamaño del archivo seleccionado no cumple con las especificaciones solicitadas.
%%	\item[Referenciado por:] \cdtIdRef{CUGA7.3-13}{Cargar identificación},
%%	\cdtIdRef{CUPA1.4-3}{Registrar datos personales},
%%	\cdtIdRef{CUPA1.4-7}{Registrar información escolar}. 
%%\end{mensaje}
%
%
%%============================== MSG14 =================================
%\begin{mensaje}{MSG14}{Eliminar elemento}{Confirmación}
%	\item[Ubicación:] \msjEmergente.
%	\item[Estatus:] Terminado
%	\item[Objetivo:] Notificar al actor que está a punto de eliminar un elemento y que se necesita su aprobación para ello.
%	\item[Redacción:] ¿Desea eliminar ARTICULO + ELEMENTO + SELECCIÓN?
%	\item[Parámetros:] El mensaje se muestra con base en los siguientes parámetros:
%	\begin{Citemize} 
%		\item ARTICULO: Es la parte de la oración que se ocupa de expresar el género (masculino/femenino).
%		\item ELEMENTO: Es el elemento que se requiere eliminar.
%		\item SELECCIÓN: Dependiendo del género en la oración se ocupa, seleccionado(masculino) o seleccionada(femenino).
%	\end{Citemize}
%	\item[Ejemplo:] ¿Desea eliminar el ciclo escolar seleccionado?
%	\item[Referenciado por:] \cdtIdRef{CUPA1.1-7}{Eliminar ciclo escolar}, 
%	\cdtIdRef{CUPA1.1-8}{Eliminar actividad},  
%	\cdtIdRef{CUPA1.2-7}{Eliminar requisito}, 
%	\cdtIdRef{CUPA1.1-13}{Eliminar etapa}, 
%	\cdtIdRef{CUPA1.2-16}{Eliminar periodo}, 
%	\cdtIdRef{CUPA1.4-2}{Eliminar medio de contacto}, \cdtIdRef{CUPA1.8.3-5}{Eliminar entrevistadores},\cdtIdRef{CUPA1.6-7}{Eliminar aspirante},
%	\cdtIdRef{CUPA1.9-4}{Eliminar aspirante de la lista de aceptados}, \cdtIdRef{CUPA1.2-24}{Eliminar convocatoria de ingreso}, \cdtIdRef{CUEA2.22}{Eliminar Configuración de Equivalencias}, \cdtIdRef{CUEA2.10}{Eliminar Equivalencia}, 
%	\cdtIdRef{CUEA2.20}{Eliminar Plan de Estudios}, \cdtIdRef{CUGA7.1-4}{Eliminar grupo}, 
%	\cdtIdRef{CUGA7.2-4}{Eliminar salón}, 
%	\cdtIdRef{CUGA7.3-4}{Eliminar profesor}, 
%	\cdtIdRef{CUGA7.3-14}{Eliminar domicilio}, 
%	\cdtIdRef{CUGA7.3-15}{Eliminar medio de contacto}, \cdtIdRef{CUGA7.3-16}{Eliminar Historial Académico}, \cdtIdRef{CUGA7.3-17}{Eliminar trayectoria profesional}, \cdtIdRef{CUGA7.4-3-3}{Eliminar ausencia temporal},
%	\cdtIdRef{CUPA1.2-22}{Eliminar actividad de convocatoria de ingreso},
%	\cdtIdRef{CUPA1.7-6}{Eliminar psicólogo},
%	\cdtIdRef{CUPA1.7-19}{Eliminar concepto}.
%	\end{mensaje}
%
%
%%============================== MSG15 =================================
%\begin{mensaje}{MSG15}{Elemento duplicado}{Error}
%	\item[Ubicación:] \msjSuperior, \msjCampo
%	\item[Estatus:] Terminado
%	\item[Objetivo:] Notificar al actor que ya existe un elemento con las mismas características.
%	\item[Redacción:] A continuación se listan las posibles redacciones que este mensaje puede tener:
%	\begin{enumerate}
%		\item CARACTERÍSTICA ya existente. Ingrese INDETERMINADO CARACTERÍSTICA diferente para DETERMINADO ELEMENTO. 
%		\item INDETERMINADO de ELEMENTOS REGISTRADO se encuentra DUPLICADO.
%		\item ELEMENTO ya existente. Seleccione INDETERMINADO CARACTERÍSTICA diferente.
%	\end{enumerate}
%
%	\item[Parámetros:] El mensaje se muestra con base en los siguientes parámetros:
%	
%	\begin{itemize}
%		\item ELEMENTO: Elemento que se encuentra duplicado.
%		\item CARACTERÍSTICA: Es la característica del elemento que causó la duplicidad.
%		\item DETERMINADO: Artículo determinado, puede ser: el, la, los y las.
%		\item INDETERMINADO: Artículo indeterminado, puede ser:  un, una, unos y unas.
%		\item ARTICULO: Artículo determinado para el elemento, puede ser: el y la.
%		\item ELEMENTOS: Elementos de los cuales alguno se encuentra duplicado.
%		\item REGISTRADO: De acuerdo al género del elemento, puede ser: registrado o registrada.
%		\item DUPLICADO: De acuerdo al género del elemento, puede ser: duplicado o duplicada.
%	\end{itemize}
%	\item[Ejemplo:] 
%	\begin{enumerate}
%		\item Correo electrónico ya existente. Ingrese un correo diferente para la cuenta.
%		\item Una de las fechas de la actividad registrada se encuentra duplicada.
%		\item Equivalencias entre planes de estudios ya existente. Seleccione un plan de estudios equivalente diferente.
%	\end{enumerate}
%	
%	\item[Referenciado por:] \cdtIdRef{CUGA7.1-2}{Registrar grupo},
%	\cdtIdRef{CUPA1.1-5}{Editar actividad de la etapa}, 
%	\cdtIdRef{CUPA1.1-12}{Registrar etapa},  
%	\cdtIdRef{CUPA1.1-6}{Registrar actividad de la etapa}, 
%	\cdtIdRef{CUGA7.1-3}{Editar grupo}, 
%	\cdtIdRef{CUGA7.2-2}{Registrar salón}, 
%	\cdtIdRef{CUGA7.2-3}{Editar salón}, 
%	\cdtIdRef{CUGA7.3-2}{Registrar profesor}, 
%	\cdtIdRef{CUGA7.3-2-1}{Crear cuenta}, 
%	\cdtIdRef{CUGA7.3-3}{Editar profesor}, 
%	\cdtIdRef{CUPA1.1-6}{Registrar actividad}, \cdtIdRef{CUPA1.2-3}{Registrar criterio}, 
%	\cdtIdRef{CUPA1.2-5}{Agregar requisito}, 
%	\cdtIdRef{CUPA1.2-6}{Editar requisito},
%	\cdtIdRef{CUPA1.2-20}{Registrar convocatoria de ingreso},
%	\cdtIdRef{CUPA1.2-21}{Editar convocatoria de ingreso},
%	\cdtIdRef{CUPA1.7-4}{Registrar psicólogo},
%	\cdtIdRef{CUPA1.7-5}{Editar psicólogo},
%	\cdtIdRef{CUPA1.7-17}{Registrar concepto},
%	\cdtIdRef{CUPA1.7-18}{Editar concepto},
%	\cdtIdRef{CUPA1.7-25}{Crear cuenta de psicólogo},
%	\cdtIdRef{CUPA1.7-26}{Editar cuenta del psicólogo}.
%\end{mensaje}
%
%%============================== MSG16 =================================
%\begin{mensaje}{MSG16}{Cuenta inactiva}{Notificación}
%	\item[Ubicación:] \msjSuperior
%	\item[Estatus:] Terminado
%	\item[Objetivo:] Notificar al actor que una cuenta está inactiva.
%	\item[Redacción:] Esta cuenta se encuentra inactiva, para mayor información comuníquese con la ELD.
%	\item[Referenciado por:] 
%\end{mensaje}
%
%
%%===========================  MSG17 ==================================
%\begin{mensaje}{MSG17}{Correo para confirmar operación}{Notificación}
%	\item[Ubicación:] Correo electrónico.
%	\item[Estatus:] Terminado
%	\item[Objetivo:] Notificar al actor que se le ha enviado un correo con una liga para realizar una operación solicitada.
%	\item[Redacción:] VALOR, para OPERACIÓN es necesario que de clic en el siguiente LINK. Si usted no solicitó OPERACIÓN, haga caso omiso a este mensaje.
%	\item[Parámetros:] El mensaje se muestra con base en los siguientes parámetros:
%	\begin{Citemize} 
%		\item VALOR: Es el nombre completo del destinatario del correo.
%		\item LINK: Es la liga al sistema para realizar la operación. 
%		\item OPERACIÓN: Es la operación solicitada.
%	\end{Citemize}
%	\item[Ejemplo:] Bruno Suárez Cruz, para restablecer su contraseña es necesario que de clic en el siguiente link: https://reestablecer-contraseña/. Si usted no solicitó el restablecimiento de su contraseña, haga caso omiso a este mensaje.
%	\item[Referenciado por:] \cdtIdRef{CUPA1.3-3}{Recuperar contraseña}.
%\end{mensaje}
%
%
%%===========================  MSG18 ==================================
%\begin{mensaje}{MSG18}{Notificación para programar cita}{Notificación}
%	\item[Ubicación:] \msjCorreo.
%	\item[Estatus:] Terminado
%	\item[Objetivo:] Notificar al aspirante que ha pasado a la etapa de entrevistas y que debe agendar una cita para entrevista.
%	\item[Redacción:] 
%	NOMBRE \\
%	Presente \\
%	
%	Conforme al proceso de selección para el ciclo escolar CICLO, de acuerdo al resultado que obtuvo en sus evaluaciones, usted ha pasado a la siguiente etapa del proceso, la cual consiste en una entrevista personal. \\
%	
%	Las entrevistas tendrán verificativo a partir del FECHA1 y hasta el FECHA2. \\
%	
%	Para agendar la cita de su entrevista, es necesario que ingrese al sistema a partir del FECHA3. \\
%	
%	Reciba un cordial saludo.
%	
%	\begin{center}
%		Atentamente,
%	\end{center}
%	\begin{center}
%		Escuela Libre de Derecho.
%	\end{center}
%	
%	\item[Parámetros:] El mensaje se muestra con base con los siguientes parámetros:
%	\begin{Citemize} 
%		\item NOMBRE: Es el nombre completo del Aspirante mostrado de la siguiente manera: Nombres Apellido Paterno Apellido Materno.
%		\item CICLO: Es el nombre del ciclo escolar al que está asociado la convocatoria.
%		\item FECHA1: Es la fecha de inicio del periodo de aplicación de entrevistas.
%		\item FECHA2: Es la fecha de término del periodo de aplicación de entrevistas.
%		\item FECHA3: Es la fecha a partir de la cual es posible agendar citas de entrevista.
%	\end{Citemize}
%	\item[Ejemplo:] 
%	Bruno Suárez Cruz \\
%	Presente \\
%	
%	Conforme al proceso de selección para el ciclo escolar 2017-2018, de acuerdo al resultado que obtuvo en sus evaluaciones, usted ha pasado a la siguiente etapa del proceso, la cual consiste en una entrevista personal. \\
%	
%	Las entrevistas tendrán verificativo a partir del 27/mar/2017 y hasta el 01/abr/2017. \\
%	
%	Para agendar la cita de su entrevista, es necesario que ingrese al sistema a partir del 15/mar/2017. \\
%	
%	Reciba un cordial saludo.
%	
%	\begin{center}
%		Atentamente,
%	\end{center}
%	\begin{center}
%		Escuela Libre de Derecho.
%	\end{center}
%	\item[Referenciado por:] \cdtIdRef{CUPA1.8.1-1}{Generar lista de aspirantes para entrevistar}.
%\end{mensaje}
%
%
%%============================== MSG19 =================================
%\begin{mensaje}{MSG19}{Confirmar acción sin cambios posteriores}{Confirmación}
%	\item[Ubicación:] \msjEmergente,\msjCampo
%	\item[Estatus:] Terminado
%	\item[Objetivo:] Preguntar al actor si desea confirmar una acción que no permite cambios posteriores.
%	
%	\item[Redacción:] Este mensaje puede tener las siguientes redacciones:
%	\begin{enumerate}
%	  \item¿Está seguro que desea ACCIÓN ARTICULO ELEMENTO? Si continúa, ARTICULO ELEMENTO ya no podrá regresar al estado actual     
%	  \item¿Está seguro que desea ACCIÓN ARTICULO ELEMENTO? Si continúa, ARTICULO ELEMENTO ya no podrá ser modificado posteriormente.
%	  \item¿Está seguro que desea ACCIÓN ARTICULO ELEMENTO? Si continúa, ARTICULO ELEMENTO no podrá ser anulado.
%
%	\end{enumerate}
%	\item[Parámetros:] El mensaje se muestra con base en los siguientes parámetros:
%	\begin{itemize}
%		\item ACCIÓN: Es la acción que el actor desea confirmar. 
%		\item ARTICULO ELEMENTO: Es el elemento sobre el cuál se realizará la acción.
%		\item ESTADO: Es el estado actual del elemento.
%	\end{itemize}
%	\item[Ejemplo:] A continuación se muestran algunos ejemplos:
%	\begin{enumerate}
%		\item ¿Está seguro que desea aprobar la equivalencia? Si continúa, la equivalencia ya no podrá regresar a estado actual.
%		\item ¿Está seguro que desea finalizar la reinscripción? Si continúa, la reinscripción ya no podrá ser modificado posteriormente.
%		\item ¿Está seguro que desea cerrar la convocatoria de ingreso? Si continúa, la convocatoria de ingreso ya no podrá regresar al estado actual.
%		\item ¿Está seguro que desea finalizar el registro de conceptos?. Si continúa, el registro de conceptos ya no podrá ser modificado posteriormente.
%		\item ¿Está seguro que desea finalizar la evaluación?. Si continúa, el registro de la evaluación ya no podrá ser modificado posteriormente.
%		\item ¿Está seguro que desea enviar el correo electrónico? Si continúa, el correo electrónico no podrá ser anulado.
%		\item ¿Está seguro que desea finalizar el registro de resultados CENEVAL? Si continúa, el registro ya no podrá ser modificado posteriormente.
%		\item ¿Está seguro que desea establecer como notificado al aspirante? Si continúa, el aspirante ya no podrá regresar al estado actual.
%		\item ¿Está seguro que desea finalizar la gestión de evaluaciones? Si continúa, las evaluaciones ya no prodrán ser modificadas posteriormente.
%		\item¿Está seguro que desea establecer como publicado el calendario escolar? Si continúa, el calendario escolar ya no podrá ser modificado posteriormente.
%		\item¿Está seguro que desea registrar la sede? Si continúa, la sede ya no podrá ser modificado posteriormente.
%		
%	\end{enumerate}
%	\item[Referenciado por:] 
%	\cdtIdRef{CUPA1.1-21}{Establecer calendario escolar como publicado}
%	\cdtIdRef{CUPA1.2-26}{Cerrar convocatoria de ingreso}, 
%	\cdtIdRef{CUPA1.6-5}{Visualizar resultados del EXANI-II},
%	\cdtIdRef{CUIR3.2-2}{Visualizar Horario}, 
%	\cdtIdRef{CUPA1.7-16}{Gestionar conceptos},
%	\cdtIdRef{CUPA1.7-24}{Aprobar evaluación final},
%	.
%\end{mensaje}  
%
%%============================== MSG20 =================================
%\begin{mensaje}{MSG20}{Finalizar gestión de elementos}{Confirmación}
%	\item[Ubicación:] \msjEmergente
%	\item[Estatus:] Terminado
%	\item[Objetivo:] Preguntar al actor si desea finalizar la gestión de algún elemento.
%	\item[Redacción:] Al finalizar la gestión de ELEMENTO serán enviadas para revisión. ¿Desea continuar?
%	\item[Parámetros:] El mensaje se muestra con base en los siguientes parámetros:
%	\begin{itemize}
%		\item ELEMENTO: Es el elemento sobre el cuál se está realizando la gestión.
%	\end{itemize}
%	\item[Ejemplo:] Al finalizar la gestión de materias serán enviadas para revisión. ¿Desea continuar?
%	\item[Referenciado por:] \cdtIdRef{CUPA1.1-2}{Gestionar actividades del ciclo escolar}.
%\end{mensaje}
%
%%============================== MSG21 =================================
%\begin{mensaje}{MSG21}{Confirmar lista para envío de correos electrónicos}{Confirmación}
%	\item[Ubicación:] \msjEmergente
%	\item[Estatus:] Terminado
%	\item[Objetivo:] Preguntar al actor si la lista de destinatarios es correcta para el envío de correos electrónicos.
%	\item[Redacción:] ¿Está seguro que la información de la lista de ELEMENTOS es correcta? Al continuar ya no se podrá modificar la lista y se enviará un correo a todos los ELEMENTOS notificándoles sobre ASUNTO.
%	\item[Parámetros:] El mensaje se muestra con base en los siguientes parámetros:
%	\begin{itemize}
%		\item ELEMENTOS: Es el tipo de elemento que conforma la lista
%	\end{itemize}
%	\item[Ejemplo:] A continuación se muestran algunos ejemplos:
%		\begin{enumerate}
%			\item ¿Está seguro que la información de la lista de aspirantes aceptados es correcta? Al continuar ya no se podrá modificar la lista y se enviará un correo a todos los aspirantes aceptados notificándoles sobre sus resultados.
%			
%			\item ¿Está seguro que la información de la lista de aspirantes seleccionados para entrevista es correcta? Al continuar ya no se podrá modificar la lista y se enviará un correo a todos los aspirantes aceptados notificándoles sobre sus resultados.
%		\end{enumerate}	 
%	\item[Referenciado por:] \cdtIdRef{CUPA1.9-1}{Configurar lista de aspirantes}.
%\end{mensaje}
%
%%============================== MSG22 =================================
%\begin{mensaje}{MSG22}{Confirmar acción con cambios condicionados}{Confirmación}
%	\item[Ubicación:] \msjEmergente
%	\item[Estatus:] Terminado
%	\item[Objetivo:] Preguntar al actor si la información de un elemento es correcta, notificándole que cambios posteriores están condicionados.
%	\item[Redacción:] Este mensaje puede tener las siguientes redacciones:
%	\begin{itemize}
%		\item ¿Está seguro que la información de ARTICULO ELEMENTO es correcta? Ya no podrá modificar ARTICULO ELEMENTO a menos que SITUACIÓN.
%		\item ¿Está seguro que la información de ARTICULO ELEMENTO es correcta? CONSECUENCIA, además, ya no se podrá modificar ARTICULO ELEMENTO a menos que SITUACIÓN.
%	\end{itemize}
%	
%	\item[Parámetros:] El mensaje se muestra con base en los siguientes parámetros:
%	\begin{itemize}
%		\item ARTICULO ELEMENTO: Es el elemento sobre el cuál se está haciendo la confirmación.
%		\item SITUACIÓN: Son los términos bajo los cuales se podría realizar una modificación al elemento.
%		\item CONSECUENCIA: Es la consecuencia que tendrá la acción realizada.
%	\end{itemize}
%	\item[Ejemplo:] A continuación se muestran algunos ejemplos:
%	\begin{enumerate}
%		\item ¿Está seguro que la información de la convocatoria de ingreso es correcta? Ya no podrá modificar la convocatoria de ingreso a menos que la Secretaría de Administración autorice cambios a dicha convocatoria.
%		\item ¿Está seguro que la información de la convocatoria de ingreso es correcta? Ya no podrá modificar la convocatoria de ingreso a menos que sea solicitado por la Secretaría de Administración.
%		\item ¿Está seguro que la información de la convocatoria de ingreso es correcta? Se eliminarán las demás convocatorias de ingreso que se encuentren en estado de edición y no estén asociadas a un ciclo escolar, además, ya no se podrá modificar la convocatoria de ingreso a menos que sea autorizado por el administrador.
%		\item 
%	\end{enumerate}
%
%	\item[Referenciado por:] 
%	\cdtIdRef{CUPA1.2-12}{Visualizar convocatoria de ingreso},
%	\cdtIdRef{CUPA1.2-13}{Aprobar convocatoria de ingreso}, 
%	\cdtIdRef{CUPA1.6-5}{Visualizar resultados del EXANI-II},
%	.
%\end{mensaje}
%
%
%%============================== MSG23 =================================
%\begin{mensaje}{MSG23}{Confirmación de solicitud de cambios}{Confirmación}
%	\item[Ubicación:] \msjCampo
%	\item[Estatus:] Terminado
%	\item[Objetivo:] Obtener confirmación por parte del usuario para regresar un elemento a su estado de edición.
%	\item[Redacción:] ¿Está seguro que desea cambiar ARTICULO ELEMENTO a estado de edición?
%	\item[Parámetros:] El mensaje se muestra con base en los siguientes parámetros:
%	\begin{itemize}
%		\item ARTICULO ELEMENTO: Es el elemento que se desea regresar a estado de edición.
%	\end{itemize}
%	\item[Ejemplo:] A continuación se muestran algunos ejemplos:
%	\begin{enumerate}
%		\item ¿Está seguro que desea cambiar el calendario escolar a estado de edición?
%		\item ¿Está seguro que desea cambiar la convocatoria de ingreso a estado de edición?
%	\end{enumerate}
%	
%	\item[Referenciado por:] \cdtIdRef{CUPA1.2-25}{Habilitar edición de convocatoria de ingreso},
%	.
%\end{mensaje}
%	
%%============================== MSG24 =================================
%\begin{mensaje}{MSG24}{Confirmación de finalización de acción}{Confirmación}
%	\item[Ubicación:] \msjEmergente
%	\item[Estatus:] Terminado
%	\item[Objetivo:] Preguntar al actor si desea finalizar una acción.
%	\item[Redacción:] ¿Está seguro que desea finalizar ARTICULO ELEMENTO?
%	\item[Parámetros:] El mensaje se muestra con base en los siguientes parámetros:
%	\begin{itemize}
%		\item ARTICULO ELEMENTO: Es la tarea que requiere finalizar.
%	\end{itemize}
%	\item[Ejemplo:] A continuación se muestran algunos ejemplos:
%	\begin{enumerate}
%		\item ¿Está seguro que desea finalizar el período de entrevistas?
%		\item ¿Está seguro que desea finalizar la gestión de conceptos?
%		\item ¿Está seguro que desea finalizar el registro de resultados?
%	\end{enumerate}
%	\item[Referenciado por:] \cdtIdRef{CUPA1.7-16}{Gestionar conceptos}.
%\end{mensaje}
%
%
%%============================== MSG25 =================================
%\begin{mensaje}{MSG25}{CURP sin información asociada}{Notificación}
%	\item[Ubicación:] \msjPop
%	\item[Estatus:] Edición
%	\item[Objetivo:] Notificar al actor que la CURP con la que se realizo la búsqueda no tiene información asociada en el sistema.
%	\item[Redacción:] La CURP ingresada no cuenta con información asociada en el sistema.
%
%	\item[Referenciado por:] \cdtIdRef{CUGA7.3-2}{Registrar profesor}.
%\end{mensaje}
%
%%============================== MSG26 =================================
%\begin{mensaje}{MSG26}{Falta dato obligatorio}{Error}
%	\item[Ubicación:] \msjCampo, \msjSuperior
%	\item[Estatus:] Terminado
%	\item[Objetivo:] Notificar al actor la omisión de algún dato obligatorio por ingresar.
%	\item[Redacción:]  A continuación se listan las posibles redacciones que este mensaje puede tener: 
%	\begin{enumerate}
%		\item Campo obligatorio
%		\item ELEMENTO1 es OBLIGATORIO si se ingresó ELEMENTO2.
%	\end{enumerate}
%	\item[Parámetros:] A continuación se describen los parámetros mencionados en la redacción del mensaje:
%	\begin{itemize}
%		\item ELEMENTO1: Es el elemento que se debe ingresar de manera obligatoria.
%		\item ELEMENTO2: Es el elemento que se ingresó y que provocó que ELEMENTO1 sea obligatorio.
%		\item OBLIGATORIO: De acuerdo al género del elemento, puede ser obligatorio u obligatoria.
%	\end{itemize}
%	\item[Ejemplos:] 
%	\begin{itemize}
%		\item La hora de inicio es obligatoria si se ingresó la hora de término.
%	\end{itemize}
%	\item[Referenciado por:] \cdtIdRef{CUGA7.1-2}{Registrar grupo},
%	\cdtIdRef{CUPA1.1-3}{Editar ciclo escolar},
%	\cdtIdRef{CUPA1.1-4}{Registrar ciclo escolar},
%	\cdtIdRef{CUPA1.1-5}{Editar actividad de la etapa},
%	\cdtIdRef{CUPA1.1-6}{Registrar actividad de la etapa},
%	\cdtIdRef{CUPA1.1-9}{Agregar actividad al ciclo escolar}, 
%	\cdtIdRef{CUPA1.1-12}{Registrar etapa}, 
%	\cdtIdRef{CUPA1.1-14}{Editar actividad del ciclo escolar}, 
%	\cdtIdRef{CUPA1.2-15}{Configurar criterio},
%	\cdtIdRef{CUPA1.7-4}{Registrar psicólogo},
%	\cdtIdRef{CUPA1.2-18}{Agregar actividad a convocatoria de ingreso}
%	\cdtIdRef{CUGA7.1-3}{Editar grupo}, 
%	\cdtIdRef{CUGA7.1.1-2}{Agregar horario de grupo}, \cdtIdRef{CUGA7.1.1-3}{Eliminar horario de grupo}, \cdtIdRef{CUGA7.1.1-4}{Editar horario de grupo}, \cdtIdRef{CUGA7.2-2}{Registrar salón}, 
%	\cdtIdRef{CUGA7.2-3}{Editar salón}, 
%	\cdtIdRef{CUGA7.3-2}{Registrar profesor}, 
%	\cdtIdRef{CUGA7.3-2-1}{Crear cuenta}, 
%	\cdtIdRef{CUGA7.3-3}{Editar profesor}, 
%	\cdtIdRef{CUGA7.3-6}{Registrar domicilio}, 
%	\cdtIdRef{CUGA7.3-7}{Editar domicilio}, 
%	\cdtIdRef{CUGA7.3-8}{Gestionar medios de contacto}, \cdtIdRef{CUGA7.3-10}{Registrar historial académico}, \cdtIdRef{CUGA7.3-11}{Registrar trayectoria docente}, \cdtIdRef{CUGA7.3-19}{Cambiar contraseña}, \cdtIdRef{CUPA1.1-6}{Registrar actividad}, \cdtIdRef{CUPA1.2-3}{Registrar criterio}, 
%	\cdtIdRef{CUPA1.2-5}{Agregar requisito}, 
%	\cdtIdRef{CUPA1.2-6}{Editar requisito}, 
%	\cdtIdRef{CUPA1.2-10}{Gestionar requisitos},
%	\cdtIdRef{CUPA1.2-19}{Editar actividad de convocatoria de ingreso},
%	\cdtIdRef{CUPA1.2-20}{Registrar convocatoria de ingreso},
%	\cdtIdRef{CUPA1.2-21}{Editar convocatoria de ingreso},
%	\cdtIdRef{CUPA1.4-3}{Registrar datos personales},
%	\cdtIdRef{CUPA1.4-4}{Registrar domicilio},
%	\cdtIdRef{CUPA1.4-5}{Registrar medios de contacto},
%	\cdtIdRef{CUPA1.4-7}{Registrar información escolar},
%	\cdtIdRef{CUPA1.6-4}{Registrar resultados},
%	\cdtIdRef{CUPA1.7-5}{Editar psicólogo},
%	\cdtIdRef{CUPA1.7-9}{Registrar vigencia},
%	\cdtIdRef{CUPA1.7-10}{Editar vigencia},
%	\cdtIdRef{CUPA1.7-15}{Registrar contraseña},
%	\cdtIdRef{CUPA1.7-17}{Registrar concepto},
%	\cdtIdRef{CUPA1.7-18}{Editar concepto},
%	\cdtIdRef{CUPA1.7-21}{Registrar entrevista},
%	\cdtIdRef{CUPA1.7-22}{Registrar examen electrónico},
%	\cdtIdRef{CUPA1.7-24}{Aprobar evaluación final},
%	\cdtIdRef{CUPA1.7-25}{Crear cuenta de psicólogo},
%	\cdtIdRef{CUPA1.7-26}{Editar cuenta del psicólogo}.
%\end{mensaje}
%
%
%%============================== MSG27 =================================
%\begin{mensaje}{MSG27}{Formato de campo incorrecto}{Error}
%	\item[Ubicación:] \msjCampo.
%	\item[Estatus:] Terminado
%	\item[Objetivo:] Indicar al actor que el dato ingresado en alguno de los campos del formulario no cumple con
%	el tipo de dato definido en el diccionario de datos.
%	\item[Redacción:] El dato ingresado es incorrecto, favor de ingresar un dato válido.
%	\item[Referenciado por:] \cdtIdRef{CUGA7.1-2}{Registrar grupo},
%	\cdtIdRef{CUPA1.1-4}{Registrar ciclo escolar}, 
%	\cdtIdRef{CUPA1.1-12}{Registrar etapa}, 
%	\cdtIdRef{CUPA1.1-6}{Registrar actividad de la etapa},
%	\cdtIdRef{CUPA1.2-15}{Configurar criterio},
%	\cdtIdRef{CUPA1.2-18}{Agregar actividad a convocatoria de ingreso}, 
%	\cdtIdRef{CUGA7.1-3}{Editar grupo}, 
%	\cdtIdRef{CUGA7.2-2}{Registrar salón}, 
%	\cdtIdRef{CUGA7.2-3}{Editar salón}, 
%	\cdtIdRef{CUPA1.2-5}{Agregar requisito}
%	\cdtIdRef{CUPA1.2-6}{Editar requisito}
%	\cdtIdRef{CUGA7.3-2}{Registrar profesor}, 
%	\cdtIdRef{CUGA7.3-2-1}{Crear cuenta}, 
%	\cdtIdRef{CUGA7.3-3}{Editar profesor}, 
%	\cdtIdRef{CUGA7.3-6}{Registrar domicilio}, 
%	\cdtIdRef{CUGA7.3-7}{Editar domicilio}, 
%	\cdtIdRef{CUGA7.3-8}{Gestionar medios de contacto}, \cdtIdRef{CUGA7.3-19}{Cambiar contraseña}, \cdtIdRef{CUPA1.1-6}{Registrar actividad}, \cdtIdRef{CUPA1.2-3}{Registrar criterio},
%	\cdtIdRef{CUPA1.2-19}{Editar actividad de convocatoria de ingreso},
%	\cdtIdRef{CUPA1.2-20}{Registrar convocatoria de ingreso},
%	\cdtIdRef{CUPA1.2-21}{Editar convocatoria de ingreso},
%	\cdtIdRef{CUPA1.4-3}{Registrar datos personales},
%	\cdtIdRef{CUPA1.4-4}{Registrar domicilio},
%	\cdtIdRef{CUPA1.4-5}{Registrar medios de contacto},
%	\cdtIdRef{CUPA1.4-7}{Registrar información escolar},
%	\cdtIdRef{CUPA1.7-4}{Registrar psicólogo},
%	\cdtIdRef{CUPA1.7-5}{Editar psicólogo},
%	\cdtIdRef{CUPA1.7-15}{Registrar contraseña},
%	\cdtIdRef{CUPA1.7-17}{Registrar concepto},
%	\cdtIdRef{CUPA1.7-18}{Editar concepto},
%	\cdtIdRef{CUPA1.7-21}{Registrar entrevista},
%	\cdtIdRef{CUPA1.7-25}{Crear cuenta de psicólogo},
%	\cdtIdRef{CUPA1.7-26}{Editar cuenta del psicólogo}. 
%\end{mensaje}
%
%
%%============================== MSG28 =================================
%\begin{mensaje}{MSG28}{Error en el correo electrónico}{Error}
%	\item[Ubicación:] \msjCampo.
%	\item[Estatus:] Terminado
%	\item[Objetivo:] Indicar al actor que el correo electrónico que ingreso no cumple con el formato requerido.
%	\item[Redacción:] El correo electrónico ingresado no tiene un formato válido, ingrese un correo electrónico válido.
%	\item[Referenciado por:] \cdtIdRef{CUPA1.3-2}{Crear cuenta},
%	\cdtIdRef{CUPA1.4-5}{Registrar medios de contacto}, 
%	\cdtIdRef{CUGA7.3-8}{Gestionar medios de contacto},
%	\cdtIdRef{CUPA1.7-25}{Crear cuenta de psicólogo},
%	\cdtIdRef{CUPA1.7-26}{Editar cuenta del psicólogo}.
%\end{mensaje}
%
%%============================== MSG29 =================================
%\begin{mensaje}{MSG29}{Nombre de usuario y/o contraseña incorrecto}{Error}
%	\item[Ubicación:] \msjCampo.
%	\item[Estatus:] Terminado
%	\item[Objetivo:] Indicar al actor que el nombre de usuario y/o contraseña son incorrectos.
%	\item[Redacción:] El nombre de usuario y/o contraseña son incorrectos. Favor de verificarlo.
%	\BRitem{Referenciado por:}{\cdtIdRef{CUPA1.3-1}{Iniciar Sesión}, \cdtIdRef{CUPA1.3-3}{Recuperar Contraseña}}.
%\end{mensaje}
%
%%============================== MSG30 =================================
%\begin{mensaje}{MSG30}{Entidad no encontrada}{Error}
%	\item[Ubicación:] \msjCampo.
%	\item[Estatus:] Terminado
%	\item[Objetivo:] Indicar al actor que no existe una entidad asociada al atributo que ha ingresado.
%	\item[Redacción:] No existe INDETERMINADO ENTIDAD asociada a DETERMINADO ATRIBUTO ingresado. Favor de verificarlo.
%		\item[Parámetros:] El mensaje se muestra con base en los siguientes parámetros:
%	\begin{Citemize} 
%		\item DETERMINADO ATRIBUTO: Es un artículo determinado más el nombre del atributo de la entidad.
%		\item INDETERMINADO ENTIDAD: Es un articulo indeterminado más el nombre de la entidad sobre la cual se desea realizar la operación.
%	\end{Citemize}
%	\item[Ejemplo:] No existe un salón asociado al grupo ingresado. Favor de verificarlo.
%	\item[Referenciado por:] \cdtIdRef{CUPA1.3-3}{Recuperar contraseña}.
%\end{mensaje}
%
%%============================== MSG31 =================================
%\begin{mensaje}{MSG31}{Sección de información incompleta}{Error}
%	\item[Ubicación:] En la parte superior de la pantalla.
%	\item[Estatus:] Terminado
%	\item[Objetivo:] Indicar al actor que existen secciones de información sin responder.
%	\item[Redacción:] Las secciones de información marcadas con el símbolo ! están incompletas, debe responder todas las secciones para concluir el registro.
%	\BRitem{Referenciado por}{\cdtIdRef{CUPA1.4}{Registrar aspirante}}.
%\end{mensaje}
%
%%============================== MSG32 =================================
%\begin{mensaje}{MSG32}{No existe información en el sistema}{Error}
%	\item[Ubicación:] \msjSuperior.
%	\item[Estatus:] Terminado
%	\item[Objetivo:] Notificar al actor que no hay información necesaria en el sistema para ejecutar la operación solicitada.
%	\item[Redacción:] Falta información necesaria en el sistema para poder realizar esta operación.
%	\item[Referenciado por:] %\cdtIdRef{CUPA1.1-3}{Editar ciclo escolar}, \cdtIdRef{CUPA1.1-4}{Registrar ciclo escolar},
%	 \cdtIdRef{CUPA1.1-3}{Editar ciclo escolar},
%	 \cdtIdRef{CUPA1.1-4}{Registrar ciclo escolar},
%	 \cdtIdRef{CUPA1.1-5}{Editar actividad}, 
%	 \cdtIdRef{CUPA1.1-6}{Registrar actividad}, 	
%	 \cdtIdRef{CUPA1.1-9}{Agregar actividad al ciclo escolar}, 
%	 \cdtIdRef{CUPA1.1-12}{Registrar etapa},
%	 \cdtIdRef{CUPA1.1-11}{Gestionar etapas}, 
%	 \cdtIdRef{CUPA1.1-14}{Editar actividad al ciclo escolar}, \cdtIdRef{CUPA1.4-4}{Registrar domicilio}, \cdtIdRef{CUPA1.2-5}{Agregar requisito}, 
%	 \cdtIdRef{CUPA1.2-6}{Editar requisito},
%	 \cdtIdRef{CUPA1.2-18}{Agregar actividad a convocatoria de ingreso}, \cdtIdRef{CUPA1.8.5-3}{Registrar evaluación de entrevistas},
%	 \cdtIdRef{CUPA1.1-11}{Gestionar etapas},
%	 \cdtIdRef{CUPA1.2-19}{Editar actividad de convocatoria de ingreso},
%	 \cdtIdRef{CUPA1.6-2}{Reservar salones},
%	 \cdtIdRef{CUPA1.7-2}{Reservar Salones},
%	 \cdtIdRef{CUGA7.3-10}{Registrar historial académico},
%	 \cdtIdRef{CUGA7.3-11}{Registrar trayectoria docente}, \cdtIdRef{CUGA7.3-7}{Editar domicilio}, 
%	 \cdtIdRef{CUGA7.3-6}{Registrar domicilio}, \cdtIdRef{CUGA7.1-2}{Registrar grupo}, 
%	 \cdtIdRef{CUGA7.1-3}{Editar grupo}, 
%	 \cdtIdRef{CUGA7.2-2}{Registrar salón}, 
%	 \cdtIdRef{CUGA7.2-3}{Editar salón}, 
%	 \cdtIdRef{CUGA7.3-2}{Registrar profesor}, 
%	 \cdtIdRef{CUGA7.3-3}{Editar profesor}, 
%	 \cdtIdRef{CUEA2.16}{Editar materia}, 
%	 \cdtIdRef{CUEA2.4}{Editar plan de estudios}, \cdtIdRef{CUEA2.8}{Registrar Configuración de Equivalencias}, \cdtIdRef{CUGA7.1-2}{Registrar grupo}, 
%	 \cdtIdRef{CUGA7.1-3}{Editar grupo}, \cdtIdRef{CUGA7.1.1-2}{Agregar horario de grupo}, 
%	 \cdtIdRef{CUGA7.1.1-3}{Eliminar horario de grupo}, \cdtIdRef{CUGA7.1.1-4}{Editar horario de grupo},
%	 \cdtIdRef{CUGA7.1-6}{Gestionar planeación de grupos},   \cdtIdRef{CUGA7.2-2}{Registrar salón},
%	 \cdtIdRef{CUGA7.2-3}{Editar salón}, 
%	 \cdtIdRef{CUGA7.3-2}{Registrar profesor}, 
%	 \cdtIdRef{CUGA7.3-3}{Editar profesor}, 
%	 \cdtIdRef{CUGA7.3-6}{Registrar domicilio}, 
%	 \cdtIdRef{CUGA7.3-7}{Editar domicilio}, 
%	 \cdtIdRef{CUGA7.3-8}{Gestionar medios de contacto}, \cdtIdRef{CUGA7.3-10}{Registrar historial académico}, \cdtIdRef{CUGA7.3-11}{Registrar trayectoria docente},
%	 \cdtIdRef{CUPA1.2-20}{Registrar convocatoria de ingreso}.
%\end{mensaje}
%
%%============================== MSG33 =================================
%\begin{mensaje}{MSG33}{Relación de fechas incorrecta}{Error}
%	\item[Ubicación:]  \msjCampo
%	\item[Estatus:] Terminado
%	\item[Objetivo:] Notificar al actor que existe una relación de fechas incorrecta.
%	\item[Redacción:] La fecha de ENTIDAD1 debe ser CONDICIÓN que la fecha de ENTIDAD2.
%	\item[Parámetros:] El mensaje se muestra con base en los siguientes parámetros:
%	\begin{itemize}
%		\item ENTIDAD1: Es una entidad que requiere de una fecha.
%		\item ENTIDAD2: Es una entidad que requiere de una fecha.
%		\item CONDICIÓN: Es la condición que debe satisfacer la relación entre las fechas.
%	\end{itemize}
%	\item[Ejemplo:] A continuación se muestran algunos ejemplos:
%	\begin{itemize}
%		\item La fecha de inicio debe ser menor que la fecha de término.
%		\item La fecha de egreso debe ser al menos 5 años mayor que la fecha de ingreso.
%		\item La fecha de titulación debe ser menor que la fecha actual.
%		\item La fecha de nombramiento debe ser menor o igual que la fecha actual.
%		\item La fecha de inicio de vigencia debe ser menor que la fecha de término de vigencia.
%		\item La fecha de término de vigencia debe ser mayor o igual  que la fecha de inicio de vigencia.
%		\item La fecha de término de vigencia debe ser menor o igual que la fecha de publicación de resultados.
%		\item La fecha de registro debe ser menor que la fecha actual.
%		\item La fecha de registro debe ser mayor que la fecha de registro de la cuenta.
%	\end{itemize} 
%	
%	\item[Referenciado por:] \cdtIdRef{CUPA1.1-3}{Editar ciclo escolar}, \cdtIdRef{CUPA1.1-9}{Agregar actividad al ciclo escolar}, \cdtIdRef{CUPA1.1-14}{Editar actividad del ciclo escolar}, \cdtIdRef{CUPA1.1-4}{Registrar ciclo escolar},
%	\cdtIdRef{CUPA1.2-18}{Agregar actividad a convocatoria de ingreso},
%	\cdtIdRef{CUPA1.2-19}{Editar actividad de convocatoria de ingreso},
%	\cdtIdRef{CUPA1.7-9}{Registrar vigencia},
%	\cdtIdRef{CUPA1.7-10}{Editar vigencia},
%	\cdtIdRef{CUPA1.7-25}{Crear cuenta de psicólogo}.
%\end{mensaje}
%
%
%%============================== MSG34 =================================
%\begin{mensaje}{MSG34}{Duración de ciclo escolar incorrecto}{Error}
%	\item[Ubicación:] \msjCampo.
%	\item[Estatus:] Terminado
%	\item[Objetivo:] Notificar al actor que la duración del ciclo escolar es menor a 2 meses o mayor a 365 días.
%	\item[Redacción:] La duración del ciclo escolar debe ser mayor a 2 meses y menor a 365 días.
%	\item[Referenciado por:] 
%	\cdtIdRef{CUPA1.1-3}{Editar ciclo escolar},
%	\cdtIdRef{CUPA1.1-4}{Registrar ciclo escolar}.
%\end{mensaje}
%
%
%%============================== MSG35 =================================
%\begin{mensaje}{MSG35}{Cantidad mínima de elementos a agregar}{Error}
%	\item[Ubicación:] \msjSuperior
%	\item[Estatus:] Terminado
%	\item[Objetivo:] Notificar al actor que se debe agregar al menos un elemento para poder realizar la operación.
%	\item[Redacción:] Se debe agregar al menos CANTIDAD ELEMENTO para poder OPERACIÓN.
%	\item[Parámetros:] El mensaje se muestra con base en los siguientes parámetros:
%	\begin{itemize}
%		\item ELEMENTO: Es el elemento del cuál se deben agregar.
%		\item OPERACIÓN: Es la operación que se quiere realizar.
%		\item CANTIDAD: Es el cantidad de elementos que deben agregarse como mínimo.
%	\end{itemize}
%	\item[Ejemplo:] Se debe agregar al menos un horario para poder agendar la entrevista.
%	\item[Referenciado por:] \cdtIdRef{CUPA1.1-2}{Gestionar actividades del ciclo escolar}.
%\end{mensaje}
%
%%============================== MSG36 =================================
%\begin{mensaje}{MSG36}{La confirmación de contraseña no coincide}{Error}
%	\item[Ubicación:] \msjCampo.
%	\item[Estatus:] Terminado
%	\item[Objetivo:] Notificar al actor que la confirmación de la contraseña no coincide.
%	\item[Redacción:] La contraseña y su confirmación no coinciden. Vuelve a intentarlo.
%	\item[Referenciado por:] \cdtIdRef{CUPA1.3-3}{Recuperar contraseña},
%	\cdtIdRef{CUPA1.7-15}{Registrar contraseña}.
%\end{mensaje}
%
%%============================== MSG37 =================================
%\begin{mensaje}{MSG37}{Error con el token de recuperación de contraseña}{Error}
%	\item[Ubicación:] Ventana emergente.
%	\item[Estatus:] Terminado
%	\item[Objetivo:] Notificar al actor que el token para poder recuperar su contraseña es inválido o ha expirado.
%	\item[Redacción:] El token de recuperación de contraseña no es válido o ha expirado, solicite recuperar su contraseña de nuevo.
%	\item[Referenciado por:] \cdtIdRef{CUPA1.3-3}{Recuperar contraseña}.
%\end{mensaje}
%
%%===========================  MSG38 ==================================
%\begin{mensaje}{MSG38}{Mensaje de ayuda para el certificado de estudios}{Notificación}
%	\item[Ubicación:] \msjIcono
%	\item[Estatus:] Por aprobar
%	\item[Objetivo:] Informar al aspirante cómo debe ser la documentación que debe subir al sistema.
%	\item[Redacción:] Si no cuenta con su certificado de NIVEL DE ESTUDIOS debido a que lo ha extraviado debe subir sus boletas finales en un archivo.
%	\item[Parámetros:] El mensaje se muestra con base en los siguientes parámetros:
%	\begin{Citemize} 
%		\item NIVEL DE ESTUDIOS: Es el nivel de estudios del certificado con el que cuenta el aspirante.
%	\end{Citemize}
%	\item[Ejemplo:] Si no cuenta con su certificado de bachillerato debido a que lo ha extraviado debe subir sus boletas finales en un archivo.
%	\item[Referenciado por:] 
%\end{mensaje}
%
%%===========================  MSG39 ==================================
%\begin{mensaje}{MSG39}{Mensaje de ayuda para el nombre de la escuela}{Notificación}
%	\item[Ubicación:] \msjIcono
%	\item[Estatus:] Por aprobar
%	\item[Objetivo:] Informar al aspirante cómo debe ingresar el nombre de su escuela al sistema.
%	\item[Redacción:] El nombre de NIVEL DE ESTUDIOS debe ser el que se encuentra en su certificado o boleta global.
%	\item[Parámetros:] El mensaje se muestra con base en los siguientes parámetros:
%		\begin{Citemize} 
%			\item NIVEL DE ESTUDIOS: Es el nivel de estudios del cual se escribirá el nombre de la institución.
%		\end{Citemize}
%	\item[Ejemplo:] A continuación se muestran algunos ejemplos:
%	\begin{itemize}
%		\item El nombre de el bachillerato debe ser el que se encuentra en su certificado o boleta global.
%		\item El nombre de la secundaria debe ser el que se encuentra en su certificado o boleta global.
%	\end{itemize}
%	    
%	\item[Referenciado por:] 
%\end{mensaje}
%
%%============================== MSG40 =================================
%\begin{mensaje}{MSG40}{Registro de aspirante concluido}{Notificación}
%	\item[Ubicación:] \msjCentro. 
%	\item[Estatus:] Terminado
%	\item[Objetivo:] Notificar al actor que su registro fue procesado correctamente.
%	\item[Redacción:] Su registro se ha procesado correctamente, su documentación será validada una vez iniciado el proceso de inscripción, es necesario que realice su pago para poder presentar los exámenes.
%	\item[Referenciado por:] \cdtIdRef{CUPA1.4}{Registrar Aspirante}.
%\end{mensaje}
%
%
%%============================== MSG41 =================================
%\begin{mensaje}{MSG41}{Formato de la contraseña}{Notificación}
%\item[Ubicación:] \msjIcono,\msjCampo.
%\item[Estatus:] Terminado
%\item[Objetivo:] Informar al aspirante cuales son los caracteres especiales permitidos en la contraseña.
%\item[Redacción:] La contraseña debe contener al menos una mayúscula, una minúscula, un número y un símbolo.Los símbolos que pueden ser utilizados en la contraseña son: . - \_ !? \% \&
%\item[Referenciado por:] \cdtIdRef{CUGA7.3-2}{Registrar profesor}, \cdtIdRef{CUGA7.3-2-1}{Crear cuenta del profesor}.
%\end{mensaje}
%
%%============================== MSG42 =================================
%\begin{mensaje}{MSG42}{Hora de inicio incorrecta}{Error}
%	\item[Ubicación:] \msjCampo
%	\item[Estatus:] Terminado
%	\item[Objetivo:] Notificar al actor que la hora inicial es mayor a la hora de término.
%	\item[Redacción:] La hora de inicio debe ser menor a la hora de término.
%	\item[Referenciado por:] 
%	\cdtIdRef{CUPA1.1-14}{Editar actividad al ciclo escolar}, \cdtIdRef{CUPA1.1-9}{Agregar actividad al ciclo escolar}, 
%	\cdtIdRef{CUPA1.2-8}{Registrar Periodo}, 
%	\cdtIdRef{CUPA1.2-9}{Editar Periodo}, 
%	\cdtIdRef{CUPA1.8.3-3}{Administrar horarios del entrevistador},
%	\cdtIdRef{CUPA1.2-18}{Agregar actividad a convocatoria de ingreso},
%	\cdtIdRef{CUPA1.2-19}{Editar actividad de convocatoria de ingreso}.
%\end{mensaje}
%
%%============================== MSG43 =================================
%\begin{mensaje}{MSG43}{Relación incorrecta de cantidades}{Error}
%	\item[Ubicación:] \msjCampo
%	\item[Estatus:] Terminado
%	\item[Objetivo:] Notificar al actor que existe una relación incorrecta entre dos cantidades.
%	\item[Redacción:] A continuación se listan las posibles redacciones que este mensaje puede tener dependiendo de lo especificado en el diccionario de datos: 
%	\begin{itemize}
%		\item ELEMENTO debe ser CONDICIÓN a NÚMERO.
%		\item ELEMENTO debe ser CONDICIÓN1 a NÚMERO1 y CONDICIÓN2 a NÚMERO2.
%	\end{itemize}
%	\item[Parámetros:] El mensaje se muestra con base en los siguientes parámetros:
%	\begin{itemize}
%		\item ELEMENTO: Es el elemento del cuál se toman las cantidades.
%		\item CONDICIÓN: Es la relación que debe existir con NÚMERO. Los valores que puede tener son: mayor, menor, igual, menor o igual y mayor o igual.
%		\item CONDICIÓN1: Es la relación que debe existir con NÚMERO1. Los valores que puede tener son: mayor, menor, menor o igual y mayor o igual.
%		\item CONDICIÓN2: Es la relación que debe existir con NÚMERO2. Los valores que puede tener son: mayor, menor, menor o igual y mayor o igual.
%		\item NÚMERO: Es la cantidad del ELEMENTO que establece una referencia.
%		\item NÚMERO1: El campo debe ser CONDICIÓN1 a este número.
%		\item NÚMERO2: El campo debe ser CONDICIÓN2 a este número.
%	\end{itemize}
%	\item[Ejemplo:]
%		\begin{itemize}
%			\item El número de aspirantes debe ser menor a 60.
%			\item La ponderación debe ser mayor a 0 y menor o igual a 100.
%			\item El promedio mínimo debe ser mayor o igual a 0 y menor o igual a 10
%			\item El número de materias obligatorias asignadas a un grado debe ser mayor a 0 y menor a 10.
%			\item El número de materias optativas asignadas a un grado debe ser mayor o igual a 0 y menor a 10. 
%			\item El número de aspirantes a asignar a un psicólogo debe ser menor o igual a 12.
%		\end{itemize}	
%	
%	\item[Referenciado por:]  \cdtIdRef{CUPA1.2-2}{Gestionar criterios}, \cdtIdRef{CUPA1.2-15}{Configurar criterio}, \cdtIdRef{CUPA1.2-20}{Registrar convocatoria de ingreso}, \cdtIdRef{CUPA1.2-21}{Editar convocatoria de ingreso},
%	\cdtRef{CUAE2.3}{Registrar plan de estudios}.
%\end{mensaje}
%
%
%%============================== MSG44 =================================
%\begin{mensaje}{MSG44}{Cantidad de elementos fuera de rango}{Error}
%	\item[Ubicación:] \msjCampo, \msjCentro
%	\item[Estatus:] Terminado
%	\item[Objetivo:] Notificar al actor que la cantidad de algún elemento está fuera del rango permitido.
%	\item[Redacción:] Cantidad de ELEMENTO fuera de rango. El rango permitido es MÍNIMO a MÁXIMO.
%	\item[Parámetros:] El mensaje se muestra con base en los siguientes parámetros:
%	\begin{itemize}
%		\item ELEMENTO: Es el elemento del cuál se está tomando la cantidad.
%		\item MÍNIMO: Es el valor mínimo del rango permitido.
%		\item MÁXIMO: Es el valor máximo del rango permitido.
%	\end{itemize}
%	\item[Ejemplo:] Cantidad de psicólogos fuera de rango. El rango permitido es 20 a 40.
%	\item[Referenciado por:] \cdtIdRef{CUPA1.9-2}{Agregar aspirantes a la lista de aceptados}.
%\end{mensaje}
%
%
%%============================== MSG45 =================================
%\begin{mensaje}{MSG45}{Cambiar estado de la entidad}{Notificación}
%	\item[Ubicación:] \msjEmergente.
%	\item[Estatus:] Terminado
%	\item[Objetivo:] Preguntar al actor si desea cambiar el estado de una entidad.
%	\item[Redacción:] ¿Está seguro que desea cambiar el estado de ENTIDAD a ESTADO?.
%	\item[Parámetros:] El mensaje se muestra con base en los siguientes parámetros:
%	\begin{Citemize} 
%		\item ENTIDAD: Es el nombre de la entidad a la que se quiere cambiar su estado.
%		\item ESTADO: Es el estado al que la entidad puede cambiarse.
%	\end{Citemize}
%	\item[Ejemplo:] ¿Está seguro que desea cambiar el estado de Eduardo Espino Maldonado a aceptado?.
%	\item[Referenciado por:] \cdtIdRef{CUPA1.9-2}{Agregar aspirantes a la lista de aceptados}.
%\end{mensaje}
%
%
%%============================== MSG46 =================================
%\begin{mensaje}{MSG46}{Mensaje de bienvenida Aspirante}{Notificación}
%	\item[Ubicación:] \msjCentro.
%	\item[Estatus:] Edición
%	\item[Objetivo:] Dar la bienvenida al Aspirante que ha ingresado a su cuenta.
%
%	\item[Redacción:] 
%	Su cuenta ha sido activada exitosamente. \\ \\
%	Para comenzar con el proceso de admisión es necesario que pase por las siguientes etapas:
%	    \begin{Citemize}
%		\item Etapa NOMBRE ETAPA. DESCRIPCIÓN.
%	    \end{Citemize}
%	Para acceder a la primera etapa oprima el botón Siguiente.
%	
%	\item[Parámetros:] El mensaje se muestra con base con los siguientes parámetros:
%		\begin{Citemize}
%			\item NOMBRE ETAPA: Nombre de la etapa indicada en la convocatoria de ingreso.
%			\item DESCRIPCIÓN: Descripción de las actividades a realizar en la etapa.
%		\end{Citemize}
%	
%	\item[Ejemplo:]
%	      Su cuenta ha sido activada exitosamente. \\ \\
%	      Para comenzar con el proceso de admisión es necesario que pase por las siguientes etapas:
%		    \begin{Citemize}
%			  \item Etapa Registro de Aspirantes. En esta etapa deberá proporcionar su información personal y académica.
%			  \item Etapa Pago de Derechos. En esta etapa deberá realizar el pago correspondiente para derecho a presentar las evaluaciones.
%			  \item Etapa Evaluaciones. En esta etapa deberá presentar las evaluaciones indicadas en la convocatoria de ingreso.
%			  \item Etapa Entrevista. En esta etapa deberá asistir a una entrevista en la ELD.
%			  \item Etapa Resultados. En esta etapa deberá esperar la fecha de publicación de resultados.
%		    \end{Citemize}
%
%	      Para acceder a la primera etapa oprima el botón Siguiente.
%
%	\item[Referenciado por:] 
%\end{mensaje}
%
%
%%============================ MSG47 =================================
%\begin{mensaje}{MSG47}{Falta información para realizar la operación}{Error}
%	\item[Ubicación:] Se muestra en la parte superior de la pantalla.
%	\item[Estatus:] Terminado
%	\item[Objetivo:] Indicar al actor que no puede realizar la operación solicitada debido a que falta información que debió ser registrada en el sistema.
%	\item[Redacción:] Debe tener CONDICION para realizar esta operación.
%	\item[Parámetros:] El mensaje se muestra con base en los siguientes parámetros:
%	\begin{Citemize} 
%		\item CONDICION: Es la condición que se debe cumplir para poder acceder a la operación solicitada.
%	\end{Citemize}
%	\item[Ejemplo:] A continuación se lista algunos ejemplos:
%	\begin{enumerate}
%		\item Debe tener un ciclo escolar aprobado o publicado sin una convocatoria de ingreso asociada para realizar esta operación.
%		\item Debe tener un ciclo escolar vigente para realizar esta operación.
%	\end{enumerate}
%	
%	\item[Referenciado por:] \cdtIdRef{CUPA1.2-26}{Cerrar convocatoria de ingreso},
%\end{mensaje}
%
%
%%============================== MSG48 =================================
%\begin{mensaje}{MSG48}{Vista previa de convocatoria}{Notificación}
%	\item[Ubicación:] \msjPantalla.
%	\item[Estatus:] Terminado
%	\item[Objetivo:] Mostrar al actor los criterios y requisitos generados de la convocatoria seleccionada.
%	\item[Redacción:] CONVOCATORIA DE SELECCIÓN DE INGRESO A LA CARRERA DE ABOGADO\\
%	CICLO ESCOLAR \{NOMBRE DE CICLO\}\\ \\
%	
%	\{ORDINAL\} ETAPA\\
%	\{PERIODO\}\\
%	\{NOMBRE DE ETAPA\}\\ \\IU1.3-4
%	
%	\{OBSERVACIÓN DE ETAPA\}\\
%	\{LISTA DE ACTIVIDADES\} \{PERIODO\}\{HORAS\} \\
%	
%	El trámite es personal por lo que en algunas etapas es indispensable la presencia del aspirante. Asimismo, para agilizar las gestiones se solicita que los aspirantes no acudan con acompañantes.
%	\item[Parámetros:] El mensaje se muestra con base en los siguientes parámetros:
%	\begin{Citemize} 
%		\item NOMBRE DE CICLO: Es el nombre del ciclo escolar a la que esta asociada la convocatoria.
%		\item ORDINAL: Número en forma ordinal.
%		\item PERIODO: Es el periodo de las actividades o etapas, estableciendo la redacción de la siguiente manera: \{FECHAS\}, \{HORAS\}
%		\item FECHAS: Es la o las fechas de la etapa quedando de alguna de las siguientes redacciones:\\
%				1. \{FECHA\}\\
%				2. Del \{FECHA DE INICIO\} al \{FECHA DE FIN\}
%		\item HORAS: Es la o las horas de la etapa quedando de alguna de las siguientes redacciones:\\
%				1. a partir de las \{HORA\} horas\\
%				2. en un horario de \{HORA DE INICIO\} a \{HORA DE FIN\} horas
%		\item FECHA: Única fecha de la etapa (DD de MES de AAAA).
%		\item FECHA DE INICIO: Fecha de inicio de la etapa (DD de MES).
%		\item FECHA DE FIN: Fecha de fin de la etapa (DD de MES de AAAA).
%		\item HORA: Única hora de la etapa.
%		\item HORA DE INICIO: Hora de inicio de la etapa.
%		\item HORA DE FIN: Hora de fin de la etapa.
%		\item NOMBRE DE ETAPA: Nombre de la etapa que pertenezca al tipo admisión.
%		\item OBSERVACIÓN DE ETAPA: Red  que describa el nombre de la etapa y una redacción que introduzca la lista de actividades.
%		\item LISTA DE ACTIVIDADES: Lista de actividades publicables que corresponden a la etapa quedando la redacción de la siguiente manera:\\
%				\{N\}. \{ACTIVIDAD\}
%		\item N: Número.
%		\item ACTIVIDAD: Nombre de la actividad publicable.
%	\end{Citemize}
%	\item[Ejemplo:] CONVOCATORIA DE SELECCIÓN DE INGRESO A LA CARRERA DE ABOGADO\\
%	CICLO ESCOLAR 2017-2018\\ \\	
%	
%	\{OBSERVACIÓN DE ETAPA\}\\
%	\{LISTA DE ACTIVIDADES\} \{PERIODO\}\{HORAS\} \
%	
%	PRIMERA ETAPA\\
%	Del 16 de enero al 20 de enero de 2017, en un horario de 10:00 a 16:00 horas\\
%	REGISTRO DE CUENTA\\ \\
%	
%	Los aspirantes deberán registrarse en la página URL y proporcionar los datos que se solicitan de las siguientes actividades:\\
%	
%	1. Registro de cuenta del 16 de enero del 2017 al 20 de enero del 2017 \\
%	2. Registro de aspirantes del 16 de enero del 2017 al 20 de enero del 2017\\ \\ \\
%	
%	
%			SEGUNDA ETAPA\\
%			Del 6 de marzo al 22 de marzo de 2017\\
%			ENTREVISTA PERSONAL \\ \\
%			
%			Únicamente los aspirantes con una calificación mínima de 8.0 (ocho punto cero) presentarán entrevista personal, realizando las siguientes actividades:\\ \\
%			
%			1. Selección de cita para entrevista el 6 de Marzo del 2017\\
%			2. Aplicación de entrevista del 7 de Marzo del 2017 al 2 de Marzo del 2017\\ \\ \\
%			
%	
%						TERCERA ETAPA\\
%						27 de marzo de 2017\\
%						ENTREGA DE RESULTADOS\\ \\
%						
%						A partir de esta fecha los aspirantes podrán consultar los resultados de su proceso de selección en la página URL realizándose las siguientes actividades:\\ \\
%						
%						1. Publicación de resultados el 27 de Marzo del 2017\\ \\ \\
%						
%	
%	El trámite es personal por lo que en algunas etapas es indispensable la presencia del aspirante. Asimismo, para agilizar las gestiones se solicita que los aspirantes no acudan con acompañantes.
%	\item[Referenciado por:] \cdtIdRef{CUPA1.2-12}{Visualizar convocatoria de ingreso},
%	\cdtIdRef{CUPA1.2-13}{Aprobar convocatoria de ingreso}.
%\end{mensaje}
%
%%============================== MSG49 =================================
%\begin{mensaje}{MSG49}{Cantidad de horas incorrecta}{Error}
%	\item[Ubicación:] \msjCampo.
%	\item[Estatus:] Terminado
%	\item[Objetivo:] Notificar al actor que la cantidad de horas ingresada es incorrecta.
%	\item[Redacción:] La cantidad de horas de ACTIVIDAD debe ser OPERACIÓN a la cantidad de horas de ACTIVIDAD.
%	\item[Parámetros:] El mensaje se muestra con base en los siguientes parámetros:
%	\begin{Citemize} 
%		\item ACTIVIDAD: Es la actividad de la que se esta haciendo referencia.
%		\item OPERACIÓN: Es la operación que indica lo que se tiene que cumplir.
%	\end{Citemize}
%	\item[Ejemplo:] La cantidad de horas del intervalo debe ser menor o igual a la cantidad de horas de la jornada laboral
%	\item[Referenciado por:] 
%\end{mensaje}
%
%
%%============================== MSG50 =================================
%\begin{mensaje}{MSG50}{Faltan actividades para asociar a los criterios de origen}{Notificación}
%	\item[Ubicación:] \msjSuperior.
%	\item[Estatus:] Terminado
%	\item[Objetivo:] Notificar al actor que faltan actividades para asociar a los criterios.
%	\item[Redacción:] Los siguientes criterios obligatorios no tienen actividades asociadas: CRITERIOS
%	\item[Parámetros:] El mensaje se muestra con base en los siguientes parámetros:
%	\begin{Citemize} 
%		\item CRITERIOS: Lista de criterios obligatorios que no tienen una actividad asociada.
%	\end{Citemize}
%	\item[Ejemplo:]  Los siguientes criterios de origen que no tienen actividades asociadas: CENEVAL, Examen Psicométrico.
%	\item[Referenciado por:] \cdtIdRef{CUPA1.2-11}{Gestionar actividades de la convocatoria de ingreso}.
%\end{mensaje}
%
%%============================== MSG51 =================================
%\begin{mensaje}{MSG51}{Registro de elementos incorrecta}{Error}
%	\item[Ubicación:] \msjEmergente, \msjCentro, \msjSuperior
%	\item[Estatus:] Terminado
%	\item[Objetivo:] Notificar al actor que el número de elementos a registrar es incorrecto.
%	\item[Redacción:] Debe registrar CONDICIÓN NÚMERO ELEMENTO.
%	\item[Parámetros:] El mensaje se muestra con base en los siguientes parámetros:
%	\begin{itemize}
%		\item CONDICIÓN: Es la condición que debe satisfacer el NÚMERO de ELEMENTO. Puede ser como mínimo, a lo más o exactamente.
%		\item NÚMERO: Es el número de ELEMENTO.
%		\item ELEMENTO: Es el nombre del elemento que se seleccionó.
%	\end{itemize}
%	\item[Ejemplo:]
%	\begin{itemize}
%		\item Debe registrar como mínimo 2 criterios.
%		\item Debe registrar exactamente 2 materias optativas.
%		\item Debe registrar como mínimo 1 salón.
%	\end{itemize}		
%	\item[Referenciado por:] 
%	
%\end{mensaje}
%
%
%%============================== MSG52 =================================
%\begin{mensaje}{MSG52}{Ponderación de criterios para promedio general de aspirantes}{Notificación}
%	\item[Ubicación:] \msjCampo
%	\item[Estatus:] Edición
%	\item[Objetivo:] Notificar al actor la ponderación utilizada para el calculo del promedio general de aspirantes a entrevista o aceptados.
%	\item[Redacción:] 
%	 
%		\item Ponderación de Promedio General(PG)
%		
%		\item $C_1$ Promedio Secundaria-Bachillerato: PPSB \%
%		\item $C_2$ N-Criterio: $PAC_2$ \%
%		\item .
%		\item .
%		\item .
%		\item $C_N$ N-Criterio: $PAC_N$ \%
%		
%		\item PG = .PPSB $C_1$ + $.PAC_2 C_2  + ... + .PAC_N C_N $ 
%	
%	\item[Parámetros:] El mensaje se muestra con base en el siguiente parámetro:
%	\begin{Citemize} 
%		\item PPSB: Ponderación asignada a promedio Secundaria-Bachillerato.
%		\item $PAC_X$: Ponderación asignada a un criterio.
%		\item N-Criterio: Nombre del criterio.
%		\item $C_X$: Id de criterio.
%		
%	\end{Citemize}
%	\item[Ejemplo:]
%		\item Ponderación de Promedio General(PG)
%		
%		\item $C_1$ Promedio Secundaria-Bachillerato: 10 \%
%		\item $C_2$ EXANI-II: 40 \%
%		\item $C_3$ Examen Psicométrico: 20 \% 
%		\item $C_4$ Entrevista: 30 \%
%		\item \item PG = .10 $C_1$ + .40 $C_2$ + .20 $C_3$ + .30 $C_4$  
%	\item[Referenciado por:] \cdtIdRef{CUPA1.9-1}{Configurar lista de aspirantes}, 
%	\cdtIdRef{CUPA1.9-3}{Visualizar lista de aspirantes aceptados}.
%\end{mensaje}
%
%%============================= MSG53 ==================================
%\begin{mensaje}{MSG53}{Términos y condiciones para el pago de derechos}{Notificación}
%	\item[Ubicación:] \msjPantalla
%	\item[Estatus:] Terminado
%	\item[Objetivo:] Notificar al actor los términos y condiciones respectivos al pago de derechos.
%	\item[Redacción:] 
%	Para poder continuar con su proceso, es necesario que realice el pago de derecho a exámenes.
%	El pago se puede realizar por los siguientes métodos: \\
%	
%	\item \textbf{Efectivo}
%	
%	\begin{itemize}
%		
%		\item OXXO
%		
%	\end{itemize}
%	
%	\item \textbf{Electrónico}
%	
%	\begin{itemize}
%		\item SPEI
%		\item Tarjeta bancaria  
%		\\
%	\end{itemize}
%	
%	\textbf {IMPORTANTE}
%	\\Una vez realizado el pago, no se harán devoluciones.\\
%	El pago no garantiza el ingreso a la Escuela Libre de Derecho.
%	\\
%	\
%	\item[Referenciado por:] \cdtIdRef{CUPA1.5-1A}{Pago de Derechos}.
%\end{mensaje}
%
%
%
%%============================= MSG54 ==================================
%\begin{mensaje}{MSG54}{Alumno sin inscripción en el ciclo escolar actual}{Notificación}
%	\item[Ubicación:] 
%	\item[Estatus:] Edición.
%	\item[Objetivo:] Notificar al actor que no cuenta con una inscripción al ciclo escolar actual.
%	\item[Redacción:] Aún no cuenta con una inscripción en el ciclo escolar actual.
%	\item[Referenciado por:] \cdtIdRef{CUIR3.2-3}{Visualizar comprobante de inscripción}.
%\end{mensaje}
%
%
%%============================= MSG55 ==================================
%\begin{mensaje}{MSG55}{Materias equivalentes duplicadas}{Error}
%	\item[Ubicación:] \msjCentro.
%	\item[Estatus:] Edición.
%	\item[Objetivo:] Notifica al actor que se ha asociado dos o más veces un materia equivalente a un materia vigente.
%	\item[Redacción:] La relación de materias equivalentes con las materias del plan de estudios vigente debe ser uno a uno. A continuación se muestran las materias equivalentes que se repitieron en más de una materia vigente las cuales están pintadas de color rojo.
%	\item[Referenciado por:] 
%\end{mensaje}
%
%
%%============================= MSG56 ==================================
%\begin{mensaje}{MSG56}{Ciclo escolar sin vigencia}{Notificación}
%	\item[Ubicación:] 
%	\item[Estatus:] Terminado
%	\item[Objetivo:] Notificar al actor que la solicitud depente de la existencia de un ciclo escolar vigente.
%	\item[Redacción:] No existen ciclos escolares vigentes, dicha información es necesaria para la consulta del comprobante de insripción.
%
%	\item[Referenciado por:] \cdtIdRef{CUIR3.2-3}{Visualizar comprobante de inscripción}.
%\end{mensaje}
%
%%============================== MSG57 =================================
%\begin{mensaje}{MSG57}{Longitud de campo incorrecta}{Error}
%	\item[Ubicación:] \msjCampo
%	\item[Estatus:] Terminado
%	\item[Objetivo:] Notificar al actor que el dato ingresado en alguno de los campos del formulario no cumple con la longitud especificada.
%	\item[Redacción:] A continuación se listan las posibles redacciones que este mensaje puede tener dependiendo de lo especificado en el diccionario de datos:
%	\begin{itemize}
%		\item La longitud del campo debe ser CONDICIÓN que NÚMERO dígito(s).
%		\item La longitud del campo debe ser mayor a NÚMERO1 y menor a NÚMERO2 dígito(s).
%	\end{itemize}
%	\item[Parámetros:] El mensaje se muestra con base en los siguientes parámetros:
%	\begin{itemize}
%
%		\item CONDICIÓN1: Es la condición que debe cumplir la longitud del campo. Puede ser mayor, menor o igual.
%		\item NÚMERO: Es el número con el que se debe cumplir la condición.
%		\item NÚMERO1: La longitud del campo debe ser mayor a este número.
%		\item NÚMERO2: La longitud del campo debe ser menor a este número.
%	\end{itemize}
%	\item[Ejemplo:] Algunos ejemplos son:
%	\begin{itemize}
%		\item La longitud del campo debe ser menor que 10 dígito(s).
%		\item La longitud del campo debe ser mayor a 5 y menor a 10 dígito(s).
%	\end{itemize}
%		
%	\item[Referenciado por:] 
%	\cdtIdRef{CUPA1.1-3}{Editar ciclo escolar},
%	\cdtIdRef{CUPA1.1-5}{Editar actividad de la etapa},
%	\cdtIdRef{CUPA1.1-6}{Registrar actividad de la etapa}, 
%	\cdtIdRef{CUPA1.1-12}{Registrar etapa}, 
%	\cdtIdRef{CUPA1.2-15}{Configurar criterio}.
%	\cdtIdRef{CUGA7.1-2}{Registrar grupo}, 
%	\cdtIdRef{CUGA7.1-3}{Editar grupo}, 
%	\cdtIdRef{CUGA7.2-2}{Registrar salón}, 
%	\cdtIdRef{CUGA7.2-3}{Editar salón}, 
%	\cdtIdRef{CUGA7.3-2}{Registrar profesor}, 
%	\cdtIdRef{CUGA7.3-2-1}{Crear cuenta},
%	\cdtIdRef{CUGA7.3-3}{Editar profesor}, 
%	\cdtIdRef{CUGA7.3-6}{Registrar domicilio}, 
%	\cdtIdRef{CUGA7.3-7}{Editar domicilio}, 
%	\cdtIdRef{CUGA7.3-8}{Gestionar medios de contacto}, \cdtIdRef{CUGA7.3-19}{Cambiar contraseña}, \cdtIdRef{CUGA7.4.3-1}{Registrar ausencia temporal}, \cdtIdRef{CUGA7.4-4}{Registrar ausencia definitiva}, \cdtIdRef{CUPA1.1-6}{Registrar actividad}, \cdtIdRef{CUPA1.2-3}{Registrar criterio}, 
%	\cdtIdRef{CUPA1.2-5}{Agregar requisito}, 
%	\cdtIdRef{CUPA1.2-6}{Editar requisito},
%	\cdtIdRef{CUPA1.2-18}{Agregar actividad a convocatoria de ingreso},
%	\cdtIdRef{CUPA1.2-19}{Editar actividad de convocatoria de ingreso},
%	\cdtIdRef{CUPA1.2-20}{Registrar convocatoria de ingreso},
%	\cdtIdRef{CUPA1.2-21}{Editar convocatoria de ingreso},
%	\cdtIdRef{CUPA1.4-3}{Registrar datos personales},
%	\cdtIdRef{CUPA1.4-4}{Registrar domicilio},
%	\cdtIdRef{CUPA1.4-5}{Registrar medios de contacto},
%	\cdtIdRef{CUPA1.4-7}{Registrar información escolar},
%	\cdtIdRef{CUPA1.7-4}{Registrar psicólogo},
%	\cdtIdRef{CUPA1.7-15}{Registrar contraseña},
%	\cdtIdRef{CUPA1.7-17}{Registrar concepto},
%	\cdtIdRef{CUPA1.7-18}{Editar concepto},
%	\cdtIdRef{CUPA1.7-21}{Registrar entrevista},
%	\cdtIdRef{CUPA1.7-24}{Aprobar evaluación final},
%	\cdtIdRef{CUPA1.7-25}{Crear cuenta de psicólogo},
%	\cdtIdRef{CUPA1.7-26}{Editar cuenta del psicólogo}.
%	
%\end{mensaje}
%
%
%%============================== MSG58 =================================
%\begin{mensaje}{MSG58}{Traslape existente}{Error}
%	\item[Ubicación:] \msjCampo
%	\item[Estatus:] Edición
%	\item[Objetivo:] Notificar al actor que existe un traslape con los datos ingresados.
%	\item[Redacción:]  A continuación se listan las posibles redacciones que este mensaje puede tener dependiendo de lo especificado en el diccionario de datos:
%	\begin{enumerate}
%		\item Existe traslape entre ENTIDAD1 que desea ACCIÓN y ENTIDAD2 previamente registrada(s).
%		\item Existe traslape entre ENTIDAD1 NOMBRE1 y ENTIDAD2 NOMBRE2.
%		\item Existe traslape en el horario del profesor PROFESOR.
%		\item Existe traslape con los horarios registrados para esta actividad.
%	\end{enumerate}
%	
%	\item[Parámetros:] El mensaje se muestra con base en los siguientes parámetros:
%	\begin{itemize}
%		\item ENTIDAD1: Es la entidad que el actor desea registrar.
%		\item ENTIDAD2: Es la entidad con la que existe el traslape.
%		\item NOMBRE1: Es el nombre del elemento que el actor desea registrar.
%		\item NOMBRE2: Es el nombre del elemento con el que existe el traslape.
%		\item ACCIÓN: Es la acción que el actor desea realizar. Este parámetro puede tener los siguientes valores: registrar o editar.
%		\item PROFESOR1: Es el profesor para el cual se está registrando la materia.
%		\item MATERIAS: Es la lista de materias en las que existe traslape con respecto al horario del profesor que las imparte. La lista debe ir separada por comas y la preposición 'y' antes de la última materia. 
%	\end{itemize}
%	\item[Ejemplo:] Pueden darse los siguientes ejemplos:
%	\begin{itemize}
%		\item Existe traslape entre la materia que desea editar y materias previamente registrada(s).
%		\item Existe traslape entre la ausencia que desea registrar y una ausencia previamente registrada(s).
%		\item Existe traslape entre la materia DERECHO I y la materia DERECHO II.
%		\item Existe traslape en el horario del profesor José Alfonso Vargas Ruiz.
%		\item Existe traslape entre la vigencia que desea registrar y alguna vigencia previamente registrada(s).
%		\item Existe traslape entre el horario que desea registrar y un horario de la misma actividad previamente registrada(s)
%	\end{itemize}
%	\item[Referenciado por:] 
%	\cdtIdRef{CUGA7.1.1-2}{Agregar horario de grupo}, 
%	\cdtIdRef{CUGA7.1.1-4}{Editar horario de grupo}, 
%	\cdtIdRef{CUIR3.2-2}{Visualizar horario},
%	\cdtIdRef{CUPA1.7-9}{Registrar vigencia},
%	\cdtIdRef{CUPA1.7-10}{Editar vigencia}.
%\end{mensaje}
%
%%============================== MSG59 =================================
%\begin{mensaje}{MSG59}{Horario de profesor dentro de periodo de ausencia}{Error}
%	\item[Ubicación:] \msjPantalla
%	\item[Estatus:] Edición
%	\item[Objetivo:] Notificar al actor que el profesor tiene un horario definido en el periodo de ausencia.
%	\item[Redacción:] El profesor tiene un horario definido en el periodo de ausencia, debe excluir al profesor del horario del grupo al que fue asignado.
%	\item[Referenciado por:] 
%\end{mensaje}
%
%%============================== MSG60 =================================
%\begin{mensaje}{MSG60}{Horas de materia incorrectas}{Error}
%	\item[Ubicación:] \msjCampo
%	\item[Estatus:] Edición
%	\item[Objetivo:] Notificar al actor que la materia seleccionada no cumple con las horas estipuladas en el plan de estudios.
%	\item[Redacción:]  La materia seleccionada debe cumplir con NUMERO horas a la semana.
%	\item[Parámetros:] El mensaje se muestra con base en los siguientes parámetros:
%	\begin{itemize}
%		\item NUMERO: Número de horas a la semana que debe cumplir la materia.
%	\end{itemize}
%	\item[Ejemplo:] La materia seleccionada debe cumplir con 40 horas a la semana.
%	\item[Referenciado por:] \cdtIdRef{CUGA7.1.1-2}{Agregar horario de grupo}, \cdtIdRef{CUGA7.1.1-4}{Editar horario de grupo}.
%\end{mensaje}
%
%%============================== MSG61 =================================
%\begin{mensaje}{MSG61}{Horario de materia incorrecto}{Error}
%	\item[Ubicación:] \msjCampo
%	\item[Estatus:] Edición
%	\item[Objetivo:] Notificar al actor que la materia seleccionada no cumple con el turno correspondiente.
%	\item[Redacción:] El horario de la materia no cumple con el horario definido para el turno matutino o vespertino.
%	\item[Referenciado por:] \cdtIdRef{CUGA7.1.1-2}{Agregar horario de grupo}, \cdtIdRef{CUGA7.1.1-4}{Editar horario de grupo}.
%\end{mensaje}
%
%%===========================  MSG62 ==================================
%\begin{mensaje}{MSG62}{Correo de creación de cuenta}{Notificación}
%	\item[Ubicación:] Correo electrónico.
%	\item[Estatus:] Terminado
%	\item[Objetivo:] Notificar al actor que se le ha enviado un correo confirmando la creación de una cuenta.
%	\item[Redacción:] VALOR. \\ Se creó su cuenta exitosamente, a continuación se detalla la información de su cuenta: 
%	\\
%	Usuario: USUARIO \\
%	Link de activación de cuenta: TOKEN\\
%	\\ Si usted no solicitó esta cuenta, haga caso omiso.
%	\item[Parámetros:] El mensaje se muestra con base en los siguientes parámetros:
%	\begin{Citemize} 
%		\item VALOR: Es el nombre completo del Aspirante.
%		\item USUARIO: Es el correo electrónico del Aspirante.
%		\item TOKEN: Es el link de activación de cuenta para el Aspirante
%	\end{Citemize}
%	\item[Ejemplo:] Adrian Flores Torres. \\ Se creó su cuenta exitosamente, a continuación se detalla la información de su cuenta:\\ 
% 	Usuario: $adrian.f.cdt@gmail.com$ \\ 
% 	Link de activación de cuenta: \underline{http://acceso.Escuela Libre de Derecho/autentica} \\ 
%	Si usted no solicitó esta cuenta, haga caso omiso.\\ 
%	\item[Referenciado por:] \cdtIdRef{CUPA1.3-2}{Crear cuenta}.
%\end{mensaje}
%
%%===========================  MSG63 ==================================
%\begin{mensaje}{MSG63}{Cita de examen psicométrico}{Notificación}
%	\item[Ubicación:] Correo electrónico.
%	\item[Estatus:] Terminado
%	\item[Objetivo:] Notificar al aspirante la fecha y lugar donde debe presentarse para realizar las pruebas psicométricas.
%	\item[Redacción:] Estimado aspirante:\\
%	Para continuar con tu proceso de admisión debes presentarte el día FECHA para realizar las actividades que conlleva el examen psicométrico teniendo una duración de DURACIÓN horas teniendo que realizar las siguientes actividades presentándote en los salones asignados y horas señaladas:\\
%	\begin{table} 
%		\begin{center}
%			\begin{tabular}{|l|l|l|}
%				\hline
%				Actividad & Salón & Hora \\
%				\hline \hline 
%				\textbf{Examen Electrónico} & SALÓN EE & HORA EE \\ \hline
%				\textbf{Prueba HTP} & SALÓN HTP & HORA HTP \\ \hline
%				\textbf{Entrevista} & SALÓN ENTREVISTA & HORA ENTREVISTA \\ \hline
%			\end{tabular}
%			\caption{Horarios de actividades.}
%			\hypertarget{tb:actividadesPsicometrico}{}
%			\label{tb:actividadesPsicometrico}
%		\end{center}
%	\end{table}
%	
%	Además, como nota importante el bloque que se te asignó es el SUBBLOQUE esto con la finalidad de llevar un control sobre la realización de tus actividades por lo que deberás permanecer en dicho bloque en todo momento.
%	\item[Parámetros:] El mensaje se muestra con base en los siguientes parámetros:
%	\begin{Citemize} 
%		\item FECHA: Es la fecha que le fue asignada la cita al aspirante.
%		\item HORA: Es la hora que le fue asignada la cita al aspirante.
%		\item SALÓN: Es el salón donde debe presentarse para realizar alguna actividad del examen psicométrico.
%		\item DURACIÓN: Es la duración total que lleva la realización de todas las actividades del examen psicométrico.
%		\item SUBBLOQUE: Es el número de subbloque que se le asigna a un aspirante.
%	\end{Citemize}
%	\item[Ejemplo:] Estimado aspirante:\\
%	Para continuar con tu proceso de admisión debes presentarte el día 10/Oct/2017 para realizar las actividades que conlleva el examen psicométrico teniendo una duración de 4 horas teniendo que realizar las siguientes actividades presentándote en los salones asignados y horas señaladas:\\
%	\begin{table} 
%		\begin{center}
%			\begin{tabular}{|l|l|l|}
%				\hline
%				Actividad & Salón & Hora \\
%				\hline \hline 
%				\textbf{Examen Electrónico} & Salón 1 & 9:00 \\ \hline
%				\textbf{Prueba HTP} & Salón 2 & 11:00 \\ \hline
%				\textbf{Entrevista} & Salón 3 & 12:00 \\ \hline
%			\end{tabular}
%			\caption{Ejemplo de horarios de actividades.}
%			\hypertarget{tb:ejemploActividadesPsicometrico}{}
%			\label{tb:ejemploActividadesPsicometrico}
%		\end{center}
%	\end{table}
%	
%	Además, como nota importante el bloque que se te asignó es el 1 esto con la finalidad de llevar un control sobre la realización de tus actividades por lo que deberás permanecer en dicho bloque en todo momento.
%	\item[Referenciado por:]
%\end{mensaje}
%
%%===========================  MSG64 ==================================
%\begin{mensaje}{MSG64}{Registro de contraseña}{Notificación}
%	\item[Ubicación:] Correo electrónico.
%	\item[Estatus:] Terminado
%	\item[Objetivo:] Notificar al psicólogo que debe de ingresar al sistema para terminar su registro.
%	\item[Redacción:] VALOR, para terminar de registrar su cuenta es necesario que ingrese al sistema.\\
%	Usuario: CORREO\\
%	Contraseña: CONTRASEÑA \\
%	
%	\item[Parámetros:] El mensaje se muestra con base en los siguientes parámetros:
%	\begin{Citemize} 
%		\item VALOR: Es el nombre completo del psicólogo.
%		\item CONTRASEÑA: Es la contraseña generada por el sistema.
%		\item CORREO: Es el correo electrónico del psicólogo al cual se le envió la contraseña.
%	\end{Citemize}
%	\item[Ejemplo:] Adrian Flores Torres, para terminar de registrar su cuenta es necesario que ingrese al sistema.\\
%	Usuario: adrian.f.cdt@gmail.com\\
%	Contraseña: Adr1an?? \\
%	
%	\item[Referenciado por:]\cdtIdRef{CUPA1.7-4}{Registrar psicólogo},
%	\cdtIdRef{CUPA1.7-5}{Editar psicólogo}.
%\end{mensaje}
%
%%===========================  MSG65 ==================================
%\begin{mensaje}{MSG65}{Inscripción de materias adeudadas}{Error}
%	\item[Ubicación:] En la parte superior de la pantalla.
%	\item[Estatus:] Edición
%	\item[Objetivo:] Notificar al actor que debe inscribir todas las materias que adeuda
%	\item[Redacción:]  Debe inscribir todas las materias adeudadas.
%	\item[Referenciado por:] \cdtIdRef{CUIR3.2-2}{Visualizar horario}
%\end{mensaje}
%
%
%
%%%===========================  MSG66 ==================================
%%\begin{mensaje}{MSG66}{Información }{Error}
%%	\item[Ubicación:] En la parte superior de la pantalla.
%%	\item[Estatus:] Edición
%%	\item[Objetivo:] Notificar al actor las etapas y actividades por las cuales debe de pasar después de realizar el pago por concepto de derechos.
%%	\item[Redacción:] 
%%	
%%	
%%	
%%		\item[Parámetros:] El mensaje se muestra con base en los siguientes parámetros:
%%	\begin{Citemize} 
%%		\item VALOR: Es el nombre completo del psicólogo.
%%		\item LINK: Es la liga al sistema para registrar la contraseña.
%%	\end{Citemize}
%%	\item[Ejemplo:]
%%	\item[Referenciado por:]
%%\end{mensaje}
%
%%===========================  MSG66 ==================================
%\begin{mensaje}{MSG66}{Motivos por los que no se concretó la operación}{Error}
%	\item[Ubicación:] En la parte inferior de la pantalla.
%	\item[Estatus:] Edición
%	\item[Objetivo:] Notificar al actor que no se ha podido realizar la operación solicitada por las causas que se enlistan.
%	\item[Redacción:] No se ha podido OPERACIÓN ELEMENTO debido a que MOTIVO.
%	\item[Parámetros:] El mensaje se muestra con base en los siguientes parámetros:
%	\begin{Citemize} 
%		\item OPERACIÓN: Es la operación que no se pudo llevar acabo.
%		\item ELEMENTO: El el nombre del elemento sobre el cual se intentó llevar a cabo la operación.
%		\item MOTIVO: Es el motivo por el cual no se pudo concretar la operación.
%	\end{Citemize} 
%	\item[Ejemplo:] A continuación se muestran ejemplos de la redacción:
%	  \begin{Citemize} 
%		\item No se ha podido finalizar el registro de resultados de CENEVAL debido a que existen inconsistencias de tipo ''No encontrado''.
%		\item No se ha podido modificar el correo electrónico del psicólogo debido a que su cuenta no se encuentra en estado ''Creada''.
%	  \end{Citemize}
%	\item[Referenciado por:]
%\end{mensaje}
%
%%===========================  MSG67 ==================================
%\begin{mensaje}{MSG67}{Horario de descanso}{Notificación}
%	\item[Ubicación:] En la parte superior de la pantalla.
%	\item[Estatus:] Terminado
%	\item[Objetivo:] Notificar al actor cual será su horario de descanso durante las actividades del examen psicométrico.
%	\item[Redacción:] El horario de descanso será de HORA INICIO a HORA FIN hrs en todos los días que se realizará el examen psicométrico
%	\item[Parámetros:] El mensaje se muestra con base en los siguientes parámetros:
%	\begin{Citemize} 
%		\item HORA INICIO: Es la hora inicial calculada de acuerdo a los parámetros ingresados en Configurar Citas.
%		\item HORA FIN: Es la hora final calculada de acuerdo a los parámetros ingresados en Configurar Citas.
%	\end{Citemize}
%	\item[Ejemplo:] El horario de descanso será de 14:00 a 15:00 hrs en todos los días que se realizará el examen psicométrico
%	\item[Referenciado por:]
%\end{mensaje}
%
%%===========================  MSG68 ==================================
%\begin{mensaje}{MSG68}{Activación de cuenta}{Notificación}
%	\item[Ubicación:] Correo electrónico.
%	\item[Estatus:] Terminado
%	\item[Objetivo:] Notificar al psicólogo que su cuenta ha sido activada.
%	\item[Redacción:] VALOR, tu cuenta para realizar las actividades del examen psicométrico ha sido activada.\\LINK.
%	\item[Parámetros:] El mensaje se muestra con base en los siguientes parámetros:
%	\begin{Citemize} 
%		\item VALOR: Es el nombre completo del psicólogo.
%		\item LINK: Es la liga al sistema.
%	\end{Citemize}
%	\item[Ejemplo:] Bruno Suárez Cruz, tu cuenta para realizar las actividades del examen psicométrico ha sido activada.\\http://www.Escuela Libre de Derecho.edu.mx/
%	\item[Referenciado por:] \cdtIdRef{CUPA1.7-9}{Registrar vigencia},
%	\cdtIdRef{CUPA1.7-10}{Editar vigencia}.
%\end{mensaje}
%
%%============================== MSG69 =================================
%\begin{mensaje}{MSG69}{Finalizar evaluación de examen psicométrico}{Confirmación}
%	\item[Ubicación:] \msjEmergente
%	\item[Estatus:] Terminado
%	\item[Objetivo:] Preguntar al actor si desea finalizar el registro de la evaluación final del examen psicométrico.
%	\item[Redacción:] Una vez que finalice la evaluación final del aspirante ya no podrá modificar el resultado, el resultado pasará a revisión por parte del \cdtRef{Actor:CP}{Coordinador de Psicólogos} quién finalmente aprobará dicha evaluación. ¿Desea continuar?
%	\item[Referenciado por:] \cdtIdRef{CUPA1.1-2}{Gestionar actividades del ciclo escolar}.
%\end{mensaje}
%
%%============================== MSG70 =================================
%\begin{mensaje}{MSG70}{Gestión de conceptos finalizada}{Error}
%	\item[Ubicación:] \msjEmergente
%	\item[Estatus:] Terminado
%	\item[Objetivo:] Notificar al actor que la gestión de conceptos ya fue finalizada anteriormente.
%	\item[Redacción:] La gestión de conceptos ya fue finalizada anteriormente.
%	\item[Referenciado por:] 
%\end{mensaje}
%
%%============================== MSG71 =================================
%\begin{mensaje}{MSG71}{Archivo con inconsistencias}{Notificación}
%	\item[Ubicación:] En la parte superior de la pantalla.
%	\item[Estatus:] Edición
%	\item[Objetivo:] Notificar al actor que el archivo tiene VALOR inconsistencias.
%	\item[Redacción:] El archivo que se subió tiene VALOR inconsistencia(s).
%	\item[Parámetros:] El mensaje se muestra con base en los siguientes parámetros:
%	\begin{Citemize} 
%		\item VALOR: Es el número de inconsistencias que tiene el archivo.
%	\end{Citemize}
%    \item[Ejemplo:] El archivo que subió tiene 10 inconsistencia(s).
%	\item[Referenciado por:] \cdtIdRef{CUPA1.6.4-2}{Visualizar resultados},\cdtIdRef{CUPA1.6-5}{Visualizar resultados del EXANI-II}.
%\end{mensaje}
%
%%============================== MSG72 =================================
%\begin{mensaje}{MSG72}{Error en registro de convocatoria}{Notificación}
%	\item[Ubicación:] Ventana emergente.
%	\item[Estatus:] Edición
%	\item[Objetivo:] Notificar al actor que no puede registrar una convocatoria.
%	\item[Redacción:] No se puede registrar una convocatoria de ingreso debido a que existe una que fue aprobada
%y no ha sido asociada a un ciclo escolar.
%	\item[Referenciado por:] \cdtIdRef{CUPA1.6.4-2}{Visualizar resultados}.
%\end{mensaje}
%
%%===========================  MSG73 ==================================
%\begin{mensaje}{MSG73}{Requisitos de archivo CENEVAL}{Notificación}
%	\item[Ubicación:] En la parte superior de la pantalla.
%	\item[Estatus:] Edición
%	\item[Objetivo:] Notificar al actor los requisitos que debe cumplir el archivo de resultados CENEVAL.
%	\item[Redacción:] El archivo de resultados debe cumplir con los siguientes requisitos:\\
%	\begin{itemize}
%		
%\item La primera columna debe contener el folio ELD del aspirante.
%		\item la segunda columna debe contener el folio CENEVAL del aspirante.
%		\item La tercera columna debe contener el ICNE.
%		\item El formato del archivo debe ser .xlsx.
%		\item El peso máximo del archivo es 500 KB.
%	\end{itemize}
%	\item[Referenciado por:] \cdtIdRef{CUPA1.6-4}{Registrar resultados},
%\end{mensaje}
%
%%===========================  MSG74 ==================================
%\begin{mensaje}{MSG74}{Capacidad mínima de lugares para aplicación del EXANI-II}{Notificación}
%	\item[Ubicación:] En la parte superior de la pantalla.
%	\item[Estatus:] Edición
%	\item[Objetivo:] Notificar el faltante de lugares para aplicación del EXANI-II.
%	\item[Redacción:] Debe de seleccionar mas salones para asegurar los lugares de los aspirantes con derecho a examen.
%	\item[Referenciado por:] \cdtIdRef{CUPA1.6-2}{Reservar salones}.
%\end{mensaje}
%
%%===========================  MSG75 ===============================IU1.3-4===
%\begin{mensaje}{MSG75}{Falta información por proporcionar}{Notificación}
%	\item[Ubicación:] En la parte superior de la pantalla.
%	\item[Estatus:] Edición
%	\item[Objetivo:] Notificar el faltante de información por parte de la Escuela Libre de Derecho.
%	\item[Redacción:] Debe de ACCIÓN DETERMINADO ELEMENTO para que la operación se pueda concretar.
%	\item[Parámetros:] El mensaje se muestra con base en los siguientes parámetros:
%	\begin{Citemize} 
%		\item ACCIÓN: Es la acción que el actor desea realizar.
%		\item DETERMINADO ELEMENTO: Es un artículo determinado más el nombre del elemento sobre la cual se realizó la acción.
%	\end{Citemize}	
%	\item[Ejemplo:] A continuación se muestran ejemplos de la redacción.
%		\begin{enumerate}
%			\item Debe de subir el archivo de resultados del EXANI-II para que la operación se pueda concretar.
%			\item Debe finalizar la gestión de evaluaciones para que la operación se pueda concretar.
%		\end{enumerate}
%	\item[Referenciado por:] \cdtIdRef{CUPA1.6-5}{Visualizar resultados del EXANI-II}, \cdtIdRef{CUPA1.8.1-1}{Generar lista de aspirantes para entrevistar}. \cdtIdRef{CUPA1.8.1-1}{Generar lista de aspirantes para entrevistar}
%\end{mensaje}
%
%%===========================  MSG76 ==================================
%\begin{mensaje}{MSG76}{Inconsistencias de folios}{Notificación}
%	\item[Ubicación:] En la parte inferior de la pantalla.
%	\item[Estatus:] Edición
%	\item[Objetivo:] Notificar al actor los tipos de inconsistencias que se pueden presentar en el archivo de resultados.
%	\item[Redacción:] No encontrado: El folio del aspirante no fue encontrado en el archivo de resultados. \\
%	No pertenece: El folio encontrado en el archivo de resultados no pertenece a la Escuela Libre de Derecho.
%	\item[Referenciado por: ] \cdtIdRef{CUPA1.6-6}{Inconsistencias de Folios}
%\end{mensaje}
%
%%===========================  MSG77 ==================================
%\begin{mensaje}{MSG77}{No existen salones disponibles}{Error}
%	\item[Ubicación:] \msjSuperior.
%	\item[Estatus:] Edición
%	\item[Objetivo:] Notificar al actor que todos los salones para psicométrico están ocupados.
%	\item[Redacción:] No hay suficientes salones de entrevistas para asignarle al psicólogo, reserve más salones o elimine la asignación de aspirantes de otro psicólogo.
%	\item[Referenciado por: ] 
%\end{mensaje}
%
%%===========================  MSG78 ==================================
%\begin{mensaje}{MSG78}{Aspirantes sin asignación}{Error}
%	\item[Ubicación:] \msjSuperior.
%	\item[Estatus:] Edición
%	\item[Objetivo:] Notificar al actor que faltan aspirantes por asignar a algún psicólogo.
%	\item[Redacción:] Para poder generar las citas, primero debe asignar a todos los aspirantes a algún psicólogo.
%	\item[Referenciado por: ] 
%\end{mensaje}
%
%
%%===========================  MSG79 ==================================
%\begin{mensaje}{MSG79}{Modificación de nombre de una actividad}{Notificación}
%	\item[Ubicación:] \msjEmergente.
%	\item[Estatus:] Edición
%	\item[Objetivo:] Notificar al actor los efectos que causará el cambiar de nombre la actividad.
%	\item[Redacción:] Si modifica el nombre de la actividad, el cambio se reflejará en los calendarios anteriores y posteriores. ¿Está seguro que desea realizar esta acción?
%	\item[Referenciado por: ] \cdtIdRef{CUPA1.1-5}{Editar actividad de la etapa}
%\end{mensaje}
%
%
%%============================== MSG80 =================================
%\begin{mensaje}{MSG80}{Cargar archivo nuevamente}{Notificación}
%	\item[Ubicación:] \msjCampo.
%	\item[Estatus:] Edición
%	\item[Objetivo:] Notificar al actor que debe cargar el archivo nuevamente.
%	\item[Redacción:] Debes cargar el archivo nuevamente.
%	
%	\item[Referenciado por:] 
%\end{mensaje}IU1.3-4
%
%
%%============================== MSG81 =================================
%\begin{mensaje}{MSG81}{Limitación de convocatorias publicadas}{Error}
%	\item[Ubicación:] \msjEmergente.
%	\item[Estatus:] Edición
%	\item[Objetivo:] Informar al actor la condición para publicar una convocatoria de ingreso.
%	\item[Redacción:] No se pudo publicar la convocatoria de ingreso debido a que sólo puede existir una convocatoria de ingreso publicada. 
%	\item[Referenciado por:] 
%\end{mensaje}
%
%%============================== MSG82 =================================
%\begin{mensaje}{MSG82}{Asociación incorrecta de materias}{Error}
%	\item[Ubicación:] \msjSuperior.
%	\item[Estatus:] Edición
%	\item[Objetivo:] Notificar al actor que el número de materias asociadas es menor al que se encuentra registrado.
%	\item[Redacción:] El número de materias asociadas debe ser mayor o igual a las materias registradas en el plan de estudios.
%	\item[Referenciado por:] 
%\end{mensaje}
%
%%===========================  MSG83 ==================================
%%\begin{mensaje}{MSG83}{Máximo de aspirantes a asignar}{Confirmación}
%%	\item[Ubicación:] \msjEmergente.
%%	\item[Estatus:] Edición
%%	\item[Objetivo:] Preguntar al actor si desea asignarle más aspirantes al psicólogo no importando rebasar el número máximo de aspirantes a asignar.
%%	\item[Redacción:] Rebasa el número de aspirantes máximo por asignar a un psicólogo. ¿Desea asignarle N aspirantes más al psicólogo?
%%	\item[Parámetros:] El mensaje se muestra con base en los siguientes parámetros:
%%	\begin{Citemize} 
%%		\item N: Diferencia entre la cantidad de aspirantes seleccionados menos la cantidad máxima de aspirantes por asignar.
%%	\end{Citemize}
%%	\item[Ejemplo:] Rebasa el número de aspirantes máximo por asignIU1.3-4ar a un psicólogo. ¿Desea asignarle 3 aspirantes más al psicólogo?
%%	\item[Referenciado por:]
%%\end{mensaje}
%
%%===========================  MSG84 ==================================
%\begin{mensaje}{MSG84}{Notificación para concluir encuesta de CENEVAL}{Notificación}
%	\item[Ubicación:] Correo electrónico.
%	\item[Estatus:] Edición
%	\item[Objetivo:] Notificar a los aspirantes su deber de concluir el registro con CENEVAL contestando la encuesta.
%	\item[Redacción:] Estimado aspirante:\\
%	Para continuar con tu proceso de admisión tienes que finalizar el registro con CENEVAL. Para poder concluir tu registro debes contestar la encuesta en la plataforma de CENEVAL, recuerda que si no cuentas con el registro, no podrás continuar en la siguiente fase del proceso de admisión de la Escuela Libre de Derecho.
% 
%	\item[Referenciado por:]
%\end{mensaje}
%
%%===========================  MSG85 ==================================
%\begin{mensaje}{MSG85}{Bienvenida}{Notificación}
%	\item[Ubicación:] \msjSuperior.
%	\item[Estatus:] Edición
%	\item[Objetivo:] Dar la bienvenida a los actores del Sistema de Administración Escolar V2.0 cuando ingresan al sistema.
%	\item[Redacción:]  SALUDO, ACTOR	\\
%	Bienvenido al Sistema de Administración Escolar V2.0, diseñado para apoyar en la consulta y realización de trámites TIPO existentes en la Escuela Libre de Derecho.
%	\item[Parámetros:] El mensaje se muestra con base en los siguientes parámetros:
%	\begin{Citemize} 
%		\item SALUDO: Expresión de cortesía que se le da a los actores cada vez que ingresan al sistema. Puede ser Buenos días, Buenas tardes o Buenas noches dependiendo del horario en que se ingrese.
%		\item ACTOR: Es el perfil que tiene cada actor dentro del sistema. Puede ser Alumno, Coordinación de Control Escolar, Secretaría de Administración, Coordinador de Psicólogos o Psicólogo.
%		\item TIPO: Es el tipo de trámites que puede realizar cada actor. Para los alumnos serán trámites escolares y para los actores restantes, trámites administrativos. 
%	\end{Citemize}
%	\item[Ejemplo:] A continuación se muestran algunos ejemplos: 
%		\begin{itemize}
%			\item Buenos días, Coordinación de Control Escolar \\
%			Bienvenido al *Sistema de Administración Escolar V2.0*, diseñado para apoyar en la consulta y realización de trámites administrativos existentes en la Escuela Libre de Derecho.
%			
%			\item Buenos días, Alumno \\
%			Bienvenido al *Sistema de Administración Escolar V2.0*, diseñado para apoyar en la consulta y realización de trámites escolares existentes en la Escuela Libre de Derecho.
%		\end{itemize}
%	\item[Referenciado por:]
%\end{mensaje}
%
%%===========================  MSG86 ==================================
%\begin{mensaje}{MSG86}{Faltan evaluaciones de examen psicométrico}{IU1.3-4Notificación}
%	\item[Ubicación:] En la parte superior de la pantalla.
%	\item[Estatus:] Edición
%	\item[Objetivo:] Notificar al actor que se deben tener evaluaciones de psicométrico registradas para poder visualizar la lista de aspirantes.
%	\item[Redacción:] Debe de registrar al menos una evaluación de psicométrico para que la operación se pueda concretar.
%	\item[Referenciado por:] \cdtIdRef{CUPA1.8.1-1}{Generar lista de aspirantes para entrevistar}.
%\end{mensaje}
%
%
%%============================== MSG87 =================================
%\begin{mensaje}{MSG87}{Confirmar generación de citas}{Confirmación}
%	\item[Ubicación:] \msjEmergente.
%	\item[Estatus:] Edición
%	\item[Objetivo:] Informar al actor los efectos que se tienen al generar citas para exámenes.
%	\item[Redacción:]¿Está seguro que desea generar las citas? Las citas serán enviadas por correo electrónico a los aspirantes.                                                        
%
%	\item[Referenciado por: \cdtIdRef{CUPA1.6-9}{Generar citas para el EXANI-II}.] 
%\end{mensaje}  
%
%
%%%============================== MSG88 =================================
%%\begin{mensaje}{MSG88}{Notificación a aspirantes con pre-registro en el CENEVAL}{Notifiación}
%%	\item[Ubicación:] Correo electrónico.IU1.3-4
%%	\item[Estatus:] Edición
%%	\item[Objetivo:] Informar al actor los efectos que se tienen al generar citas para exámenes.
%%	\item[Redacción:]¿Está seguro que desea generar las citas?, Si continúa, se notificará por correo electrónico a los aspirantes y las citas no podrán ser modificadas posteriormente.
%%
%%	\item[Referenciado por:] 
%%\end{mensaje}  
%
%%===========================  MSG89 ==================================
%\begin{mensaje}{MSG89}{Cita de aplicación de EXANI-II}{Notificación}
%	\item[Ubicación:] Correo electrónico.
%	\item[Estatus:] Terminado
%	\item[Objetivo:] Notificar al aspirante la fecha y lugar donde debe presentarse para realizar el EXANI-II.
%	\item[Redacción:] Estimado aspirante:\\
%	Para continuar con tu proceso de admisión deberás presentarte el día FECHA a las HORA horas en nuestras instalaciones, deberás presentarte en el salón SALÓN para realizar la aplicación del EXANI-II. 
%	\item[Parámetros:] El mensaje se muestra con base en los siguientes parámetros:
%	\begin{Citemize} 
%		\item FECHA: Es la fecha que le fue asignada la cita al aspirante.
%		\item HORA: Es la hora que le fue asignada la cita al aspirante.
%		\item SALÓN: Es el salón donde debe presentarse para realizar alguna actividad del examen psicométrico.
%	
%	\end{Citemize}
%	\item[Ejemplo:] Estimado aspirante:\\
%	Para continuar con tu proceso de admisión deberás presentarte el día 20 de Dic del 2017 a las 18:00 horas en nuestras instalaciones, deberás presentarte en el salón 201 para realizar la aplicación del EXANI-II. 
%	\item[Referenciado por: \cdtIdRef{CUPA1.6-9}{Generar citas para el EXANI-II}.]
%\end{mensaje}
%
%%===========================  MSG90 ==================================
%\begin{mensaje}{MSG90}{Formato incorrecto en el archivo de resultados}{Notificación}
%	\item[Ubicación:] En una tabla.
%	\item[Estatus:] Terminado
%	\item[Objetivo:] Notificar al actor que existen errores en el archivo de resultados.
%	\item[Redacción:] ITEM ERROR en la columna NOMBRE.
%	\item[Parámetros:] El mensaje se muestra con base en los siguientes parámetros:
%	\begin{Citemize} 
%		\item ITEM: Es el atributo que es erróneo y puede tener los siguientes valores: formato o longitud.
%		\item ERROR: Es el error que tiene el dato y puede tener los siguientes valores: incorrecto o incorrecta.
%		\item NOMBRE: Es el nombre de columna en la que se detecto el error.
%	\end{Citemize}
%	\item[Ejemplo:] A continuación se muestran ejemplos de la redacción:
%	\begin{itemize}
%		\item Formato incorrecto en la columna Escuela Libre de Derecho.
%		\item Longitud incorrecta en la columna CENEVAL.
%		\item Formato incorrecto en la columna ICNE.
%	\end{itemize}
%	\item[Referenciado por:]
%\end{mensaje}
%
%%===========================  MSG91 ==================================
%\begin{mensaje}{MSG91}{Celdas vacías en el archivo de resultados}{Notificación}
%	\item[Ubicación:] En una tabla.
%	\item[Estatus:] Terminado
%	\item[Objetivo:] Notificar al actor que existen celdas vacías en el archivo de resultados.
%	\item[Redacción:] Existen celdas vacías.
%	\item[Referenciado por:]
%\end{mensaje}
%
%%===========================  MSG92 ==================================
%\begin{mensaje}{MSG92}{Valor fuera de rango}{Notificación}
%	\item[Ubicación:] En una tabla.
%	\item[Estatus:] Terminado
%	\item[Objetivo:] Notificar al actor que existen valores fuera del rango establecido.
%	\item[Redacción:] El valor del DATO está fuera de rango.
%	\item[Parámetros:] El mensaje se muestra con base en los siguientes parámetros:
%	\begin{Citemize} 
%		\item DATO: Es el nombre del dato que está fuera de rango.
%	\end{Citemize}
%	\item[Ejemplo:] A continuación se muestran ejemplos de la redacción:
%	\begin{itemize}
%		\item El valor del ICNE está fuera de rango.
%	\end{itemize}
%	\item[Referenciado por:]
%\end{mensaje}
%
%
%%===========================  MSG93 ==================================
%\begin{mensaje}{MSG93}{Correo de aspirante aceptado}{Notificación}
%	\item[Ubicación:] Correo electrónico.
%	\item[Estatus:] Edición
%	\item[Objetivo:] Notificar al aspirante que ha sido aceptado en la Escuela Libre de Derecho e indicarle cuándo deberá presentarse con los documentos requeridos para continuar con su inscripción.
%	\item[Redacción:] 
%	C. ASPIRANTE, por medio de la presente la Escuela Libre de Derecho tiene el placer de informarle que su solicitud de ingreso ha sido aceptada por los directivos de esta honorable institución. \\
%	
%	Reciba la más cordial de nuestras felicitaciones por la aceptación para formar parte del alumnado de nuestra Escuela y por lo cual, le solicitamos se presente en nuestras instalaciones el FECHA para comenzar los tramites de inscripción; deberá presentar los siguientes documentos: \\
%
%		DOCUMENTOS
%
%	\item[Parámetros:] El mensaje se muestra con base en los siguientes parámetros:
%	\begin{Citemize} 
%		\item ASPIRANTE: Es el nombre del aspirante como se muestra a continuación: APELLIDO PATERNO APELLIDO MATERNO NOMBRES.
%		\item FECHA: Es la fecha en que al aspirante deberá presentarse para entregar sus documentos de acuerdo con el calendario escolar, con formato: DD/MMM/AAAA.
%		\item DOCUMENTOS: Es la lista de documentos definidos en la convocatoria de admisión.
%	\end{Citemize}
%	\item[Ejemplo:] C. Granados Puerto Carlos, por medio de la presente la Escuela Libre de Derecho tiene el placer de informarle que su solicitud de ingreso ha sido aceptada por los directivos de esta honorable institución. \\
%	
%	Reciba la más cordial de nuestras felicitaciones por la aceptación para formar parte del alumnado de nuestra Escuela y por lo cual, le solicitamos se presente en nuestras instalaciones el día 07/ene/2017 para comenzar los tramites de inscripción. Para esto deberá presentar los siguientes documentos: \\
%
%		\begin{Citemize} 
%			\item CURP
%			\item Acta de nacimiento actualizada.
%			\item Identificación oficial o equivalente.
%			\item Certificado de secundaria o equivalente.
%			\item Certificado de bachillerato o equivalente.
%		\end{Citemize}
%
%	\item[Referenciado por:] \cdtIdRef{CUPA1.9-1}{Configurar lista de aspirantes}.
%\end{mensaje}
%
%%============================== MSG94 =================================
%\begin{mensaje}{MSG94}{Faltan resultados por asociar}{Error}
%	\item[Ubicación:] \msjSuperior.
%	\item[Estatus:] Edición
%	\item[Objetivo:] Notificar al actor que existen aspirantes sin resultado asociado.
%	\item[Redacción:] Existen aspirantes sin resultado asociado.
%	\item[Referenciado por:] \cdtIdRef{CUPA1.6-5}{Visualizar resultados del EXANI-II}.
%\end{mensaje}
%
%%%============================== MSG95 =================================
%\begin{mensaje}{MSG95}{Cuenta bloqueada temporalmente}{Notificación}
%	\item[Ubicación:] \msjEmergente
%	\item[Estatus:] Edición
%	\item[Objetivo:] Notificar al actor la consecuencia de exceder el número de intentos de login.
%	\item[Redacción:] Excedió el número de intentos. Intenta ingresar en 30 minutos o recupere su contraseña.
%	\item[Referenciado por:] {CUPA1.3-1}{Iniciar Sesión},
%	\cdtIdRef{CUGA7.3-2}{Registrar profesor}.
%\end{mensaje}
%
%%%============================== MSG96 =================================
%\begin{mensaje}{MSG96}{Capacidad insuficiente}{Error}
%	\item[Ubicación:] \msjEmergente
%	\item[Estatus:] Edición
%	\item[Objetivo:] Notificar al actor que la capacidad total de los salones reservados para alguna de las actividades del examen EXANI-II o examen psicométrico no es suficiente para recibir a todos los aspirantes contemplados para dichas evaluaciones.
%	\item[Redacción:] La capacidad total de los salones para ARTICULO ACTIVIDAD no es suficiente para atender a todos los aspirantes contemplados para dicha actividad.
%	\item[Parámetros:] El mensaje se muestra con base en los siguientes parámetros:
%	\begin{Citemize} 
%		\item ARTICULO: Es la parte de la oración que se ocupa de expresar el género (masculino/femenino).
%		\item ACTIVIDAD: Es el nombre de la actividad de la que se detectó la insuficiencia de capacidad.
%	\end{Citemize}
%	\item[Ejemplo:] A continuación se muestran ejemplos de la redacción:
%	\begin{itemize}
%		\item La capacidad total de los salones para el examen EXANI-II no es suficiente para atender a todos los aspirantes contemplados para dicha actividad.
%		\item La capacidad total de los salones para el examen electrónico no es suficiente para atender a todos los aspirantes contemplados para dicha actividad.
%		\item La capacidad total de los salones para la entrevista no es suficiente para atender a todos los aspirantes contemplados para dicha actividad.
%	\end{itemize}
%	\item[Referenciado por:] 
%\end{mensaje}
%
%%%============================== MSG97 =================================
%\begin{mensaje}{MSG97}{Incumplimiento de criterios}{Error}
%	\item[Ubicación:] \msjSuperior.
%	\item[Estatus:] Edición
%	\item[Objetivo:] Notificar al actor que no se ha cumplido alguno de los criterios para poder calcular los parámetros de la configuración de citas.
%	\item[Redacción:] ELEMENTO debe ser CONDICIÓN a ELEMENTO.
%	\item[Parámetros:] El mensaje se muestra con base en los siguientes parámetros:
%	\begin{itemize}
%		\item ELEMENTO: Es el elemento del cuál se toman las cantidades.
%		\item CONDICIÓN: Es la relación que debe existir con la cantidad del ELEMENTO que establece una referencia. Los valores que puede tener son: mayor, menor, igual, menor o igual y mayor o igual.
%	\end{itemize}
%	\item[Ejemplo:]
%	\begin{itemize}
%		\item La jornada laboral debe ser mayor o igual a la duración del intervalo.
%		\item La duración del intervalo debe ser mayor a la duración del descanso.
%		\item La duración del intervalo debe ser mayor a cero.
%	\end{itemize}
%	\item[Referenciado por:] 
%\end{mensaje}
%
%%%============================== MSG98 =================================
%\begin{mensaje}{MSG98}{Aspirantes sin calificación asignada}{Error}
%	\item[Ubicación:] \msjSuperior.
%	\item[Estatus:] Edición
%	\item[Objetivo:] Notificar al actor que existen aspirantes sin alguna calificación asignada para alguna de las evaluaciones registradas en la convocatoria de ingreso.
%	\item[Redacción:] Existen aspirantes sin calificación asignada en alguna de las evaluaciones. Intente subir los archivos nuevamente.
%	\item[Referenciado por:] \cdtIdRef{CUPA1.2-28}{Gestionar evaluaciones}.
%\end{mensaje}
%
%%===========================  MSG99 ==================================
%\begin{mensaje}{MSG99}{Requisitos de archivo de evaluaciones}{Notificación}
%	\item[Ubicación:] En la parte superior de la pantalla.
%	\item[Estatus:] Edición
%	\item[Objetivo:] Notificar al actor los requisitos que debe cumplir el archivo de evaluaciones.
%	\item[Redacción:]El archivo de resultados de evaluaciones debe cumplir con los siguientes requisitos: \\  
%	\begin{itemize}
%		\item El nombre de las columnas debe estar en la primera fila del documento.
%		\item La primera columna debe contener el folio ELD del aspirante.
%		\item La segunda columna debe contener la calificación final de la evaluación de cada aspirante.
%		\item El formato del archivo debe ser XSLX.
%		\item El tamaño máximo del archivo es de 500 KB.
%	\end{itemize}
%	\item[Referenciado por:] \cdtIdRef{CUPA1.2-29}{Cargar archivo de evaluación}.
%\end{mensaje}
%
%%===========================  MSG100 ==================================
%\begin{mensaje}{MSG100}{Convocatoria sin evaluaciones}{Notificación}
%	\item[Ubicación:] En la sección evaluaciones.
%	\item[Estatus:] Edición
%	\item[Objetivo:] Notificar al actor la inexistencia de evaluaciones.
%	\item[Redacción:] No se registraron evaluaciones en la convocatoria actual.  
%	\item[Referenciado por:] \cdtIdRef{CUPA1.8.5-3}{Registrar evaluación de entrevistas}.
%\end{mensaje}
%
%%===========================  MSG101 ==================================
%\begin{mensaje}{MSG101}{Evaluaciones}{Notificación}
%	\item[Ubicación:] \msjPantalla
%	\item[Estatus:] Edición
%	\item[Objetivo:] Notificar al actor las evaluaciones que debe presentar y en qué fecha para continuar con el proceso de admisión.
%	\item[Redacción:] 
%	    En esta etapa deberá asistir a la Escuela Libre de Derecho el día indicado a continuación con el fin de presentar las evaluaciones indicadas en la convocatoria de ingreso: \\
%	    NÚM. EVALUACIÓN PERIODO DE APLICACIÓN.
%	
%	    Una vez concluidas las evaluaciones deberá esperar su resultado para continuar con la entrevista presencial.
%	    
%	\item[Parámetros:] El mensaje se muestra con base en los siguientes parámetros:
%	\begin{Citemize} 
%	    \item NÚM: Es el número evaluación del listado de evaluaciones.
%	    \item EVALUACIÓN: En el nombre de la evaluación indicado en la convocatoria de ingreso.
%	    \item PERIODO DE APLICACIÓN: Es el periodo en el que se debe presentar la evaluación.
%	\end{Citemize}
%	\item[Ejemplo:]
%	    En esta etapa deberá asistir a la Escuela Libre de Derecho el día indicado a continuación con el fin de presentar las evaluaciones indicadas en la convocatoria de ingreso: \\
%	    \begin{itemize}
%		\item 1. Examen psicométrico el día 02/mar/2018
%		\item 2. EXANI-II (CENEVAL) el día 10/mar/2018
%		\item 3. Entrevista del 17/mar/2018 al 25/mar/2018
%	    \end{itemize}
%	
%	    Una vez concluidas las evaluaciones deberá esperar su resultado para continuar con la entrevista presencial.
%	\item[Referenciado por:] 
%\end{mensaje}
%
%
%%===========================  MSG103 ==================================
%\begin{mensaje}{MSG103}{Cuenta con registro}{Notificación}
%	\item[Ubicación:] \msjPantalla
%	\item[Estatus:] Edición
%	\item[Objetivo:] Notificar al actor que la cuenta con la que desea ingresar fue registrada en otra convocatoria.
%	
%	\item[Redacción:] Estimado aspirante, la cuenta con la que desea ingresar fue asociada a otra convocatoria, para que pueda iniciar con el proceso de admisión es necesario que acceda al link de creación de cuenta, el cual se encuentra en la convocatoria actual.
%	
%	\item[Referenciado por:] 
%\end{mensaje}
%
%
%%===========================  MSG104 ==================================
%\begin{mensaje}{MSG104}{Cuenta activada}{Notificación}
%	\item[Ubicación:] \msjPantalla
%	\item[Estatus:] Edición
%	\item[Objetivo:] Notificar al actor que la cuenta que desea activar ya se encuentra activa.
%	\item[Redacción:] La cuenta que se desea activar ya fue activada, para ingresar al sistema es necesario que ingrese por Login.
%	\item[Referenciado por:] 
%\end{mensaje}
%
%
%%============================== MSG105 =================================
%\begin{mensaje}{MSG105}{La confirmación de correo electrónico no coincide}{Error}
%	\item[Ubicación:] \msjCampo.
%	\item[Estatus:] Edición
%	\item[Objetivo:] Notificar al actor que la confirmación del correo electrónico no coincide.
%	\item[Redacción:] El correo electrónico y su confirmación no coinciden. 
%	\item[Referenciado por:]
%\end{mensaje}
%
%%===========================  MSG106 ==================================
%\begin{mensaje}{MSG106}{Información para agendar entrevista}{Notificación}
%	\item[Ubicación:] \msjPantalla
%	\item[Estatus:] Edición
%	\item[Objetivo:] Notificar al actor sobre las entrevista que deberá realizar después de obtener el resultado de las evaluaciones presentadas anteriormente y en qué fecha podrá agendar la cita para presentarse en la Escuela Libre de Derecho.
%	\item[Redacción:] 
%	    Para continuar con su proceso de admisión es necesario que se presente a una entrevista con el personal de la Escuela Libre de Derecho en la que se evaluarán algunos aspectos de su comportamiento. \\
%	    
%	    Las entrevistas se llevarán a cabo a partir del día FECHA1 y hasta el FECHA2. \\
%	    
%	    Para agendar su cita para la entrevista, consulte los horarios disponibles oprimiendo el botón Siguiente a partir del día FECHA3. \\
%	    
%	    Información de cita: \\
%	    Fecha: FECHA4 \\
%	    Hora: HORA \\
%	    Salón: SALÓN
%	    
%	\item[Parámetros:] El mensaje se muestra con base en los siguientes parámetros:
%	\begin{Citemize} 
%	    \item FECHA1: Es la fecha de inicio del periodo de entrevistas indicado en el calendario del proceso de admisión.
%	    \item FECHA2: Es la fecha de término del periodo de entrevistas indicado en el calendario del proceso de admisión.
%	    \item FECHA3: Es la fecha de inicio para agendar citas de entrevistas.
%	    \item FECHA4: Es la fecha de la cita para la entrevista del Aspirante. Si aún no se ha agendado la cita, se mostrará ''No agendado''.
%	    \item HORA: Es la hora de la cita para la entrevista del Aspirante. Si aún no se ha agendado la cita, se mostrará ''No agendado''.
%	    \item SALÓN: Es el salón en donde se realizará la entrevista del Aspirante. Si aún no se ha agendado la cita, se mostrará ''No agendado''.
%	\end{Citemize}
%	\item[Ejemplo:]
%	    Para continuar con su proceso de admisión es necesario que se presente a una entrevista con el personal de la Escuela Libre de Derecho en la que se evaluarán algunos aspectos de su comportamiento. \\
%	    
%	    Las entrevistas se llevarán a cabo a partir del día 10/abr/2017 y hasta el 15/abr/2017. \\
%	    
%	    Para agendar su cita para la entrevista, consulte los horarios disponibles oprimiendo el botón Siguiente a partir del día 7/abr/2017. \\
%	    
%	    Información de cita: \\
%	    Fecha: 11/abr/2017 \\
%	    Hora: 10:00 \\
%	    Salón: Salón 1
%	    
%	\item[Referenciado por:] \cdtIdRef{CUPA1.8.4-1}{Seleccionar horario de entrevista}.
%\end{mensaje}
%
%%============================== MSG107 =================================
%\begin{mensaje}{MSG107}{No hay horarios para entrevista disponibles}{Error}
%	\item[Ubicación:] \msjSuperior.
%	\item[Estatus:] Edición
%	\item[Objetivo:] Notificar al actor que no existen horarios disponibles para cambiar su cita.
%	\item[Redacción:] No existen horarios disponibles para cambiar su cita. 
%	\item[Referenciado por:] \cdtIdRef{CUPA1.8.4-1}{Seleccionar horario de entrevista}.
%\end{mensaje}
%
%
%%============================== MSG108 =================================
%\begin{mensaje}{MSG108}{Términos y condiciones para el manejo de datos personales}{}
%	\item[Ubicación:] 
%	\item[Estatus:] Edición
%	\item[Objetivo:] Informar al actor sobre los términos y condiciones.
%	\item[Redacción:]
%	\begin{center}
%		\textbf{Otorgo mi consentimiento y acepto el siguiente aviso de privacidad:}\\
%
%		\underline{\textbf{AVISO DE PRIVACIDAD}}\\
%
%		\underline{\textbf{IDENTIDAD Y DOMICILIO DEL RESPONSABLE.}}\\
%		
%	\end{center}
%Con fundamento en los artículos 15 y 16 de la Ley Federal de Protección de Datos Personales en Posesión de Particulares, la Escuela Libre de Derecho (en adelante también referida como la 'Escuela'), por conducto de la Lic. Renata Laura Sandoval Sánchez, con domicilio en Doctor José María Vertiz número 12, esquina Arcos de Belén, Colonia Doctores, Delegación Cuauhtémoc, 06720 Ciudad de México, Ciudad de México, ES RESPONSABLE del uso y protección de sus datos personales que Usted como Titular proporcionará en la hoja de ingreso como aspirante a ser alumno de la Escuela.\\
%\begin{center}
%	\underline{\textbf{FINALIDADES DEL TRATAMIENTO DE LOS DATOS PERSONALES.}}\\	
%\end{center}
%	\begin{UClist}
%	Las finalidades del tratamiento de sus datos personales:
%	\begin{itemize}
%		\item Apertura de un expediente como candidato de ser alumno en la Escuela.
%		\item Asignación de una matrícula como candidato a ser alumno en la Escuela.
%		
%	\end{itemize}	
%	
%\end{UClist}
%\begin{center}
%	\underline{\textbf{DATOS PERSONALES QUE SE TRATARAN.}}\\	
%\end{center}
%Los datos personales que serán requeridos para cumplir con las finalidades señaladas en el párrafo anterior serán los siguientes:\\
%
%\begin{itemize}
%	\item Apellido Paterno del Titular.
%	\item Apellido Materno del Titular.
%	\item Nombre (s) del Titular.
%	\item Sexo del Titular.
%	\item Edad del Titular.
%	\item CURP del Titular.
%	\item Fecha de Nacimiento del Titular.
%	\item Ciudad de Nacimiento del Titular.
%	\item Entidad Federativa de Nacimiento del Titular.
%	\item País de Nacimiento del Titular.
%	\item Número de Teléfono Principal del Titular.
%	\item Número de Teléfono Celular del Titular.
%	\item Otro número telefónico del Titular.
%	\item Nombre completo de quien depende económicamente el Titular para el sostenimiento de sus estudios.
%	\item Correo electrónico de quién depende económicamente el	Titular para el sostenimiento de sus estudios.
%	\item Correo electrónico del Titular.
%	\item Calle y número del Domicilio (Casa) del Titular.
%	\item Delegación del Domicilio (Casa) del Titular.
%	\item Código Postal del Domicilio (Casa) del Titular.
%	\item Colonia del Domicilio (Casa) del Titular.
%	\item Entidad Federativa del Domicilio (Casa) del Titular.
%	\item Código Postal del Domicilio (Casa) del Titular.
%	\item Nombre de la secundaria a la cual el Titular asistió.
%	\item Nombre de la secundaria a la cual el Titular asistió.
%	\item Nombre de la preparatoria a la cual el Titular asistió
%	\item Si el Titular actualmente adeuda materias del bachillerato.
%	\item Nombre de las materias que el Titular adeuda.
%\end{itemize}
%	Adicionalmente, se protegerán los datos personales contenidos en la documentación que Usted como aspirante entrega a la Escuela, mismos que se enlistan a continuación.
%	\begin{enumerate}
%		\item Certificado de Secundaria.
%		\item Certificado de Bachillerato o historial académico.
%		\item Acta de Nacimiento.
%		\item CURP.
%	\end{enumerate}
%\begin{center}
%	\underline{\textbf{TRANSFERENCIAS DE DATOS.}}\\	
%\end{center}
%La Escuela no llevará a cabo transferencias de los datos personales recabados y proporcionados por el Titular.\\
%\underline{\textbf{MEDIOS PARA EL EJERCICIO DE LOS DERECHOS ARCO Y MEDIOS PARA LIMITAR}}\\ \underline{\textbf{USO O DIVULGACIÓN DE SUS DATOS PERSONALES.}} Usted tiene derecho a conocer qué datos personales tenemos de usted, para qué se utilizan y las condiciones de uso que se les da (Acceso). Asimismo, es su derecho solicitar la corrección de su información personal en caso de que esté desactualizada, sea inexacta o incompleta (Rectificación); que será eliminada de los registros o bases de datos cuando considere que la misma no está siendo utilizada conforme a los principios deberes y obligaciones previstas en la normativa (Cancelación); así como oponerse al uso de sus datos personales para fines específicos (Oposición). Estos derechos se conocen como Derechos ARCO.
%Cualquier solicitud de limitación en el uso o divulgación de sus datos personales o cualquier acto relacionado con el tratamiento de sus datos personales, así como el ejercicio de los Derechos ARCO, usted como Titular se podrá comunicar directamente al número de teléfono 5588 0211 o al correo \underline{mgonzalez@eld.edu.mx} con la ENCARGADA la C. Mayra González Sánchez, quién le hará saber los requisitos de su solicitud, información que deberá acompañar a ésta, plazos de respuesta y los medios de respuesta.\\
%\begin{center}
%	\underline{\textbf{REVOCACIÓN DEL CONSENTIMIENTO.}}\\	
%\end{center}
%Cualquier solicitud de revocación en el uso de sus datos personales, usted como Titular se podrá comunicar directamente al número de teléfono 5588 0211 o al correo \underline{mgonzalez@.eld.edu.mx} con la ENCARGADA la C. Mayra González Sánchez, quién le hará saber los requisitos de su solicitud, información que deberá acompañar a ésta, plazos de respuesta y los medios de respuesta.\\
%\begin{center}
%	\underline{\textbf{CAMBIOS AL AVISO DE PRIVACIDAD.}}\\	
%\end{center}
%El procedimiento para el cambio de este aviso de privacidad consiste en la obtención de la autorización correspondiente por parte de la Junta Directiva de la Escuela, asimismo, el medio por el que se comunicará el contenido de dicho cambio, en su caso, será mediante la publicación en el dominio de internet: \underline{http://wvvw.eld.edu.mx.}
%	\item[Referenciado por:] \cdtIdRef{CUPA1.8.4-1}{Seleccionar horario de entrevista}.
%\end{mensaje}
%
%
%%============================== MSG109 =================================
%\begin{mensaje}{MSG109}{Correo de aceptación en el proceso de selección}{}
%	\item[Ubicación:] 
%	\item[Estatus:] Edición
%	\item[Objetivo:] Notificar al actor su estatus de aceptado en el proceso de selección.
%	\item[Redacción:]
%	
%	\begin{flushright}
%		Ciudad de México, a FECHA         
%	\end{flushright}  
%	NOMBRE\\
%	\textbf{Presente\\}
%	Conforme al proceso de selección para el ciclo escolar CICLO, de acuerdo al resultado que obtuvo en sus evaluaciones, usted ha pasado a la siguiente etapa del proceso, la cual consiste en una entrevista personal.\\
%	Para agendar la cita de su entrevista, es necesario ingrese al sistema con su cuenta personal, a
%	partir del PERIODO.\\ 
%	Reciba un cordial saludo.
%	\begin{center}
%		Atentamente.
%	\end{center}
%	\begin{center}
%		\textbf{Renata Sandoval Sánchez\\}
%		Secretaria de Administración
%		
%	\end{center}
%
%	\item[Parámetros:] El mensaje se muestra con base en los siguientes parámetros:
%	\begin{Citemize} 
%		\item FECHA: Es la fecha en la cual se envía el correo electrónico.
%		\item NOMBRE: Es el nombre del Aspirante.
%		\item CICLO: Es el nombre del ciclo que esta asociada a la convocatoria actual.
%		\item PERIODO: Es el periodo que se definió en la convocatoria actual para la calendarización de entrevistas. 
%	\end{Citemize}
%	\item[Ejemplo:]
%	\begin{flushright}
%		Ciudad de México, a 16 de octubre del 2017         
%	\end{flushright}  
%	Adrian Flores Torres\\
%	\textbf{Presente\\}
%	Conforme al proceso de selección para el ciclo escolar 2017-2018, de acuerdo al resultado que obtuvo en sus evaluaciones, usted ha pasado a la siguiente etapa del proceso, la cual consiste en una entrevista personal.\\
%	Para agendar la cita de su entrevista, es necesario ingrese al sistema con su cuenta personal, a
%	partir del 1 de noviembre del 2017 al 10 de noviembre del 2017.\\ 
%	Reciba un cordial saludo.
%	\begin{center}
%		Atentamente.
%	\end{center}
%	\begin{center}
%		\textbf{Renata Sandoval Sánchez\\}
%		Secretaria de Administración
%		
%	\end{center}
%	
%	\item[Referenciado por:] 
%\end{mensaje}
%
%%============================== MSG110 =================================
%\begin{mensaje}{MSG110}{Correo de rechazo en el proceso de selección}{}
%	\item[Ubicación:] 
%	\item[Estatus:] Edición
%	\item[Objetivo:] Notificar al actor su estatus de rechazado en el proceso de selección.
%	\item[Redacción:]
%	
%	\begin{flushright}
%		Ciudad de México, a FECHA         
%	\end{flushright}  
%	NOMBRE\\
%	\textbf{Presente\\}
%	Conforme al proceso de selección para el ciclo escolar CICLO, de acuerdo al resultado que obtuvo en las evaluaciones, lamento informarle que usted no alcanzó la calificación requerida para poder pasar a la siguiente etapa.\\
%	Con base en lo anterior, ha concluido su proceso, lo cual no obsta para que pueda aplicar nuevamente el próximo año, si así es de su interés.\\
%	
%	Reciba un cordial saludo.
%	\begin{center}
%		Atentamente.
%	\end{center}
%	\begin{center}
%		\textbf{Renata Sandoval Sánchez\\}
%		Secretaria de Administración
%		
%	\end{center}
%	
%	\item[Parámetros:] El mensaje se muestra con base en los siguientes parámetros:
%	\begin{Citemize} 
%		\item FECHA: Es la fecha en la cual se envía el correo electrónico.
%		\item NOMBRE: Es el nombre del Aspirante.
%		\item CICLO: Es el nombre del ciclo que esta asociada a la convocatoria actual.
%	\end{Citemize}
%	\item[Ejemplo:]
%	
%	\begin{flushright}
%		Ciudad de México, a 16 de octubre del 2017         
%	\end{flushright}  
%	Adrian Flores Torres\\
%	\textbf{Presente\\}
%	Conforme al proceso de selección para el ciclo escolar 2017-2018, de acuerdo al resultado que obtuvo en las evaluaciones, lamento informarle que usted no alcanzó la calificación requerida para poder pasar a la siguiente etapa.\\
%	Con base en lo anterior, ha concluido su proceso, lo cual no obsta para que pueda aplicar nuevamente el próximo año, si así es de su interés.\\
%	
%	Reciba un cordial saludo.
%	\begin{center}
%		Atentamente.
%	\end{center}
%	\begin{center}
%		\textbf{Renata Sandoval Sánchez\\}
%		Secretaria de Administración
%		
%	\end{center}
%	
%	
%	\item[Referenciado por:] 
%\end{mensaje}
%
%
%%===========================  MSG111 ==================================
%\begin{mensaje}{MSG111}{Información de resultado}{Notificación}
%	\item[Ubicación:] \msjPantalla
%	\item[Estatus:] Edición
%	\item[Objetivo:] Notificar al actor la fecha en la que podrá consultar su resultado de ingreso.
%	\item[Redacción:] Tu resultado estará disponible a partir del FECHA por lo que tendrás que revisar tu correo electrónico en esta fecha.
%	\item[Parámetros:] El mensaje se muestra con base en los siguientes parámetros:
%	\begin{Citemize} 
%		\item FECHA: Es la fecha a partir de la cual se enviará el correo electrónico con el resultado al aspirante.
%	\end{Citemize}
%	\item[Ejemplo:] A continuación se muestran ejemplos de la redacción:
%	\begin{itemize}
%		\item Tu resultado estará disponible a partir del 26/jun/2017 por lo que tendrás que revisar tu correo electrónico en esta fecha.
%	\end{itemize}
%	\item[Referenciado por:] \cdtIdRef{CUPA1.9-5}{Resultados}.
%\end{mensaje}
%
%
%%===========================  MSG112 ==================================
%\begin{mensaje}{MSG112}{Etapa fuera de periodo}{Notificación}
%	\item[Ubicación:] \msjPantalla
%	\item[Estatus:] Edición
%	\item[Objetivo:] Notificar al actor que se la etapa en la que se encuentra ya no está vigente.
%	\item[Redacción:] Estimado aspirante, el periodo de la etapa en la que se encuentra ha concluido por lo que no podrá continuar con su proceso de admisión a la Escuela Libre de Derecho.
%	\item[Referenciado por:] \cdtIdRef{CUPA1.3-1}{Iniciar Sesión}.
%\end{mensaje}
%
%
%%============================== MSG113 =================================
%\begin{mensaje}{MSG113}{Cuenta registrada}{Notificación}
%	\item[Ubicación:] \msjSuperior
%	\item[Estatus:] Edición
%	\item[Objetivo:] Notificar al actor que su cuenta fue registrada de forma exitosa y su deber de activarla.
%	\item[Redacción:] El registro de su información se realizo de forma exitosa, para activar su cuenta es necesario que ingrese a la liga que fue enviada a su correo electrónico.
%	\item[Referenciado por:] \cdtIdRef{CUPA1.3-2}{Crear cuenta}. %%CAMBIAR EN EL CU
%\end{mensaje}
%
%
%%============================== MSG114 =================================
%\begin{mensaje}{MSG114}{No existe una convocatoria vigente}{Notificación}
%	\item[Ubicación:] \msjSuperior
%	\item[Estatus:] Edición
%	\item[Objetivo:] Notificar al actor que no existe una convocatoria vigente para realizar su proceso de admisión.
%	\item[Redacción:] No existe una convocatoria vigente por lo que no podrá acceder al sistema. Para realizar su proceso de admisión espere la publicación de la convocatoria del siguiente ciclo escolar.
%	\item[Referenciado por:] \cdtIdRef{CUPA1.3-1}{Iniciar Sesión}. 
%\end{mensaje}
%
%%============================== MSG115 =================================
%\begin{mensaje}{MSG115}{El Aspirante no pertenece a la convocatoria vigente}{Notificación}
%	\item[Ubicación:] \msjSuperior
%	\item[Estatus:] Edición
%	\item[Objetivo:] Notificar al aspirante que no se encuentra 
%	\item[Redacción:] Su cuenta no se encuentra asociada a la convocatoria de ingreso actual. Para poder acceder al sistema deberá realizar su registro nuevamente.
%	\item[Referenciado por:] \cdtIdRef{CUPA1.3-1}{Iniciar Sesión}.
%\end{mensaje}
%
%%============================== MSG115 =================================
%\begin{mensaje}{MSG116}{Cuenta sin activar}{Notificación}
%	\item[Ubicación:] \msjSuperior
%	\item[Estatus:] Edición
%	\item[Objetivo:] Notificar al aspirante que su cuenta aún no esta activada.
%	\item[Redacción:] Su cuenta aún no ha sido activada. Para activarla ingrese al link enviado al correo electrónico que ingreso en el registro de la cuenta.
%	\item[Referenciado por:] \cdtIdRef{CUPA1.3-1}{Iniciar Sesión}.
%\end{mensaje}
%
%
%%============================== MSG117 =================================
%\begin{mensaje}{MSG117}{Error con el token de activación de cuenta}{Error}
%	\item[Ubicación:] Ventana emergente.
%	\item[Estatus:] Terminado
%	\item[Objetivo:] Notificar al actor que el token para poder activar su cuenta no es inválido o ha expirado.
%	\item[Redacción:] El token de activación de cuenta no es válido o ha expirado, para obtener un nuevo token debe de realizar un nuevo registro de cuenta.
%	\item[Referenciado por:] \cdtIdRef{CUPA1.3-3}{Recuperar contraseña}.
%\end{mensaje}
%
%
%%============================== MSG118 =================================
%\begin{mensaje}{MSG118}{Correo de aspirante rechazado}{Notificación}
%	\item[Ubicación:] Correo electrónico.
%	\item[Estatus:] Edición
%	\item[Objetivo:] Notificar al aspirante que ha sido rechazado en la Escuela Libre de Derecho.
%	\item[Redacción:] C. ASPIRANTE, Conforme al proceso de selección para el ciclo escolar CICLO, lamentamos informarle que usted no cumplió con todos los requisitos para formar parte de la Escuela Libre de Derecho, lo cual no obsta para que pueda aplicar nuevamente el próximo año, si así es de su interés. \\
%	Reciba un cordial saludo
%	\begin{center}
%		Atentamente,
%	\end{center}
%	\begin{center}
%		Escuela Libre de Derecho.
%	\end{center}
%	\item[Parámetros:] El mensaje se muestra con base en los siguientes parámetros:
%	\begin{Citemize} 
%		\item ASPIRANTE: Es el nombre completo del aspirante.
%		\item CICLO: Ciclo escolar vigente.
%	\end{Citemize}
%	\item[Referenciado por:] \cdtIdRef{CUPA1.9-1}{Configurar lista de aspirantes}.
%\end{mensaje}
%
%
%%============================== MSG119 =================================
%\begin{mensaje}{MSG119}{Convocatoria sin aprobar}{Error}
%	\item[Ubicación:] Ventana emergente.
%	\item[Estatus:] Edición
%	\item[Objetivo:] Notificar al actor que primeramente se debe aprobar la convocatoria de ingreso en el sistema para poder acceder a la pantalla.
%	\item[Redacción:] Para poder acceder a esta pantalla primero se debe aprobar la convocatoria de ingreso.
%	\item[Referenciado por:]
%\end{mensaje}
%
%
%%============================== MSG120 =================================
%\begin{mensaje}{MSG120}{Ayuda para registro de la sede CENEVAL}{Notificación}
%	\item[Ubicación:] Icono de ayuda.
%	\item[Estatus:] Terminado
%	\item[Objetivo:] Informar al actor la forma correcta de ingresar la sede CENEVAL.
%	\item[Redacción:] El número de sede que debe registrar, es proporcionado por el CENEVAL a la Escuela Libre de Derecho cuando solicita los folios correspondientes a la convocatoria de ingreso. En el caso de que se cuente con mas de una sede, debe de registrar una diferente para cada aplicación.
%	\item[Referenciado por:] .
%\end{mensaje}
%
%
%%============================== MSG121 =================================
%\begin{mensaje}{MSG121}{Citas sin aprobar}{Error}
%	\item[Ubicación:] Ventana emergente.
%	\item[Estatus:] Edición
%	\item[Objetivo:] Notificar al actor que primeramente se deben aprobar las citas en el sistema para poder acceder a la pantalla.
%	\item[Redacción:] Para poder acceder a esta pantalla primero se debe aprobar la generación de citas.
%	\item[Referenciado por:]
%\end{mensaje}
%
%%============================== MSG122 =================================
%\begin{mensaje}{MSG122}{No existe información registrada en la configuración de citas}{Error}
%	\item[Ubicación:] Ventana emergente.
%	\item[Estatus:] Edición
%	\item[Objetivo:] Notificar al actor que primeramente se debe registrar información de la configuración de citas en el sistema para poder acceder a la pantalla.
%	\item[Redacción:] Para poder acceder a esta pantalla primero se debe registrar información de la configuración de citas.
%	\item[Referenciado por:]
%\end{mensaje}
%
%%============================== MSG123 =================================
%\begin{mensaje}{MSG123}{Aceptar términos y condiciones para continuar}{Error}
%	\item[Ubicación:] \msjSuperior
%	\item[Estatus:] Edición
%	\item[Objetivo:] Notificar al actor que debe aceptar los términos y condiciones bajo los cuales se llevará a cabo el proceso de admisión a la ELD.
%	\item[Redacción:] Para continuar debe aceptar los términos y condiciones.
%	\item[Referenciado por:] \cdtIdRef{CUGA7.4-4}{Registrar ausencia definitiva}.
%\end{mensaje}
%
%
%%============================== MSGX =================================
%% Mensaje para CENEVAL
%% \begin{mensaje}{MSGX}{Folios incorrectos}{Notificación}
%%	\item[Ubicación:] \msjEmergente
%%	\item[Estatus:] Terminado
%%	\item[Objetivo:] Notificar al actor que el archivo ingresado contiene información de folios incorrecta.
%%	\item[Redacción:] Existe CARACTERÍSTICA de folios en el archivo seleccionado.
%%	\item[Parámetros:] El mensaje se muestra con base en los siguientes parámetros:
%%	\begin{itemize}
%%		\item CARACTERÍSTICA: Es la característica de los folios en el archivo seleccionado. Puede ser excedente, déficit o duplicidad.
%%	\end{itemize}
%%	\item[Ejemplo:] Existe duplicidad de folios en el archivo seleccionado.
%%	\item[Referenciado por:] 
%%\end{mensaje}
%
%%============================== MSGX =================================
%%Mensaje para CENEVAL
%%\begin{mensaje}{MSGX}{Operación necesaria sin realizar}{Notificación}
%%	\item[Ubicación:] \msjEmergente
%%	\item[Estatus:] Terminado
%%	\item[Objetivo:] Notificar al actor que es necesario realizar una operación para continuar.
%%	\item[Redacción:] Es necesario que complete OPERACIÓN para continuar.
%%	\item[Parámetros:] El mensaje se muestra con base en los siguientes parámetros:
%%	\begin{itemize}
%%		\item OPERACIÓN: Es la operación que debe ser realizada por el actor.
%%	\end{itemize}
%%	\item[Ejemplo:] Es necesario que complete la encuesta para continuar.
%%	\item[Referenciado por:] 
%%\end{mensaje}

