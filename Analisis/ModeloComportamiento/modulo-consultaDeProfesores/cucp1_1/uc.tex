
\begin{UseCase}{CUCP1.1}{Visualizar información}
    {
    Para que un \cdtRef{Actor:alumno}{Alumno}, \cdtRef{Actor:profesor}{Profesor} o  \cdtRef{Actor:invitado}{Invitado} pueda localizar a un profesor, es necesario que conozca la información a detalle del mismo, esto se logra gracias a la operación visualizar, dicha acción le permite conocer la información publica del profesor.
    
}
    

    \UCitem{Versión}{1.0}
    \UCccsection{Administración de Requerimientos}
    \UCitem{Autor}{Adrian Flores Torres}
    \UCccitem{Evaluador}{}
    \UCitem{Operación}{Visualizar}
    \UCccitem{Prioridad}{Alta}
    \UCccitem{Complejidad}{Baja}
    \UCccitem{Volatilidad}{Baja} %Que tanto es el proceso susceptible a cambios
    \UCccitem{Madurez}{Alta}
    \UCitem{Estatus}{En revisión}
    \UCitem{Fecha del último estatus}{9 Dic 2017
    
}
    
%% Copie y pegue este bloque tantas veces como revisiones tenga el caso de uso.
%% Esta sección la debe llenar solo el Revisor
% %--------------------------------------------------------
 	\UCccsection{Revisión Versión 1.0} % Anote la versión que se revisó.
% 	% FECHA: Anote la fecha en que se terminó la revisión.
 	\UCccitem{Fecha}{} 
% 	% EVALUADOR: Coloque el nombre completo de quien realizó la revisión.
 	\UCccitem{Evaluador}{Fabiola Jaramillo Loredo}
% 	% RESULTADO: Coloque la palabra que mas se apegue al tipo de acción que el analista debe realizar.
 	\UCccitem{Resultado}{Corregir}
% 	% OBSERVACIONES: Liste los cambios que debe realizar el Analista.

% %--------------------------------------------------------

	%qué: enviar la información de los aspirantes que realizaron el pago correspondiente para presentar el examen CENEVAL
    	%quién: coordinacionControlEscolar
    	%para qué: para que CENEVAL asigne los folios a los aspirantes y puedan presentar el examen 
    	%por qué: porque desea realizar el prerregistro de aspirantes
		\UCccitem{Observaciones}{
	
	}
    \UCsection{Atributos}
    \UCitem{Actor(es)}{
    	\cdtRef{Actor:alumno}{Alumno}, \cdtRef{Actor:profesor}{Profesor},  \cdtRef{Actor:invitado}{Invitado}.
    }
    \UCitem{Propósito}{Visualizar información.}
    \UCitem{Entradas}{
	\begin{UClist}
		\UCli  \cdtRef{persona:nombre}{Nombre}, \cdtRef{persona:primerApellido}{Primer apellido}, \cdtRef{persona:segundoApellido}{Segundo apellido}. \ioObtener. Esta información se registro en la base de datos.
		\UCli \cdtRef{persona:nombre}{Nombre}. \ioEscribir.
		\UCli  \cdtRef{horario:dia}{Horario}. \ioObtener.
		\UCli  \cdtRef{contacto:telefono}{Teléfono}. \ioObtener.
		\UCli  \cdtRef{contacto:telefono}{Correo Electrónico}. \ioObtener.
	\end{UClist}
	}
    \UCitem{Salidas}{
    \begin{UClist}
    	\UCli  \cdtRef{persona:nombre}{Nombre}, \cdtRef{persona:primerApellido}{Primer apellido}, \cdtRef{persona:segundoApellido}{Segundo apellido}. \ioObtener. Esta información se registro en la base de datos.
    	\UCli \cdtRef{persona:nombre}{Nombre}. \ioEscribir.
    	\UCli  \cdtRef{horario:dia}{Horario}. \ioObtener.
    	\UCli  \cdtRef{contacto:telefono}{Teléfono}. \ioObtener.
    	\UCli  \cdtRef{contacto:telefono}{Correo Electrónico}. \ioObtener.
    \end{UClist}
}
    \UCitem{Precondiciones}{
	\begin{UClist}
	    \UCli {\bf Interna:} Que la información del profesor se encuentre registrada en la BD
	\end{UClist}
    }
    
    \UCitem{Postcondiciones}{
	Ninguno
    }

    %Reglas de negocio: Especifique las reglas de negocio que utiliza este caso de uso
    \UCitem{Reglas de negocio}{
    	Ninguno
    }
    \UCitem{Errores}{
    Ninguno.
    
  
    }
    \UCitem{Tipo}{Secundario, extiende del caso de uso \cdtIdRef{CUCP1}{Buscar profesor}}
\end{UseCase}

 \begin{UCtrayectoria}

    \UCpaso[\UCactor] Presiona el nombre de un profesor en la pantalla \cdtIdRef{IUCP1}{Realizar Busqueda}.
    \UCpaso[\UCsist]  Obtiene el nombre del profesor.
    \UCpaso[\UCsist] Obtiene la ubicación del profesor.
    \UCpaso[\UCsist] Obtiene el horario del profesor.
    \UCpaso[\UCsist]Obtiene el correo electrónico del profesor.
    \UCpaso[\UCsist]  Obtiene el teléfono del profesor.
    \UCpaso[\UCsist] Construye la pantalla \cdtIdRef{IUCP1-1 }{Visualizar información}.
    \UCpaso[\UCsist] Muestra la pantalla \cdtIdRef{IUCP1-1 }{Visualizar información}.
    \UCpaso[\UCactor] Visualiza la información del profesor.
    
\end{UCtrayectoria}



