
\begin{UseCase}{CUCP1-2}{Compartir Contacto}
    {
    	Permite a un \cdtRef{Actor:alumno}{Alumno}, \cdtRef{Actor:profesor}{Profesor} o  \cdtRef{Actor:invitado}{Invitado} compartir la información de un profesor que muestra la aplicación, esto con la finalidad proporcionar la información a otra persona que desea conocer la información y en el momento no tiene acceso a la aplicación. 
    
}
    

    \UCitem{Versión}{1.0}
    \UCccsection{Administración de Requerimientos}
    \UCitem{Autor}{Adrian Flores Torres}
    \UCccitem{Evaluador}{}
    \UCitem{Operación}{Compartir}
    \UCccitem{Prioridad}{Alta}
    \UCccitem{Complejidad}{Baja}
    \UCccitem{Volatilidad}{Baja} %Que tanto es el proceso susceptible a cambios
    \UCccitem{Madurez}{Alta}
    \UCitem{Estatus}{En revisión}
    \UCitem{Fecha del último estatus}{28 Septiembre 2017}
    
%% Copie y pegue este bloque tantas veces como revisiones tenga el caso de uso.
%% Esta sección la debe llenar solo el Revisor
% %--------------------------------------------------------
 	\UCccsection{Revisión Versión 1.0} % Anote la versión que se revisó.
% 	% FECHA: Anote la fecha en que se terminó la revisión.
 	\UCccitem{Fecha}{} 
% 	% EVALUADOR: Coloque el nombre completo de quien realizó la revisión.
 	\UCccitem{Evaluador}{Fabiola Jaramillo Loredo}
% 	% RESULTADO: Coloque la palabra que mas se apegue al tipo de acción que el analista debe realizar.
 	\UCccitem{Resultado}{Corregir}
% 	% OBSERVACIONES: Liste los cambios que debe realizar el Analista.

% %--------------------------------------------------------

	%qué: enviar la información de los aspirantes que realizaron el pago correspondiente para presentar el examen CENEVAL
    	%quién: coordinacionControlEscolar
    	%para qué: para que CENEVAL asigne los folios a los aspirantes y puedan presentar el examen 
    	%por qué: porque desea realizar el prerregistro de aspirantes
		\UCccitem{Observaciones}{
		\begin{UClist}
			% 			% PC: Petición de Cambio, describa el trabajo a realizar, si es posible indique la causa de la PC. Opcionalmente especifique la fecha en que considera razonable que se deba terminar la PC. No olvide que la numeración no se debe reiniciar en una segunda o tercera revisión.
			\RCitem{PC1}{\TOCHK{Paso 1: Este caso de uso viene de un botón que aparece en la pantalla de Evaluaciones del aspirante}}{18 de Octubre del 2017}
			\RCitem{PC2}{\TOCHK{No entiendo la trayectoria A}}{18 de Octubre del 2017}
			\RCitem{PC3}{\TOCHK{Paso 2: qué pasa si el estado del aspirante es registrado?}}{18 de Octubre del 2017}
			\RCitem{PC4}{v{Paso 7: Referenciar máquina de estados}}{18 de Octubre del 2017}
			\RCitem{PC5}{\TOCHK{Entre paso 8 y 9: Escribir que el aspirante responde la encuesta de ceneval}}{18 de Octubre del 2017}
			\RCitem{PC6}{\TOCHK{Paso 9: Link de trayectoria B roto}}{18 de Octubre del 2017}
			\RCitem{PC7}{\TOCHK{Paso 12: No muestra esa pantalla}}{18 de Octubre del 2017}
			
		\end{UClist}		
	}
    \UCsection{Atributos}
    \UCitem{Actor(es)}{
    	\cdtRef{Actor:alumno}{Alumno}, \cdtRef{Actor:profesor}{Profesor} o  \cdtRef{Actor:invitado}{Invitado}.
    }
    \UCitem{Propósito}{Compartir contacto de un profesor.}
    \UCitem{Entradas}{
	\begin{UClist}
		\UCli  \cdtRef{persona:nombre}{Nombre}, \cdtRef{persona:primerApellido}{Primer apellido}, \cdtRef{persona:segundoApellido}{Segundo apellido}. \ioObtener. Esta información se registro en la base de datos.
		\UCli \cdtRef{persona:nombre}{Nombre}. \ioEscribir.
		\UCli  \cdtRef{horario:dia}{Horario}. \ioObtener.
		\UCli  \cdtRef{contacto:telefono}{Teléfono}. \ioObtener.
		\UCli  \cdtRef{contacto:telefono}{Correo Electrónico}. \ioObtener.
	\end{UClist}
	}
    \UCitem{Salidas}{
    \begin{UClist}
    	\UCli  \cdtRef{persona:nombre}{Nombre}, \cdtRef{persona:primerApellido}{Primer apellido}, \cdtRef{persona:segundoApellido}{Segundo apellido}. \ioObtener. Esta información se registro en la base de datos.
    	\UCli \cdtRef{persona:nombre}{Nombre}. \ioEscribir.
    	\UCli  \cdtRef{horario:dia}{Horario}. \ioObtener.
    	\UCli  \cdtRef{contacto:telefono}{Teléfono}. \ioObtener.
    	\UCli  \cdtRef{contacto:telefono}{Correo Electrónico}. \ioObtener.
    	\UCli  Mensaje con información del profesor.
    \end{UClist}
}
    \UCitem{Precondiciones}{
	\begin{UClist}
	    \UCli {\bf Interna:} Que el profesor tenga información asociada.
	\end{UClist}
    }
    
    \UCitem{Postcondiciones}{
	\begin{UClist}
	    \UCli {\bf Interna:} Enviará la información del profesor.
	\end{UClist}
    }

    %Reglas de negocio: Especifique las reglas de negocio que utiliza este caso de uso
    \UCitem{Reglas de negocio}{
    Ninguno
    }
    \UCitem{Errores}{
    Ninguno.
    
  
    }
    \UCitem{Tipo}{Secundario, extiende del caso de uso \cdtIdRef{CUCP1.1}{Visualizar información}}
\end{UseCase}

 \begin{UCtrayectoria}

    \UCpaso[\UCactor] Presiona el botón \cdtButton{Compartir} en la pantalla \cdtIdRef{IUCP1-1 }{Visualizar información}.
    \UCpaso[\UCsist]  Obtiene el nombre del profesor.
    \UCpaso[\UCsist] Obtiene la ubicación del profesor.
    \UCpaso[\UCsist] Obtiene el horario del profesor.
    \UCpaso[\UCsist]Obtiene el correo electrónico del profesor.
    \UCpaso[\UCsist]  Obtiene el teléfono del profesor.
    \UCpaso[\UCsist] Construye el mensaje con la información del profesor.
    \UCpaso[\UCsist] Envía la información del profesor.
    \UCpaso[\UCsist] Muestra el mensaje \cdtIdRef{MSG1}{Operación exitosa}.
    

\end{UCtrayectoria}





