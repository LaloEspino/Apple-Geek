% !TEX root = ../../../analisis.tex

\begin{UseCase}{CUH1.3}{Modificar Materia}{
		
		Permite editar una \cdtRef{gls:unidadDeAprendizaje}{Unidad de Aprendizaje} del horario del actor para poder tener el regitro de la hora, y grupo en el que ésta se imparte actualizados.
			
}

    \UCitem{Versión}{1.0}
    \UCccsection{Administración de Requerimientos}
    \UCitem{Autor}{Eduardo Espino Maldonado}
    \UCccitem{Evaluador}{}
    \UCitem{Operación}{Editar}
    \UCccitem{Prioridad}{Alta}
    \UCccitem{Complejidad}{Baja}
    \UCccitem{Volatilidad}{Baja}
    \UCccitem{Madurez}{Alta}
    \UCitem{Estatus}{En Desarrollo}
    \UCitem{Fecha del último estatus}{10 de Diciembre del 2017}
    
    % 	\UCccsection{Revisión Versión 1.0} 
    % 	\UCccitem{Fecha}{} 
    % 	\UCccitem{Evaluador}{}
    % 	\UCccitem{Resultado}{}
    %	\UCccitem{Observaciones}{\begin{UClist}
    %		\RCitem{PC1}{\TODO{}}{fecha}
    %	\end{UClist}}
    
    \UCsection{Atributos}
    
    \UCitem{Actor(es)}{\cdtRef{Actor:Alumno}{Alumno} y \cdtRef{Actor:Profesor}{Profesor}}
    
    \UCitem{Propósito}{Proporcionar una herramienta que permita editar una unidad de aprendizaje en el horario del actor.}
    
    \UCitem{Entradas}{\begin{UClist}
    		
    		\UCli \cdtRef{unidadaprendizaje:nivel}{Nivel al cual pertenece la Unidad de Aprendizaje}: \ioSeleccionar
    		
    		\UCli \cdtRef{unidadaprendizaje:nombre}{Nombre de la Unidad de Aprendizaje}: \ioSeleccionar
    		
    		\UCli \cdtRef{horario:hora}{Hora a la cual será impartida la Unidad de Aprendizaje}: \ioSeleccionar
    		
		\UCli \textbf{Turno en el que se imparte la Unidad de Aprendizaje} \ioSeleccionar
    		
    		\UCli \cdtRef{horario:grupo}{Grupo en el cual se imparte la Unidad de Aprendizaje}: \ioSeleccionar
    		
    	\end{UClist}}
    
    \UCitem{Salidas}{\begin{UClist}
    
    		\UCli \cdtRef{unidadaprendizaje:nivel}{Nivel al cual pertenece la Unidad de Aprendizaje registrada}: \ioSeleccionar
    		
    		\UCli \cdtRef{unidadaprendizaje:nombre}{Nombre de la Unidad de Aprendizaje registrada}: \ioSeleccionar
    		
    		\UCli \cdtRef{horario:hora}{Hora a la cual será impartida la Unidad de Aprendizaje registrada}: \ioSeleccionar
    		
		\UCli \textbf{Turno en el que se imparte la Unidad de Aprendizaje registrada} \ioSeleccionar
    		
    		\UCli \cdtRef{horario:grupo}{Grupo en el cual se imparte la Unidad de Aprendizaje registrada}: \ioSeleccionar
        		
		\UCli Se muestra el mensaje \cdtIdRef{MSG1}{Operación Exitosa} en la pantalla \cdtIdRef{IUH1}{Horario}.
            
    \end{UClist}}
    
    \UCitem{Precondiciones}{\begin{UClist}
    
    	\UCli {\bf Interna:} Que exista al menos un \cdtRef{unidadaprendizaje:nivel}{Nivel} registrado en el sistema.
	
    	\UCli {\bf Interna:} Que exista al menos un \cdtRef{unidadaprendizaje:nombre}{Nombre} registrado en el sistema.

	\UCli {\bf Interna:} Que exista al menos un \textbf{Turno} registrado en el sistema.

	\UCli {\bf Interna:} Que exista al menos un \textbf{Horario} registrado en el sistema.
    	    
    \end{UClist}}
    
    \UCitem{Postcondiciones}{{\bf Interna:} Se actualizará la Unidad de Aprendizaje del Horario del actor.}
    
    \UCitem{Reglas de negocio}{\cdtIdRef{RN-N1}{Selección de Grupo}}
    
    \UCitem{Errores}{Se muestra el mensaje \cdtIdRef{MSG3}{Operación Fallida}.}
    
    \UCitem{Tipo}{Secundario, extiende de \cdtIdRef{CUH1}{Gestionar Horario}.}
    
\end{UseCase}
    
\begin{UCtrayectoria}

    \UCpaso[\UCactor] Solicita modificar una unidad de aprendizaje a su horario presionando el botón \btnEditar en la pantalla \cdtIdRef{IUH1}{Horario}.
    
    \UCpaso[\UCsist] Muestra la pantalla \cdtIdRef{IUH1-2}{Horario}.
    
    \UCpaso[\UCactor] Selecciona la unidad de aprendizaje que desea modificar presionando en botón \btnArrowRigth. \refTray{A}
    
    \UCpaso[\UCsist] Obtiene la información registrada de la unidad de aprendizaje.
        
    \UCpaso[\UCsist] Muestra la pantalla \cdtIdRef{IUH1.3}{Editar UA} con la información obtenida.
    
    \UCpaso[\UCactor] \label{CUH1.3:Registrar} Ingresa los datos solicitados. \refTray{B}
    
    \UCpaso[\UCactor] Solicita finalizar el registro de la unidad de aprendizaje presionando el botón \cdtButton{Agregar}.
    
    \UCpaso[\UCsist] Verifica que todos los datos hayan sido ingresados con base en la regla de negocio \cdtIdRef{RN-S3}{Datos requeridos}. \refTray{C}
    
    \UCpaso[\UCsist] Crea el grupo con base en la regla de negocio \cdtIdRef{RN-N1}{Selección de Grupo}.
    
    \UCpaso[\UCsist] Actualiza la información de la unidad de aprendizaje. \refTray{D}
    
    \UCpaso[\UCsist] Muestra el mensaje \cdtIdRef{MSG1}{Operación Exitosa} en la pantalla \cdtIdRef{IUH1}{Horario}.
        
\end{UCtrayectoria}

% Trayectorias Alternas

%-------------------------Trayectoria A------------------------------
\begin{UCtrayectoriaA}[Fin del caso de uso]{A}{El actor desea cancelar la operación}
	
	\UCpaso[\UCactor] Presiona el botón \btnEditar.
	
	\UCpaso[\UCsist] Muestra la pantalla \cdtIdRef{IUH1}{Horario}.
		
\end{UCtrayectoriaA}

%-------------------------Trayectoria B------------------------------
\begin{UCtrayectoriaA}[Fin del caso de uso]{B}{El actor desea cancelar la operación}
	
	\UCpaso[\UCactor] Presiona el botón \cdtButton{Atrás}.
	
	\UCpaso[\UCsist] Muestra la pantalla \cdtIdRef{IUH1}{Horario}.
		
\end{UCtrayectoriaA}

%-------------------------Trayectoria C------------------------------
\begin{UCtrayectoriaA}[Fin de la trayectoria]{C}{El actor no ingreso un dato obligatorio.}
		
	\UCpaso[\UCsist] Muestra el mensaje \cdtIdRef{MSG6}{Faltan datos obligatorios}
	
	\UCpaso Continua en el paso \ref{CUH1.3:Registrar} de la trayectoria principal.
		
\end{UCtrayectoriaA}

%-------------------------Trayectoria D------------------------------
\begin{UCtrayectoriaA}[Fin de la trayectoria]{C}{La operación no se pudo completar.}
		
	\UCpaso[\UCsist] Muestra el mensaje \cdtIdRef{MSG3}{Operación Fallida}
		
	\UCpaso Continua en el paso \ref{CUH1.3:Registrar} de la trayectoria principal.
		
\end{UCtrayectoriaA}
